\documentclass[a4paper,10pt,oneside]{article}
\usepackage[utf8x]{inputenc}
\usepackage{amsmath}
\usepackage{amsthm}
\usepackage{amssymb}
\usepackage{hhline}
\usepackage{amsfonts}
\usepackage[italian]{babel}
\usepackage{hyphenat}
\usepackage[linktoc=all]{hyperref}
\newcommand\restr[2]{{% we make the whole thing an ordinary symbol
  \left.\kern-\nulldelimiterspace % automatically resize the bar with \right
  #1 % the function
  \vphantom{\big|} % pretend it's a little taller at normal size
  \right|_{#2} % this is the delimiter
  }}
%Ricordarsi di cambiare la lingua e mettere la sillabazione

\DeclareMathOperator{\cof}{cof}
\DeclareMathOperator{\im}{Im}
\DeclareMathOperator{\Fun}{Fun}

\newcommand{\trash}{76}
\newcommand{\myname}[1]{\emph{#1}}
\newcommand{\nin}{\not\in}
\newcommand{\parti}[1]{\mathcal{P}(#1)}
\newcommand{\rel}{\mathcal R}
\newcommand{\abs}[1]{\left|#1\right|}

%TEOREMI IN CORSIVO
\theoremstyle{plain}
\newtheorem{mytheorem}{Teorema}[section]
\newtheorem{mydef}[mytheorem]{Definizione}
\newtheorem{mylemma}[mytheorem]{Lemma}
\newtheorem{myprop}[mytheorem]{Proposizione}

\newtheorem{myax}[mytheorem]{Assioma}


%TEOREMI NON IN CORSIVO
\theoremstyle{definition}
\newtheorem{myex}{Esercizio}
\newtheorem{mycor}[mytheorem]{Corollario}
%Note
\theoremstyle{remark}
\newtheorem*{myobs}{Osservazione}


\setcounter{tocdepth}{4}


%\setlength{\parskip}{\baselineskip}%
\setlength{\parindent}{0pt}%

\title{Elementi di Teoria degli Insiemi}
\author{Dario Ascari \and Matteo Migliorini}
\date{}


\begin{document}


\maketitle

\cleardoublepage

%\section*{Nota}
%Questo file è pensato per essere una sintesi dei concetti e delle idee necessarie per affrontare l'esame di Elementi di Teoria degli Insiemi: sono per questo riportate tutte le definizioni, teoremi e fatti che è utile sapere. Tuttavia questo esso non deve essere inteso come un documento utile per ripassare, ma come uno stru

\cleardoublepage

\tableofcontents
\cleardoublepage




\section{Classi e insiemi}

Una \myname{classe} è una collezione di oggetti. 

\begin{myax}[Assioma di estensionalità]
 Due classi sono uguali se hanno gli stessi elementi.
\end{myax}
\begin{myax}[Schema di assiomi di astrazione]
 Data una proprietà ben definita $P(x)$, esiste la classe $\{x: P(x)\}$
\end{myax}

Una classe puo essere di due tipi:

\begin{itemize}
 \item un \myname{insieme}, se è anche un oggetto;
 \item una \myname{classe propria}, se non è un insieme.
\end{itemize}

L'idea di questa distinzione è di evitare di cadere in paradossi come quello di Russel.
\begin{myprop}[Classe di Russel]
  La classe $R=\{x:x\not \in x\}$ non è un insieme.
\end{myprop}
\begin{proof}
 Se fosse un insieme, o $R\in R$ o $R \nin R$. Ma per definizione, $R\in R \Leftrightarrow R \nin R$, assurdo.
\end{proof}

Noi adesso possiamo dimostrare che alcune classi sono classi proprie (ad esempio come abbiamo fatto per la classe di Russel). Tuttavia abbiamo bisogno di assiomi costruttivi per garantire l'esistenza di insiemi.

\begin{myax}[Assioma di comprensione]
 Una sottoclasse di un insieme è un insieme.
\end{myax}
\begin{myax}[Insieme vuoto]
 La classe vuota è un insieme e si denota con $\emptyset$. 
\end{myax}
\begin{myax}[Assioma della coppia]
 Dati due oggetti $a$ e $b$, non necessariamente distinti, la classe $\{a,b\} = \{x:x=a \lor x=b\}$ è un insieme.
\end{myax}
\begin{myax}[Assioma dell'unione]
 Se $X$ è un insieme di insiemi, allora $\bigcup_{A\in X} A$ è un insieme.
\end{myax}
\begin{myax}[Assioma di Bagonzo]
 Ogni oggetto è un insieme.
\end{myax}
\begin{myax}[Assioma di potenza]
 Dato un insieme $A$, $\mathcal{P}(A)$ è un insieme.
\end{myax}

Ora serve definire le nozioni di coppia ordinata, prodotto cartesiano e relazione. 

\begin{mydef}[Coppia di Kuratowski]
 Se definisce la \myname{coppia ordinata} $(a,b) = \{\{a\},\{a,b\}\}$.
\end{mydef}
La coppia di Kuratowski è un insieme per l'assioma della coppia. Inoltre dalla definizione e dall'assioma di estensionalità vale $(a,b)=(c,d) \Leftrightarrow a=c \lor b=d$.

\begin{mydef}[Prodotto cartesiano]
 Si definisce il prodotto cartesiano tra classi $A\times B := \{(a,b): a\in A, b\in B\}$
\end{mydef}

Si dimostra che, se $A$ e $B$ sono insiemi, $A\times B$ è un insieme, dato che $A\times B \subseteq \parti{\parti{A\cup B}}$

\begin{mydef}[Relazione]
 Una \myname{relazione} $\rel$ tra $A$ e $B$ è un sottoinsieme del prodotto cartesiano $A\times B$. Si dice che $a\rel b \Leftrightarrow (a,b) \in \rel$.
\end{mydef}

\begin{mydef}[Relazione di equivalenza]
 Una relazione $\rel \subseteq A\times A$ è detta di equivalenza se valgono le proprietà:
 \begin{itemize}
  \item riflessiva, cioè per ogni $a\in A$ vale $a\rel a$;
  \item simmetrica, cioè $a \rel b \Leftrightarrow b \rel a$;
  \item transitiva, cioè $a \rel b \land b \rel c \Rightarrow a \rel c$.
 \end{itemize}
\end{mydef}

\begin{mydef}[Relazione d'ordine]
 Una relazione $\rel \subseteq A\times A$ è detta d'ordine se valgono le proprietà:
 \begin{itemize}
  \item riflessiva, cioè per ogni $a\in A$ vale $a\rel a$;
  \item antisimmetrica, cioè $a \rel b \land b \rel a\Rightarrow a = b$;
  \item transitiva, cioè $a \rel b \land b \rel c \Rightarrow a \rel c$.
 \end{itemize}
\end{mydef}

Normalmente si indica $x \preceq y$ se $x \rel y$ (o con altri simboli analoghi).

\begin{mydef}[Relazione d'ordine stretto]
 Una relazione $\rel \subseteq A\times A$ è detta d'ordine stretto se valgono le proprietà:
 \begin{itemize}
  \item asimmetrica, cioè $a \prec b \Rightarrow \lnot (b \prec a)$;
  \item transitiva, cioè $a \prec b \land b \prec c \Rightarrow a \prec c$.
 \end{itemize}
 dove usiamo $\prec$ per ricordare che è un ordine stretto.
\end{mydef}
Nelle relazioni di ordine stretto, la proprietà asimmetrica implica la proprietà irriflessiva, ossia $\forall a \in A \;\lnot (a \rel a)$.

\begin{mydef}
 Un ordine si dice totale se per ogni $x,y \in A$ vale $x \preceq y \lor y \preceq x \lor x=y$.
\end{mydef}
\begin{mydef}
 Una relazione d'ordine totale si dice \myname{buon ordine} se ogni sottoinsieme non vuoto di $A$ ammette minimo.  
\end{mydef}
\begin{mydef}
 Si dice che $X\subseteq A$ è un \myname{segmento iniziale} di $A$ se per ogni $x\in X$ vale l'implicazione \[a\preceq x \Rightarrow a \in X\]
\end{mydef}

Se $(A,\prec)$ è un buon ordine, i suoi segmenti iniziali propri sono in corrispondenza biunivoca con gli elementi di $A$: dato $a\in A$ si indica con $A_{a}=\{x\in A: x\prec a\}$ il segmento iniziale degli elementi minori di $a$, e ogni segmento iniziale proprio è di questa forma (basta prendere il minimo dei maggioranti, che esiste perchè è un buon ordine). Inoltre se $X,Y\subseteq A$ sono segmenti iniziali, allora $X\subseteq Y \vee Y\subseteq X$.

\begin{mydef}[Relazione funzionale]
 Una relazione $\rel \subseteq A\times B$ si dice funzionale se \[\forall a \in A \exists! b \in B: a\rel b\] 
 
 Detta $f$ tale relazione, l'unico $b$ in relazione con $a$ si denota con $f(a)$.
 
 L'insieme $A$ si dice \myname{dominio} della funzione, mentre l'$B$ si dice \myname{codominio}. Inoltre si definisce l'immagine di un sottoinsieme $X$ del dominio come $f(X)=\{f(x):x \in X\}$, mentre si definisce la controimmagine di un sottoinsieme $Y$ del codominio come $f^{-1}(Y)=\{x\in A: f(x)\in Y\}$. Inoltre si denota con $\im f = f(A)$.
\end{mydef}

\textbf{Achtung!}\texttrademark \ Due funzioni possono essere uguali (come insiemi) nonostante abbiano un codominio diverso!\vspace{6px}

Convenzionalmente, si chiama \myname{funzione} una relazione funzionale tra insiemi, mentre per ricordare che dominio e/o codominio potrebbero non essere insiemi si usa il termine \myname{funzione classe}.

\begin{myax}[Schema di assiomi di rimpiazzamento]
Data una proprietà $P(C,D)$, se $P$ è funzionale (cioè per ogni $C$ esiste un unico $D$ tale che $P(C,D)$) allora dato un qualunque insieme $A$ esiste un unico insieme $B$ che contiene tutte e sole le immagini degli elementi di $A$ secondo $P$.
\end{myax}
In pratica, data una funzione definita da una formula da un insieme ad una classe, anche $\im f$ è un insieme.

Inoltre, anche se in seguito non verrà usato, si può assumere il seguente
\begin{myax}[Assioma di buona fondazione]
 Per ogni insieme $X$, esiste $a \in X$ tale che $A \cap x = \emptyset$.
\end{myax}

\begin{myobs}
 Sono definite l'intersezione e l'unione di una classe di insiemi; l'unione di una classe di insieme potrebbe non essere un insieme, mentre l'intersezione lo è sempre per l'assioma di comprensione (sempre che la classe sia non vuota!).
\end{myobs}


\subsection{Assioma dell'infinito e numeri naturali}

Vogliamo definire in modo formale i numeri naturali. Innanzitutto ci serve la funzione successore, definita da $S(X)=X\cup\{X\}$. 
\begin{mydef}[Insiemi $S$-saturi]
Un insieme $A$ si dice $S$-saturo se $\emptyset \in A$ e $A$ è chiuso per successore, cioè $X\in A \Rightarrow S(X)\in A$.
\end{mydef}
\begin{myax}[Assioma dell'infinito]
 La classe $\mathcal S$ degli insiemi $S$-saturi è non vuota.
\end{myax}

\begin{myprop}
 L'intersezione di insiemi $S$-saturi è $S$-satura.
\end{myprop}

Si definisce quindi l'insieme dei numeri naturali $\mathbb N = \bigcap_{X\in \mathcal S} X$. 

Intuitivamente con questa definizione ogni numero naturale $n$ viene identificato come l'insieme dei naturali più piccoli di lui, e la relazione d'ordine tra i naturali non è nient'altro che l'appartenenza: vedremo tra poco come tutto ciò si generalizza agli ordinali.

\begin{mytheorem}[Induzione]
 Se una proprietà $P(X)$ è tale che $P(0)$ e per ogni $X$ vale $P(X)\Rightarrow P(S(X))$, allora $P(n)\;\forall n\in \mathbb N$
\end{mytheorem}
\begin{proof}
 La classe $\{x: P(x)\}$ è $S$-satura, quindi si può dimostrare che contiene $\mathbb N$ (cfr Esercizio \ref{ex:classisature}).
\end{proof}


\section{Ordinali}

\begin{mydef}
 Un insieme $A$ si dice \myname{transitivo} se per ogni $x,y$ vale l'implicazione 
 \[x\in y \in A \Rightarrow x\in A\]
\end{mydef}

\begin{mydef}
Un insieme $\alpha$ si dice \myname{ordinale} se è transitivo e bene ordinato da $\in$ (come relazione stretta).
\end{mydef}

Chiameremo $ON$ la classe di tutti gli ordinali.

Iniziamo a dimostrare alcuni fatti base sugli ordinali. Innanzitutto per ogni $\alpha$ ordinale, $\alpha \not\in \alpha$: infatti la relazione di appartenenza è una relazione d'ordine stretto, e quindi vale la proprietà antiriflessiva.

\begin{myprop}\label{prop:xinord}
Se $\alpha$ è un ordinale, allora ogni $x\in \alpha$ è un ordinale.
\end{myprop}
\begin{proof}
 Utilizzando la transitività di $\alpha$, si ottiene che $z\in y \in x \in \alpha\Rightarrow z,y,x\in \alpha$. A questo punto poichè in $\alpha$ l'appartenenza è un ordine totale, in particolare vale la proprietà transitiva e quindi $z\in x$: quindi $x$ è transitivo. Inoltre $x$ è un sottoinsieme di $\alpha$ che è un buon ordine, quindi anche $x$ è bene ordinato.
\end{proof}

\begin{myprop}
Sono fatti equivalenti, per ogni $\alpha$ e $ \beta$ ordinali:
\begin{enumerate}
 \item $\alpha \in \beta$;
 \item $\alpha \subset \beta$;
 \item $\alpha$ è un segmento iniziale proprio di $\beta$.
\end{enumerate}
\end{myprop}
\begin{proof}
 $1. \Rightarrow 2.$ deriva dalla transitività e da $\alpha \nin \alpha$. 
 
 $2. \Rightarrow 3.$ è banale (basta scrivere la definizione di segmento iniziale per accorgersene).
 
 $3. \Rightarrow 1.$ Sia $\gamma = \min\{x\in\beta: a\in x \; \forall a\in \alpha \}$: l'insieme considerato è non vuoto (perchè $\alpha$ è segmento iniziale proprio di $\beta$) ed il minimo esiste per il buon ordinamento. 
 
 Vogliamo ora mostrare che $\alpha = \gamma$. $\alpha \subseteq \gamma$ è ovvio, dalla definzione di $\gamma$. Supponiamo esista $\delta\in\gamma$ tale che $\delta \nin \alpha$. Allora per ogni $a \in \alpha$ vale $a \in \delta$ (perchè $\alpha$ è un segmento iniziale). Ma $\delta \in \gamma$, quindi $\gamma$ non era il minimo, assurdo.
\end{proof}


\begin{myprop}
L'intersezione di una classe non vuota di ordinali è un ordinale.
\end{myprop}
\begin{proof}
 Sia $X=\bigcap_{i\in I}A_i$, e siano $x,y$ tali che $x\in y \in X$. Allora per ogni $i$ vale $x\in y \in A_i \Rightarrow x\in A_i$. Poichè $x$ sta in tutti gli $A_i$, allora sta nell'intersezione. 
 
 Il buon ordinamento segue da $X\subset A_i$.
\end{proof}
\begin{myprop}[Tricotomia degli ordinali]
 Per ogni $\alpha$ e $\beta$ ordinali \[\alpha \in \beta \vee \beta \in \alpha \vee \alpha=\beta\]
\end{myprop}
\begin{proof}
 Sia $\gamma = \alpha \cap \beta$. Quindi $\gamma$ è un ordinale: se $\gamma \subset \alpha$ e $\gamma \subset \beta$, allora $\gamma \in \alpha$ e $\gamma \in \beta$, quindi $\gamma \in \alpha \cap \beta = \gamma$, assurdo. Quindi l'intersezione coincide con uno tra $\alpha$ e $\beta$, e da qui si conclude.
\end{proof}

\begin{myprop}
 Se $X\ne \emptyset$ è un insieme di ordinali, allora $M = \sup X=\bigcup_{\gamma \in X}\gamma$ è un ordinale.
\end{myprop}
\begin{proof}
 La transitività è banale, e il fatto che $\in$ sia un ordine totale deriva dalla tricotomia.
 
 Per dimostrare che $\in$ è un buon ordine, prendiamo un sottoinsieme $Y \subseteq M$ non vuoto, e mostriamo che ha minimo. Consideriamo $\bar\gamma \in X$ tale che $\bar\gamma \cap Y \ne \emptyset$ (perchè esiste un tale $\bar\gamma$?). Sia $m=\min (\bar\gamma\cap Y)$, che esiste perchè $\bar\gamma$ è un buon ordine. Vogliamo mostrare che $m=\min Y$.
 
 Se esistesse $y\in Y$ tale che $y\in m$, allora $y\in m \in \bar\gamma$, quindi dalla transitività di $\bar\gamma$ otteniamo $y \in \bar\gamma\cap Y$, contraddicendo la minimalità di $m$.
\end{proof}

\begin{myprop}
  Il successore di $\alpha$, ovvero $\alpha \cup \{\alpha\}$, è un ordinale.
\end{myprop}


\begin{myobs}
 Si dimostra per induzione che i naturali sono ordinali, e lo è anche la loro unione (che è un insieme per l'assioma dell'infinito e quello dell'unione), che chiameremo $\omega$.
\end{myobs}

\begin{mylemma}\label{prop:fcresc}
 Se $\alpha$ è un ordinale, e $f:\alpha \rightarrow ON$ è una funzione classe crescente, allora $f(x)\ge x$ per ogni $x\in \alpha$.
\end{mylemma}
\begin{proof}
 Sia $\gamma = \min{\delta \in \alpha: f(\delta)<\delta}$. Allora $f(f(\delta))<f(\delta)$, quindi anche $f(\delta)$ stava nell'insieme e vale $f(\delta)<\delta$, contraddicendo la minimalità di $\delta$. Quindi quell'insieme è vuoto.
\end{proof}

\begin{myprop}
 Due ordinali isomorfi (rispetto alla relazione d'ordine $\in$) sono uguali.
\end{myprop}
\begin{proof}
 Basta applicare il lemma \ref{prop:fcresc}: una funzione crescente non può andare da un ordinale ad uno più piccolo.
\end{proof}



\subsection{Induzione e ricorsione transfinite}

\begin{mytheorem}[Induzione transfinita]
 
Sia $(A,\prec)$ un buon ordine e $P(x)$ una proprietà tale che \[\forall a\in A ((\forall x\prec b: P(x)) \Rightarrow P(b))\]

Allora $\forall a \in A$ vale $P(a)$.
\end{mytheorem}
\begin{proof}
 Basta considerare il minimo $a\in A$ tale che la proprietà non è verificata per ottenere un assurdo.
\end{proof}
In particolare, sulla classe $ON$ degli ordinali basta verificare che valga:
\begin{itemize}
 \item $P(0)$;
 \item $P(\alpha)\Rightarrow P(\alpha+1);$
 \item Se $\lambda$ è un ordinale limite, allora $(\forall \gamma<\lambda\; P(\gamma))\Rightarrow P(\lambda)$.
\end{itemize}

\begin{mytheorem}[Ricorsione transfinita]
 Dato un buon ordine $(A,\prec)$, si può definire una funzione $f:A \rightarrow V$ ($V$ è una classe qualunque) dando il valore in un generico punto $x$ in funzione di $x$ e di tutti i valori assunti dalla funzione negli $y\prec x$. In particolare si può definire una funzione da un ordinale in $V$ specificando il valore assunto in $\alpha$ in funzione di $\alpha$ e dei valori assunti in tutti i $\delta \in \alpha$.
\end{mytheorem}
\begin{proof}
 L'idea è costruirsi delle funzioni parziali, che partano dai segmenti iniziali di $A$, dimostrare che sono ``coerenti'' tra di loro (cioè contenute l'una nell'altra) e unirle tutte per ottenere $f$.
 
 Dato $X$ segmento iniziale di $A$, supponiamo che esistano due funzioni distinte $g_1 , g_2: X \rightarrow V$ che verificano la ricorsione: considero il minimo $x\in X$ in cui le funzioni assumono valori distinti e trovo un assurdo.
 
 Quindi per ogni segmento iniziale esiste al più una funzione definita su quel segmento che soddisfa la ricorsione.
 Inoltre date $g_X$ e $g_Y$ con $X\subseteq Y \subseteq A$ segmenti iniziali, la funzione $\restr{g_Y}X$ ($g_Y$ ristretta ad $X$) soddisfa ancora la ricorsione (ed è definita su $X$), quindi per quanto detto prima $\restr{g_Y}X=g_X$, quindi $g_X\subseteq g_Y$.
 
 Posso dunque definire la funzione $f:=\bigcup g_X$ dove l'unione è intesa su tutti i segmenti iniziali $X$ tali che esiste una $g_X$: tale funzione è definita su tutto $A$ (altrimenti prendo il minimo $a \in A$ tale che $f$ non è definita su $a$, e costruisco con la ricorsione la funzione $g_{A_a \cup \{a\}}$, che per definizione dovrebbe essere sottoinsieme di $f$).
\end{proof}

Con questo si può dimostrare la seguente importante proposizione.

\begin{myprop}\label{prop:isomord}
 Ogni buon ordine è isomorfo ad esattemente un ordinale.
\end{myprop}
\begin{proof}
 Dato $(A,\prec)$ definisco per ricorsione transfinita $f:A\rightarrow C$ ($C$ qui indica la classe di tutti gli insiemi) usando come legge di ricorsione $f(a):=\{f(x): x \prec a\}$: è facile verificare per induzione transfinita che $f(a)$ è un ordinale $\forall a\in A$. Quindi $\im f$ è un ordinale. Il fatto che $(A,\prec)$ non sia isomorfo a più di un ordinale deriva dal fatto che due ordinali distinti non sono fra loro isomorfi.
\end{proof}

\subsection{Operazioni con gli ordinali}
\begin{mydef} Si definisce la somma di due ordinali per ricorsione transfinita sul secondo termine:
\begin{itemize}
   \item $\alpha + 0 = \alpha$
   \item $\alpha + (\beta+1)=(\alpha+\beta)+1$
   \item $\alpha + \lambda= \bigcup_{\gamma<\lambda}(\alpha+\gamma)$  
\end{itemize}
per ogni $\alpha, \beta$ ordinali e per ogni $\lambda$ ordinale limite.
\end{mydef}

Si dimostrano (per lo più per induzione) i seguenti fatti (dimostrateli!):
\begin{itemize}
 \item $0+\alpha=\alpha$;
 \item $\alpha \le \beta \Rightarrow \alpha+\gamma \le \beta + \gamma$;
 \item $\alpha \le \beta \Leftrightarrow \gamma+\alpha\le \gamma + \beta$;
 \item $\alpha + \bigcup_{\gamma \in X}\gamma = \bigcup_{\gamma \in X} (\alpha+\gamma)$;
 \item $(\alpha + \beta)+\gamma= \alpha + (\beta+\gamma)$. \qquad \qquad
\end{itemize}

\begin{mydef} Si definisce il prodotto di due ordinali in modo analogo per ricorsione sul secondo termine:
\begin{itemize}
   \item $\alpha \cdot 0 = 0$
   \item $\alpha \cdot (\beta+1) =\alpha\cdot\beta+\alpha$
   \item $\alpha \cdot \lambda= \bigcup_{\gamma<\lambda}(\alpha\cdot\gamma)$  
\end{itemize}
\end{mydef}

Si dimostrano come sopra i seguenti fatti:
\begin{itemize}
 \item $0\cdot\alpha=0$;
 \item $1\cdot \alpha =\alpha$;
 \item $\alpha \le \beta \Rightarrow \alpha\cdot\gamma \le \beta \cdot \gamma$;
 \item $\alpha \le \beta \Leftrightarrow \gamma\cdot\alpha\le \gamma \cdot \beta$;
 \item $\alpha \cdot \bigcup_{\gamma \in X}\gamma = \bigcup_{\gamma \in X} \alpha\cdot\gamma$;
 \item $\alpha\cdot(\beta+\gamma)=\alpha\cdot\beta+\alpha\cdot\gamma$;
 \item $(\alpha \cdot \beta)\cdot\gamma= \alpha \cdot (\beta\cdot\gamma)$. 
\end{itemize}

\begin{mydef} Si definisce l'esponenziazione di due ordinali in modo analogo per ricorsione sul secondo termine:
\begin{itemize}
   \item $\alpha^0 = 1$
   \item $\alpha^{\beta+1} =\alpha^\beta\cdot\alpha$
   \item $\alpha^\lambda= \bigcup_{\gamma<\lambda}\alpha^\gamma$  
\end{itemize}
\end{mydef}

Analogamente a sopra valgono i seguenti fatti:
\begin{itemize}
 \item $\alpha \le \beta \Rightarrow \alpha^\gamma \le \beta^\gamma$;
 \item $\alpha \le \beta \Leftrightarrow \gamma^\alpha\le \gamma^\beta$ per ogni $\gamma > 1$;
 \item $\alpha^{\bigcup_{\gamma \in X}\gamma} = \bigcup_{\gamma \in X} \alpha^\gamma$;
 \item $\alpha^{(\beta+\gamma)}=\alpha^\beta\cdot\alpha^\gamma$;
 \item $(\alpha^\beta)^\gamma=\alpha^{\beta\gamma}$.
\end{itemize}
\noindent


\subsection{Differenza e divisione con resto}
\begin{myprop}
Per ogni $\alpha \le \beta$ ordinali, esiste un unico $\gamma$ tale che $\alpha + \gamma=\beta$.
\end{myprop}

\begin{myprop}[Divisione euclidea]
 Per ogni $\alpha, \beta$, con $\beta \ne 0$, esistono unici $\xi, \rho$ ordinali tali che \[\alpha = \beta \xi + \rho, \qquad \rho < \beta.\]
\end{myprop}

\begin{myprop}[Forma normale di Cantor]
 Per ogni $\alpha,\beta$ con $\beta \ne 0$, esistono unici $k<\omega, \gamma_1>\gamma_2>\dots>\gamma_k$ e $\delta_1,\delta_2,\dots\,\delta_k$ con $0<\delta_i<\beta$ tali che 
 \[\alpha = \beta^{\gamma_1}\cdot \delta_1 + \dots +\beta^{\gamma_k}\cdot \delta_k\]
\end{myprop}

\subsection{Interpretazioni}
\begin{myprop}
 L'ordinale $\alpha + 
\noindent
\beta$ è isomorfo all'insieme bene ordinato 
 \[\alpha \sqcup \beta= \alpha \times \{0\} \cup \beta \times \{1\}\]
 dove $\forall a\in \alpha \times\{0\}, \forall b \in \beta \times \{1\}$ vale $b>a$.
\end{myprop}

\begin{myprop}
 L'ordinale $\alpha \cdot \beta$ è isomorfo all'insieme bene ordinato $\alpha \times \beta$, con la relazione d'ordine \myname{lessicografica}, cioè si confrontano prima le seconde componenti e in caso di uguaglianza si confrontano le prime.
\end{myprop}

La forma normale di Cantor non è altro che la scrittura di un ordinale $\alpha$ in base $\beta$.

\begin{myprop}
 L'ordinale $\alpha^\beta$ è isomorfo all'insieme delle funzioni $f:\beta \rightarrow \alpha$ \emph{a supporto finito}, ossia sempre nulle ad eccezione di un numero finito di valori. Per confrontare due funzioni si guarda la massima componente sulla quale sono diverse, cioè si confrontano $f(\gamma)$ e $g(\gamma)$, con $\gamma := \max\{\delta < \beta: f(\delta)\ne g(\delta)\}$. Il $\max$ esiste perché le funzioni sono a supporto finito.
\end{myprop}

Nota. La funzione a supporto a finito non è nient'altro che la la rappresentazione di un ordinale $\gamma < \alpha^\beta$ scritto in forma normale di Cantor in base $\alpha$: $f(b)$, con $b< \beta$, è il coefficiente del termine $\alpha^b$. Il fatto che la funzione sia a supporto finito corrisponde al fatto che nella forma normale di Cantor ci sono un numero finito di termini.

\subsection{Ulteriori proprietà delle operazioni tra ordinali}

In base al tipo di ordinale di $\alpha$ e $\beta$ (limite o successore) si può stabilire il tipo di $\alpha + \beta$, $\alpha \beta$, $\alpha^\beta$. I risultati ottenuti si possono sintetizzare nella tabella seguente.

\begin{center}
\begin{tabular}{|c||c|c|c|c|}
\hline 
$\alpha$ & S & S & L & L \\
\hline
$\beta$ & S & L & S & L \\
\hline \hline
$\alpha+\beta$ & S & L & S & L \\
\hline 
$\alpha \beta$ & S & L & L & L \\
\hline
$\alpha^\beta$ &   & L & L & L \\
\hline
\end{tabular}
\end{center}

Per gli esponenziali vale $\alpha^\beta$ successore se e solo se $\alpha$ è successore e $\beta$ è finito.

\subsection{Funzioni crescenti e punti fissi}

\begin{mydef}
 Una funzione definita da un ordinale o da $ON$ si dice \myname{continua} se per ogni ordinale limite $\lambda$ nel dominio vale $f(\lambda)=\bigcup_{\gamma\in\lambda}f(\gamma)$.
\end{mydef}

\begin{myprop}
 Data una funzione $f:\alpha \rightarrow ON$ strettamente crescente,per ogni $\beta\in\alpha$ vale $f(\beta)\ge\beta$.
\end{myprop}
\begin{proof}
 Prendiamo per assurdo il minimo $\bar\beta\in\alpha$ tale che $f(\bar\beta)<\bar\beta$: vale $f(f(\bar\beta))<f(\bar\beta)$, quindi $\bar\beta$ non era il minimo a soddisfare quella proprietà...
\end{proof}

\begin{mytheorem}
 Data una funzione $f:ON \rightarrow ON$ crescente e continua, ammette punti fissi (ovvero dei $\gamma$ tali che $f(\gamma)=\gamma$). Inoltre se è strettamente crescente ammette punti fissi grandi a piacere.
\end{mytheorem}
\begin{proof}
 L'idea è che dato un qualunque $\beta\in ON$ si può costruire per ricorsione un punto fisso $\beta_\omega$ considerando $\beta_0:=\beta$, $\beta_{n+1}:=f(\beta_n)$ e $\beta_\omega:=\bigcup_{i\in\omega}\beta_i$.
\end{proof}

Notare che, quando conosceremo il concetto di cofinalità, si potrà ripetere il ragionamento per funzioni che vanno da un qualunque ordinale di cofinalità più che numerabile in se stesso (vedi esercizio \ref{ex:puntifissi}).


\section{Assioma di scelta}

\begin{myax}[Assioma di scelta]
 Dato un insieme $X$ i cui elementi sono insiemi non vuoti a due a due disgiunti, esiste un insieme $S$ che interseca ogni elemento di $X$ in esattamente un elemento.
\end{myax}
\begin{myax}[Assioma di scelta (forma equivalente)]
 Dato in insieme $I$ di indici e una famiglia $\langle A_i : i \in I\rangle$ di insiemi, esiste una funzione $f:I \rightarrow \bigcup_{i\in I}A_i$ tale che $f(i)\in A_i$ per ogni $i\in I$.
\end{myax}

I seguenti enunciati sono equivalenti all'assioma di scelta:

\begin{myprop}[Teorema di Zermelo]
 Ogni insieme $X$ ammette un buon ordine. 
\end{myprop}
\begin{myprop}[Lemma di Zorn]
 Sia $(S,\preceq)$ un insieme parzialmente ordinato non vuoto e tale che ogni suo sottoinsieme totalmente ordinato (detto \myname{catena}) ammetta un maggiorante
 (cioè dato $X \subseteq S$ totalmente ordinato esiste $s\in S$ tale che $x\preceq s$ per ogni $x\in X$). Gli insiemi siffatti si dicono \myname{induttivi}. Allora esiste in $S$ un elemento massimale
 (cioè un elemento $\bar s$ tale che $s \preceq \bar s$ per ogni $s\in S$).
\end{myprop}

\begin{proof}\
\begin{itemize}
 \item Assioma di scelta $\Rightarrow$ Lemma di Zorn (prima dimostrazione).
 
 L'idea è più o meno la stessa alla base della dimostrazione di ricorsione transfinita: fisso una funzione di scelta, considero le catene ``coerenti'' con la funzione di scelta, dimostro che sono contenute una nell'altra e le unisco tutte.
 
 Dato $(S,\preceq)$ parzialmente ordinato, per assioma di scelta esiste una una funzione di scelta $f: \mathcal P (S)\setminus{\emptyset} \rightarrow S$ che associa ad ogni sottoinsieme di $S$ un suo elemento. Diciamo ora che $X\subseteq A$ è una \myname{f-catena} se è una catena bene ordinata e $\; \forall x\in X \quad x=f(\{s\in S: s\succ y \quad \forall y\in X_x\})$
 
 Di fatto sto imponendo che per ogni elemento $x$ della catena, questo sia stato scelto dalla funzione $f$ tra tutti gli elememti papabili dopo aver fissato gli tutti quelli precedenti (cioè quelli di $X_x$).
 
 Date due \myname{f-catene} $X$ e $Y$, ho che $X\subseteq Y \vee Y\subseteq X$. Infatti, essendo buoni ordini, devono essere uno isomorfo ad un segmento iniziale dell'altro (vedi \ref{prop:isomord}): diciamo che $X$ è isomorfo ad un segmento iniziale proprio di $Y$ tramite un isomorfismo $\phi$: sia per assurdo $\bar x\in X$ il minimo tale che $\phi(\bar x)\not = \bar x$; ma allora $\forall x\in X_{\bar x} \quad \phi(x)=x \;$ e $\; X_{\bar x}=Y_{\phi(\bar x)}$. Quindi $\quad\bar x=f(\{s\in S:s\succ x \quad \forall x\in X_{\bar x}\})=f(\{s\in S:s\succ y \quad \forall y\in Y_{\phi(\bar x)}\})=\phi(\bar x)$, assurdo.
 
 Quindi l'unione $U$ di tutte le \myname{f-catene} di $S$ è una \myname{f-catena}. Se $U$ ammettesse un maggiorante stretto, potrei aggiungere usando $f$ un elemento ad $U$, ottenendo una \myname{f-catena} non sottoinsieme di $U$, assurdo. Quindi prendo il maggiorante $u$ di $U$ che esiste per ipotesi, e questo è anche un elemento massimale.
 
 \item Assioma di scelta $\Rightarrow$ Lemma di Zorn (seconda dimostrazione).
 
 Supponiamo per assurdo che $S$ non abbia massimale. Notiamo innanzitutto che ogni catena ha maggiorante stretto: infatti ha un maggiorante $M$, e se non esistesse $N>M$ allora $M$ sarebbe massimale.
	
 Vorrei ora dimostrare che posso costruire una funzione iniettiva dagli ordinali a $S$, che porterebbe a un assurdo. L'idea è costruirsela per ricorsione controllando che ad ogni passo gli elementi precedenti siano mandati in una catena. Se ciò non accade faccio assumere alla funzione un valore ``cestino''.
 Per farlo definiamo una funzione $g: ON \rightarrow S \sqcup \{\trash\}$ tale che
 \[g(\alpha)=\left\{
 \begin{array}{l}
 f(\{s\in S:s>g(\gamma)\; \forall \gamma < \alpha\}) \textrm{ se } \{g(\gamma): \gamma < \alpha\} \textrm{ è una catena;}\\
 \trash \textrm{ altrimenti.}
 \end{array}
 \right.	
 \]
 La funzione è ben definita, perchè se gli ordinali minori di $\alpha$ hanno per immagine una catena, allora posso trovarne un maggiorante stretto.
	
 Sia ora per assurdo $m$ il minimo ordinale tale che $g(m)=\trash$. Se $m=\beta+1$ è successore, vuol dire che $\{g(\gamma):\gamma\le\beta\}$ non formano una catena, ma $\{g(\gamma):\gamma<\beta\}$ sì (perchè $g(\beta)\ne \trash$), e inoltre $g(\beta)$ è confrontabile con tutti i precedenti per definizione, assurdo. Se $m=\lambda$ è limite, allora esistono $\delta_1<\delta_2<\lambda$ con $g(\delta_1),g(\delta_2)$ non confrontabili, ma allora $g(\delta_2+1)=\trash$, contraddicendo la minimalità di $\lambda$, ancora assurdo.

 Quindi $f(\alpha)\ne\trash$ per ogni $\alpha$, e per come è definita è crescente (quindi iniettiva): posso quindi invertirla rispetto all'immagine, e ottengo una funzione suriettiva da un insieme alla classe degli ordinali, che contraddice assioma di rimpiazzamento.

 \item Zorn $\Rightarrow$ Zermelo. Sia $S$ un insieme, e sia $(B,\preceq)$ un insieme parzialmente ordinato dove $B$ è l'inseme delle coppie $(X,\le)$ dove $X$ è bene ordinato rispetto a $\le$, e dove $(X, \le_X) \preceq (Y,\le_Y)$ se $X$ è segmento iniziale di $Y$ e $\le_X, \le_Y$ coincidono su $X$, cioè $le_X \subseteq le_Y$
 
 Sia ora $C=\{(X_i,\le_i):i\in I\}$ una catena. Allora si può verificare che $(\bigcup_{i\in I}X_i,\bigcup_{i\in I}\le_i)$ appartiene a $B$ ed è un maggiorante.
 
 Quindi per il lemma di Zorn esiste un elemento $(M,\le_M)$ massimale. Allora $M=S$, altrimenti posso scegliere $n\in S\setminus M$ e estendere l'ordine imponendo $n>m$ per ogni $m\in M$.
 
 \item Zermelo $\Rightarrow$ Assioma di scelta. Voglio costruire una funzione di scelta $f: I  \rightarrow \bigcup_{i\in I}A_i$, in modo che $f(i)\in A_i$ per ogni $i\in I$. Consideriamo un buon ordine su $\bigcup_{i\in I}A_i$, e possiamo definire la funzione di scelta come $f(i)=\min(A_i)$.

\end{itemize}
\end{proof}

Sono equivalenti ad assioma di scelta (ma non lo dimostreremo) anche le seguenti proprietà:

\begin{myprop}[Confrontabilità delle cardinalità]
 Dati $A$ e $B$ insiemi, esiste una funzione iniettiva da $A$ in $B$ o una da $B$ in $A$.
\end{myprop}


\begin{myprop}
 Per ogni $A$ e $B$ insiemi, se esiste una funzione suriettiva da $B$ in $A$ allora ne esiste una iniettiva da $A$ in $B$.
\end{myprop}

\section{Cardinalità e cardinali}

\begin{myprop}
 Esiste una funzione iniettiva da $A$ in $B$ se e solo se ne esiste una suriettiva da $B$ in $A$.
\end{myprop}
La $(\Rightarrow)$ si dimostra facilmente senza usare assioma di scelta; $(\Leftarrow)$ invece è equivalente ad assioma di scelta.

\begin{mydef}[Cardinalità]
 Due insiemi $X$ e $Y$ si dicono equipotenti se esiste $f$ bigettiva da $X$ a $Y$, e si scrive $\abs X=\abs Y$. Inoltre si scrive $\abs X \le \abs Y$ se esiste una funzione iniettiva da $X$ a $Y$.
\end{mydef}

Si verifica facilmente (senza usare assioma di scelta) che valgono le seguenti proprietà:

\begin{itemize}
 \item $\abs X=\abs X$
 \item $\abs X=\abs Y \Leftrightarrow \abs Y=\abs X$
 \item $(\abs X=\abs Y \land \abs Y=\abs Z) \Rightarrow \abs X=\abs Z$
 \item $(\abs X \le \abs Y \land \abs Y \le \abs Z)\Rightarrow \abs X \le \abs Z$
\end{itemize}

Inoltre usando il lemma di Zorn si dimostra che

\begin{myprop}
 Dati $A$ e $B$ insiemi, vale $\abs A\le\abs B \vee \abs B\le\abs A$
\end{myprop}

Infine senza usare assioma di scelta si dimostra il seguente

\begin{mytheorem}[Cantor-Bernstein]
 Dati $A$ e $B$ insiemi tali che $\abs A\le\abs B$ e $\abs B\le\abs A$, vale $\abs A=\abs B$.
\end{mytheorem}
\begin{proof}
 Siano $f:A\rightarrow B$ e $g:B\rightarrow A$ iniettive.
 Vogliamo costruire una funzione $h:A\rightarrow B$ invertibile.
 Preso un elemento in $A\setminus g(B)$, possiamo considerare la sua ``prole'' ottenuta applicando ad esso prima la funzione $f$, poi la funzione $g$, poi di nuovo la funzione $f$, e così via un numero finito di volte: diciamo che gli elementi così ottenuti, che stanno un po' in $A$ ed un po' in $B$, hanno come \myname{primo progenitore} un elemento di $A$. Analogamente definiamo gli elementi che hanno come \myname{primo progenitore} un elemento di $B$. Notare che non è possibile che un elemento abbia primo progenitore sia in $A$ che in $B$; è invece possibile che un elemento non abbia primo progenitore.
 Dividiamo dunque $A$ e $B$ ciascuno in tre parti: $A_A$ gli elementi di $A$ che hanno primo progenitore in $A$, $A_B$ quelli che lo hanno in $B$, $A_{\infty}$ quelli che non hanno primo progenitore (ed analoga suddivisione per $B$). 
 Si ha che $f$ è invertibile da $A_A$ in $B_A$, $g$ è invertibile da $B_B$ in $A_B$, entrambe $f$ e $g$ sono invertibili tra $A_{\infty}$ e $B_{\infty}$. Quindi definiamo la funzione $h:A\rightarrow B$ come: $h=f$ sugli elementi di $A_A$ e di $A_{\infty}$, $h=g^{-1}$ sugli elementi di $A_B$.
\end{proof}



\subsection{Cardinali}

\begin{mydef}
 Si dice \myname{cardinale} un insieme $\kappa$ tale che $\kappa$ è un ordinale e $\abs\alpha < \abs\kappa$ per ogni $\alpha\in\kappa$.
\end{mydef}



\begin{mydef}
 Dato un insieme $A$, si definisce la funzione di Hartogs \[H(A)=\{\alpha \in ON:\abs\alpha\le\abs A\}\]
\end{mydef}

Si ha che, per ogni insieme $A$,
\begin{myprop}\
\begin{itemize}
 \item $H(A)$ è un insieme;
 \item $H(A)$ è un cardinale;
 \item $\abs{H(A)}>\abs A$ e $H(A)$ è il più piccolo ordinale con cardinalità strettamente maggiore di quella di $A$.
\end{itemize}
\end{myprop}
Ad esempio $\omega_1=H(\omega)$ è il più piccolo ordinale non numerabile. 
\begin{proof}
 Considero l'insieme $B$ di tutte le coppie $(X,\prec)$ dove $X$ è un sottoinsieme di $A$ ben ordinato da $\prec$ (perchè $B$ è un insieme?). Per \ref{prop:isomord} ogni coppia $(X,\prec)$ in $B$ è isomorfa ad esattamente un ordinale, quindi possiamo considerare la funzione $f$ che associa ad ogni $(X,\prec)$ l'ordinale corrispondente. Per l'assioma di rimpiazzamento, l'immagine $f(B)$ di tale funzione è un insieme. Si verifica (vedi esercizio \ref{ex:Hart}) che tale insieme è proprio $H(A)$. Le altre proprietà asserite sono di facile verifica.
\end{proof}

Da ora in poi identifico $\abs A$ con l'unico cardinale equipotente ad $A$. L'unicità è ovvia, l'esistenza è equivalente al teorema di Zermelo.

\begin{mydef}[Somma e prodotto tra cardinali]
 Dati $\kappa$ e $\mu$ cardinali, si definisce
 \begin{itemize}
  \item $\kappa + \mu = \abs{\kappa \sqcup \mu};$
  \item $\kappa \cdot \mu = \abs{\kappa \times \mu};$
  \item $\kappa^\mu = \abs {\{f: \mu \rightarrow \kappa\}}$;
  \item $\kappa^+=H(\kappa)$.
 \end{itemize}

\end{mydef}

\begin{myprop}\label{prop:cardsum}
 Per ogni insieme infinito $A$ vale $\abs A + \abs A = \abs A$.
\end{myprop}

\begin{proof}
 Si applica Zorn sulle coppie $(X,f)$ dove $X\subseteq A$ e $f:X \sqcup X\rightarrow X$ è bigettiva, parzialmente ordinate dall'inclusione. 
\end{proof}



\begin{myprop}\label{prop:cardprod}
 Per ogni insieme infinito $A$ vale $\abs A \cdot \abs A = \abs A$.
\end{myprop}

\begin{proof}
 Dimostro usando Zorn che $A\times A$ è equipotente ad $A$. Prendo l'insieme $\mathcal A$ delle coppie $(X,f)$ dove $X\subseteq A$ e $f:X\times X\rightarrow X$ è bigettiva, parzialmente ordinato dall'inclusione tra funzioni. Questo insieme è chiaramente induttivo (posso maggiorare le catene con l'unione). Abbiamo quindi un massimale $(M,f_M)$, quindi $\abs M \cdot \abs M=\abs M$.
 
 Supponiamo ora esista $M'\subseteq A\setminus M$ con $\abs M=\abs{M'}$. Allora \[\abs{(M\cup M')\times(M\cup M')\setminus M \times M} = \abs {M'}\cdot \abs{M'} + 2 \cdot \abs{M}\cdot \abs{M'} = \abs {M'}\] per il Lemma \ref{prop:cardsum}. Quindi posso costruire una funzione bigettiva da $(M\cup M')\times(M\cup M')\setminus M \times M$ a $M'$, che unita con $f_M$ da $M\times M$ a $M$ dà origine a una funzione bigettiva da $(M\cup M')\times(M\cup M')$ a $M \cup M'$, contraddicendo la massimalità. Quindi $\abs {A\setminus M}< \abs M$, ovvero $\abs M= \abs A$. Ma allora $\abs A = \abs M = \abs{M\times M} = \abs {A \times A}$.
\end{proof}

\begin{mycor}
 Dalle proposizioni \ref{prop:cardsum} e \ref{prop:cardprod} otteniamo facilmente che $\kappa + \mu = \kappa \cdot \mu = \max\{\kappa, \mu\}$ per ogni $\kappa,\mu$ cardinali infiniti.
\end{mycor}

%Definizione di $\kappa^+$.
%Disuguaglianze faciline e leggi sui cardinali infiniti.

\begin{myprop}
 Per ogni $\alpha,\beta$ ordinali infiniti, vale 
 \[ \left|\alpha+\beta\right|=\left|\alpha\cdot\beta\right|=\left|\alpha^\beta\right|=\max\{|\alpha|,|\beta|\}\]
 dove le operazioni sono intese tra ordinali e non tra cardinali.
\end{myprop}

\begin{myprop}
 Per ogni $\kappa, \mu$ cardinali con $\kappa \le 2^\mu$, vale \[\kappa^\mu=2^\mu\] 
\end{myprop}
\begin{proof} 
  Basta maggiorare $\kappa$ con $2^\mu$.
\end{proof}



\begin{mydef}
 Si definisce la sommatoria di cardinali come segue:
 \[\sum_{i\in I}\kappa_i = \left|\bigsqcup_{i\in I}\kappa_i\right|\]
\end{mydef}

\begin{mydef}
 Allo stesso modo si definisce la produttoria di cardinali:
 \[\prod_{i\in I}\kappa_i = \left|\prod_{i\in I}\kappa_i\right|\]
 dove la seconda è intesa come prodotto cartesiano tra insiemi.
\end{mydef}

\`E facile dimostrare i seguenti fatti.
\begin{myprop}\
\begin{itemize}
 \item $\sum _{i<\nu} \kappa =\kappa\cdot \nu$
 \item $\prod_{i<\nu} \kappa =\kappa^\nu$
 \item $\prod_{i<\nu}\alpha^{\beta_i}=\alpha^{\sum_{i<\nu}\beta_i}$
 \item $\sum_{i<\nu}\sum_{j<\mu} \alpha_{ij}=\sum_{j<\mu}\sum_{i<\nu} \alpha_{ij} $
 \item $\prod_{i<\nu}\prod_{j<\mu} \alpha_{ij}=\prod_{j<\mu}\prod_{i<\nu} \alpha_{ij}$
\end{itemize}
\end{myprop}

Inoltre, quando serve calcolare somme e prodotti di cardinali non costanti, si possono utilizzare le seguenti formule.

\begin{myprop}[Formula per la somma di cardinali]
 Sia $\mu$ un cardinale, e siano $\{\kappa_i:i<\mu\}$ cardinali non nulli. Allora \[\sum_{i< \mu} \kappa_i = \mu\cdot \sup \kappa_i\]
\end{myprop}
\begin{myprop}[Formula per il prodotto di cardinali]
 Sia $\mu$ un cardinale, e sia $\{\kappa_i:i<\mu\}$ una $\mu$-successione crescente. Allora \[\prod_{i<\mu}\kappa_i=(\sup \kappa_i)^{\mu}\]
\end{myprop}

\begin{proof}
 Innanzitutto è possibile suddividere i $\mu$ cardinali di cui sto facendo il prodotto in $\mu$ famiglie di $\mu$ elementi, dato che $\mu= \mu\cdot \mu$. Inoltre l'insieme degli indici di ciascuna famiglia è illimitato in $\mu$, perchè se fosse limitato allora apparterrebbe a un ordinale minore di $\mu$ di cardinalità almeno $\mu$, in contraddizione con la definizione di cardinale; quindi per monotonia anche i cardinali di ogni famiglia sono illimitati in $\sup \kappa_i$.
 
 Siano quindi $F_\alpha$ queste famiglie, con $\alpha<\mu$.
 Allora abbiamo
 \[\prod_{i<\mu}\kappa_i = \prod_{\alpha<\mu}\left(\prod_{\gamma\in F_\alpha}\gamma\right)\ge \prod_{\alpha<\mu}\left(\sup F_\alpha \right)=\left(\sup\kappa_i\right)^\mu \ge \prod_{i<\mu}\kappa_i \]
 da cui la tesi.
\end{proof}

\begin{mytheorem}[Teorema di K\"onig]
  Siano $\alpha_i<\beta_i$ cardinali, con $i\in I$. Allora \[\sum_{i\in I} \alpha_i < \prod_{i\in I} \beta_i\]
\end{mytheorem}

\begin{proof} Supponiamo che esista una funzione $f$ suriettiva da $\sum_{i\in I} \alpha_i$ a $\prod_{i\in I} \beta_i$. Supponiamo per semplicità gli $\alpha_i$ disgiunti. Sia $\sigma_i: \alpha_i\rightarrow \beta_i$ una funzione che associa a $\gamma \in \alpha_i$ la componente lungo $\beta_i$ di $f (\gamma)$. Ora chiaramente ciascuna $\sigma_i$ non è suriettiva per motivi di cardinalità, quindi posso prendere $x_i \in \beta_i$ con $x_i \nin \sigma_i(\alpha_i)$. Ma allora l'elemento del prodotto cartesiano la cui componente $i$-esima è $x_i$ non è nell'immagine di $f$, assurdo.
\end{proof}
%Cantor 

\subsection{Cofinalità di un ordinale}

Dato un insieme totalmente ordinato $A$, si definisce la \myname{cofinalità} di $A$
\[ \cof(A) := \min\{|X|:X\subseteq A \textrm{ è illimitato in }A\}\]  
Chiaramente si ha $\cof A \le |A|$, e A si dice
\begin{itemize}
 \item \myname{regolare} se $\cof A=|A|$;
 \item \myname{singolare} se $\cof A<|A|$.
\end{itemize}

Possiamo caratterizzare la cofinalità anche in quest'altro modo:
\begin{myprop}

 La cofinalità di $\kappa$ è il minimo cardinale $\mu$ tale che esiste una $\mu$-successione $\kappa_i<\kappa$ tale che $\kappa =\sum_{i\in \mu} \kappa_i$.
\end{myprop}

Possiamo inoltre calcolare la cofinalità di un cardinale distinguendo il caso successore e limite.
\begin{myprop}
 Per ogni $\kappa$ cardinale infinito, vale $\cof\kappa^+ = \kappa^+$. Equivalentemente \[\cof\aleph_{\alpha+1}=\aleph_{\alpha+1}\]
\end{myprop}
\begin{proof}
 Se così non fosse, avrei un sottoinsieme di cardinalità $\kappa$ illimitato in $\kappa^+$. Allora posso scrivere $\kappa^+$ come unione di $\kappa$ insiemi di cardinalità $\kappa$, assurdo perchè $\kappa\cdot\kappa=\kappa$.
\end{proof}
\begin{myprop}
 Per ogni $\lambda$ ordinale limite, vale \[\cof \aleph_\lambda =\cof \lambda\]
\end{myprop}
\begin{proof}
 L'idea è di mettere in corrispondenza i $\gamma < \lambda$ con gli $\aleph_\gamma$.
\end{proof}

\begin{myprop}
 Per ogni $\alpha, \beta$ ordinali, vale $\cof(\alpha+\beta) = \cof \beta$.
\end{myprop}

\begin{myprop}[Teorema di Hausdorff]
 Per ogni $\kappa,\nu$ cardinali con $\kappa$ infinito vale ${\left(\kappa^+\right)}^\nu=\kappa^+\cdot \kappa^\nu$. Quindi
 \[\aleph_{\alpha+1}^\nu=\aleph_{\alpha}^\nu\cdot \aleph_{\alpha+1}\]
\end{myprop}
\begin{proof}
 Basta dimostrare ${\left(\kappa^+\right)}^\nu \le\kappa^+\cdot \kappa^\nu$, in quanto l'altro verso è ovvio. 
 
 Se $2^\nu \ge \kappa^+$ allora basta maggiorare $\kappa^+$ con $2^\nu$ per ottenere la tesi.
 
 Altrimenti se $\kappa^+>2^\nu$ si ha $ \cof\kappa^+ =\kappa^+>2^\nu>\nu $, quindi ogni funzione da $\nu$ in $\kappa^+$ è limitata. Quindi otteniamo
 
 \[{\left(\kappa^+\right)}^\nu = \abs{\Fun(\nu,\kappa^+)} = \bigcup_{\alpha\in \kappa^+} \abs{\Fun(\nu,\alpha)} \le \sum_{\alpha<\kappa^+} \alpha^\nu \le \kappa^+\cdot \kappa^\nu\]
 
 da cui la tesi.
\end{proof}

\begin{myprop}
 Per ogni $\kappa,\nu$ cardinali, con $\kappa$ cardinale limite e $\nu \ge \cof \kappa$, vale \[\kappa^\nu={\left(\sup_{\gamma<\kappa}\gamma^\nu\right)}^{\cof\kappa}\]
\end{myprop}

\begin{proof}
 Sia $\{\alpha_i\}_{i<\cof\kappa}$ una successione crescente illimitata in $\kappa$. Allora, per la formula del prodotto vale
 
 \[\prod_{i<\cof\kappa} \alpha_i^\nu = \left(\sup_{i<\cof\kappa}\alpha_i^\nu\right)^{\cof\kappa}={\left(\sup_{\gamma<\kappa}\gamma^\nu\right)}^{\cof\kappa}\]
 
 Ma vale anche 
 \[\prod_{i<\cof\kappa} \alpha_i^\nu=\left(\prod_{i<\cof\kappa} \alpha_i\right)^\nu=\left(\left(\sup_{i<\cof\kappa}\alpha_i\right)^{\cof\kappa}\right)^\nu=\kappa^\nu\]
 
 Uguagliando le due espressioni ottengo la tesi.
\end{proof}

\begin{myprop}
 Per ogni $\kappa,\nu$ cardinali con $\nu < \cof \kappa$ vale \[\kappa^\nu=\kappa \cdot \sup_{\gamma<\kappa}\gamma^\nu\]
\end{myprop}
\begin{proof} La dimostrazione si fa stimando con le funzioni limitate come in Hausdorff.
 
\end{proof}


Quindi se devo calcolare $\kappa^\nu$ distinguo i casi a seconda che $\kappa$ sia successore, nel qual caso uso il teorema di Hausdorff (eventualmente ripetutamente, finchè non mi ritrovo con un cardinale limite), oppure limite, nel qual caso distinguo ulteriormente due casi a seconda che l'esponente $\nu$ sia $<\cof \kappa$ o $\ge \cof \kappa$.

Inoltre valgono i seguenti fatti: 

\begin{itemize}
\item Per ogni $\kappa$ cardinale infinto vale $\kappa^{\cof\kappa}>\kappa$;
\item Sempre per $\kappa$ infinito $\cof 2^\kappa>\kappa$;

\item $|\{X\subseteq \kappa: |X|=\nu\}|=\kappa^\nu$
\end{itemize}

\section{Bonus}
\subsection{Gerarchia di von Neumann}
Si definiscono ricorsivamente
\begin{itemize}
   \item $V_0 = \emptyset$
   \item $V_{\alpha +1}=\mathcal{P}(V_\alpha)$
   \item $V_{\lambda}=\bigcup_{\gamma<\lambda}V_\gamma$ se $\lambda$ è un ordinale limite
\end{itemize}
Valgono i seguenti fatti:
\begin{itemize}
  \item $x\in y\in V_\alpha \Rightarrow \exists \beta <\alpha \quad x \in V_\beta$
  \item $\alpha\le\beta \Rightarrow V_\alpha\subseteq V_\beta$
  \item $x\in V_\alpha \Rightarrow x \subset V_\alpha$ (quindi $V_\alpha \notin V_\alpha$)
  \item $V_\alpha$ è transitivo
  \item $\alpha \subseteq V_\alpha$
  \item $\alpha \notin V_\alpha$
\end{itemize}

\subsection{Beth}
Si definiscono per ricorsione transfinita i cardinali
\begin{itemize}
 \item $\beth_0:=\aleph_0$
 \item $\beth_{\alpha+1}:=2^{\beth_\alpha}$
 \item $\beth_\lambda:=\bigcup_{\gamma\in\lambda}\beth_\gamma$ se $\lambda$ è un ordinale limite.
\end{itemize}

\subsection{Chiusura transitiva}
Dato un insieme $A$, ci chiediamo quale sia il più ``piccolo'' insieme transitivo che lo contiene; definiamo
\begin{itemize}
 \item $A_0:=A$
 \item $A_{n+1}:=\bigcup_{B\in A_n}B$
 \item $A_\omega:=\bigcup_{i\in\omega}A_i$
\end{itemize}
Si ha che $A_\omega$ è il più ``piccolo'' insieme transitivo che contiene $A$ nel senso che
\begin{myprop}
 $A_\omega$ è transitivo e contiene $A$. Inoltre, se $X$ è transitivo e contiene $A$, allora $X$ contiene $A_\omega$.
\end{myprop}

\subsection{$\sigma$-algebra dei boreliani}

\begin{mydef}
Una $\sigma$-algebra su un insieme $X$ è una famiglia $\mathcal F$ di sottoinsiemi di $X$ tale che:
\begin{itemize}
 \item $\emptyset\in\mathcal F$;
 \item $\mathcal F$ è chiusa per passaggio al complementare;
 \item $\mathcal F$ è chiusa per unione numerabile.
\end{itemize}
\end{mydef}

\begin{mydef}
La $\sigma$-algebra dei boreliani di $\mathbb R$ è la più piccola tra (nel senso che è l'intersezione di) tutte le $\sigma$-algebre su $\mathbb R$ che contengono tutti gli aperti.
\end{mydef}

Ci chiediamo quale sia la cardinalità della $\sigma$-algebra dei boreliani.

\begin{mydef}
Definiamo per ricorsione transfinita:
\begin{itemize}
 \item $B_0:=\{A : A$ è un aperto di $\mathbb R\}$;
 \item $B_{\alpha+1}:=B_\alpha\cap\{A : A^c\in B_\alpha\}\cup\{A : A$ è un'unione numerabile di elementi di $B_\alpha\}\cup\{A : A$ è un'intersezione numerabile di elementi di $B_\alpha\}$;
 \item $B_\lambda:=\bigcup\limits_{\gamma<\lambda}B_\gamma$.
\end{itemize}
\end{mydef}
Allora la $\sigma$-algebra dei boreliani è $B_{\omega_1}$: infatti è facile dimostrare per induzione transfinita che se $\alpha<\omega_1$ allora $B_\alpha$ è sottoinsieme dei boreliani, da cui si ha che $B_{\omega_1}=\bigcup\limits_{\gamma<\omega_1}B_\gamma$ è sottoinsieme dei boreliani. Inoltre è altrettanto facile verificare che $B_{\omega_1}$ è una sigma algebra che contiene tutti gli aperti di $\mathbb R$. Quindi $B_{\omega_1}$ contiene i boreliani. Quindi i boreliani sono $B_{\omega_1}$.
La cardinalità di $B_0$ è $\mathfrak c$; si dimostra facilmente per induzione transfinita che se $\alpha<\omega_1$ allora la cardinalità  di $B_\alpha$ è sempre $\mathfrak c$. Infine, la cardinalità di $B_{\omega_1}$ è $c$ perchè $B_{\omega_1}$ è un'unione di $\omega_1$ insiemi con cardinalità del continuo, e $\omega_1\le \mathfrak c$.
Quindi i boreliani di $\mathbb R$ hanno cardinalità $\mathfrak c$.

\subsection{Lebesgue-misurabili di $\mathbb R$}

La famiglia dei sottoinsiemi misurabili secondo Lebesgue di $\mathbb R$ ha cardinalità $2^c$: una disuguaglianza è ovvia; l'altra segue dalle seguenti due osservazioni:
\begin{itemize}
 \item un sottoinsieme di un insieme con misura secondo Lebesgue nulla è ancora Lebesgue-misurabile, ed ha misura nulla;
 \item l'insieme di Cantor ha cardinalità del continuo e misura di Lebesgue nulla.
\end{itemize}
Quindi ognuno dei $2^c$ sottoinsiemi dell'insieme di Cantor è Lebesgue-misurabile.

\section{Esercizi}
\subsection{Classi e insiemi}
\begin{myex}
 Dire che cos'è $\bigcup_{X\in \emptyset} X$ e $\bigcap_{X\in \emptyset} X$, e dire se sono classi proprie o insiemi.
\end{myex}
\begin{myex}
 La classe di tutti gli insiemi non è un insieme.
\end{myex}
\begin{myex}
 La classe di tutti i campi non è un insieme.
\end{myex}

\subsection{Modello in cui valgano gli assiomi}

Ci chiediamo se esista dentro ZF un insieme che possa essere preso come classe di tutti gli insiemi, e dentro il quale valgano gli assiomi. Questo ci porta alla domanda: che caratteristiche deve avere un insieme $X$ per soddisfare gli assiomi? (Nota: imporremo delle condizioni in generale più forti dello stretto necessario)

\begin{itemize}
 \item (Assioma di estensionalità) Vogliamo che ogni oggetto che stia in un insieme stia anche nella classe di tutti gli oggetti, cioè che l'insieme $X$ sia transitivo. In questo modo due insiemi sono uguali se e solo se hanno gli stessi elementi (dove elementi va inteso come oggetti nella classe $X$, cioè posso parlare solo di oggetti che appartengono ad $X$).
\end{itemize}
Ora poniamo $X=V_\alpha$ (che è transitivo) e ci chiediamo che condizioni servano su $\alpha$.
\begin{itemize}
 \item (Assioma di comprensione) Gratis dalle proprietà della gerarchia di von Neumann.
 \item (Assioma della coppia) \'E sufficiente imporre che $\alpha$ sia un ordinale limite.
 \item (Assioma dell'unione) Ogni $\alpha$ va bene, la verifica è lasciata per esercizio al lettore.
 \item (Assioma di potenza) \'E sufficiente imporre $\alpha$ limite.
 \item (Assioma dell'infinito) \'E sufficiente imporre $\alpha>\omega$.
 \item (Assioma di rimpiazzamento) Posto $\alpha$ limite, serve che $\forall Y\in V_\alpha$ e per ogni $\langle x_i | i\in Y\rangle$ con $x_i \in V_\alpha$ valga $\bigcup x_i\in V_\alpha$. Essendo $\alpha$ limite, per ogni $i$ esiste $\gamma_i$ tale che $x_i\in V_{\gamma_i}$: affinchè valga ciò che vogliamo è sufficiente che i $\gamma_i$ siano limitati in $\alpha$, cioè che $\forall Y\in V_\alpha \quad \cof\alpha>\abs{Y}$. Inoltre se $Y\in V_\alpha$ esiste $\beta<\alpha$ tale che $Y\subseteq V_\beta$ da cui $\abs{Y}\le\abs{V_\beta}=\beth_{\beta-\omega}$. Se $\beta$ è abbastanza grosso (tipo almeno $\omega^2$) $\beta-\omega=\beta$. Infine se fosse $\cof\alpha<\alpha$ ci sarebbe una successione di $\gamma_i$ indicizzata con $V_{\cof\alpha}$ illimitata in $\alpha$, e l'unione dei $V_{\gamma_i}$ non apparterrebbe a $V_\alpha$.

Quindi imponiamo $\alpha=\cof\alpha$ e $\forall\beta<\alpha\quad\beth_\beta<\alpha$.
\end{itemize}

\begin{myex}
 Dati $A$ e $B$ insiemi, la classe delle funzioni $f: A\rightarrow B$ è un insieme.
\end{myex}

\begin{myex}\label{ex:classisature}
 Ricordando che $\mathbb N$ è l'intersezione di tutti gli \emph{insiemi} $S$-saturi, dimostrare che una classe $S$-satura contiene $\mathbb N$.
\end{myex}



%Fatti facili che vengono anche smanettando con i reali
\subsection{Ordinali}

\begin{myex} Dimostrare che i seguenti fatti sono equivalenti:
\begin{itemize}
 \item $\forall \beta < \alpha \quad \beta + \alpha=\alpha$;
 \item $\forall \beta,\gamma <\alpha \quad \beta +\gamma <\alpha$;
 \item $\exists \delta: \; \alpha=\omega^\delta$.
\end{itemize}
\end{myex}

\begin{myex}\label{ex:puntifissi}
 Dimostrare che, se $\gamma$ è un ordinale successore, ogni funzione crescente e continua $f:\omega_1\gamma\rightarrow\omega_1\gamma$ ammette punti fissi.
\end{myex}


\subsection{Cardinalità e cardinali}
\begin{myex}
 Dimostrare, senza usare assioma di scelta, che per ogni insieme $A$ esiste un buon ordine che lo renda isomorfo a ogni ordinale $\alpha$ con $\abs\alpha=\abs A$.
\end{myex}
\begin{myex}\label{ex:Hart}
 Dimostrare che $H(A)$ è un insieme, dove $H$ è la funzione di Hartogs.
\end{myex}
\begin{myex}
 Dimostrare le seguenti identità tra cardinali:
 \begin{itemize}
  \item $\prod_{n<\omega}\aleph_n=\aleph_\omega^{\aleph_0}$
  \item $\aleph_\omega^{\aleph_1}=2^{\aleph_1}\cdot \aleph_\omega^{\aleph_0}$
 \end{itemize}
\end{myex}

\subsection{Bonus}

\begin{myex}
 Dimostrare che $\abs{V_{\omega+\alpha}}=\beth_\alpha$.
\end{myex}

\begin{myex}
 Dimostrare che esistono infiniti cardinali $\kappa$ tali che $\beth_\kappa=\kappa$.
\end{myex}

\end{document}
