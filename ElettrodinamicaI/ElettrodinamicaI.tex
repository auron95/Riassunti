\documentclass[a4paper,10pt,oneside]{math_article}
\usepackage{bm}
\newcommand{\rot}{\vec{\nabla} \times}
\renewcommand{\div}{\vec{\nabla} \cdot}
\renewcommand{\phi}{\varphi }
\newcommand{\grad}{\vec{\nabla}}
\newcommand{\lapl}{{\nabla}^2}

\begin{document}
	\section{Teoria}
	\subsection{Densità e correnti}
	La densità di carica $\rho(t,\vec r)$, è {\it l'unica} funzione scalare tale che, per ogni scelta di un volume chiuso $V$ si abbia:
	\[
		\int_V  \rho(t,\vec{r}\ ') \de^3r'=\mbox{carica netta contenuta dentro $V$ al tempo $t$}
	\]
	La definzione di $\vec J(t,\vec r)$ è simile: è {\it l'unico} campo vettoriale tale che, per ogni scelta di superficie $S$, orientata da un versore normale $\hat \nu(\vec r)$, con bordo $\partial S$, si abbia:
	\[
		\int_S \vec J(t,\vec{r}\ ')\cdot \hat\nu(\vec{r}\ ')=\mbox{corrente concatenata a $\partial S$ al tempo $t$}
	\]
	naturalmente tutte le quantità nell'ultima formula devono essere prese con orientazione destrorsa coerente.\\
	Dalle definzioni ricaviamo le unità di misura di $\rho$ e $\vec J$:
	\[
		[\rho]=[Q][L^{-3}]\qquad \qquad [J]=[Q][L^{-2}][T^{-1}]
	\]
	Naturalmente possiamo ``concentrare'' le nostre distribuzioni di sorgenti su insiemi bidimensionali o monodimensionali, aiutandoci con delle Delte. Ad esempio una densità lineare lungo $\hat z$ si può scrivere formalmente come (per quanto sia tipicamente inutile):
	\[
		\rho(x,y,z)=\delta(x)\delta(y)\lambda
	\]
	con $\lambda$ che ha dimensioni $[Q][L^{-1}]$ e le delta hanno dimensioni $[L^{-1}]$.\\
	Oppure una distribuzione di corrente uniforme sul piano $x=0$ si scrive come:
	\[
		\vec J(x,y,z)=\delta(x)\vec\chi
	\]
	con ovvie dimensioni.\\
	Tra $\rho$ e $J$ intercorre una fondamentale relazione di continuità che si esprime nella formula:
	\begin{equation}
		\diffp \rho t +\div \vec J=0\label{eq:continuita}
	\end{equation}
	
	Nel formalismo relativistico si ha che la seguente quantità è un quadrivettore:
	\[
		j^\mu=\left(c\rho,\vec J\right)
	\]
	e l'equazione di continuità si scrive compattamente come:
	\[
		\partial_\mu j^\mu=0
	\]
	Supponiamo ora $S$ superficie orientata da $\hat\nu$ e supponiamo di avere dei campi interni ed esterni. Fissato un punto $p$ della superficie operiamo una misura dei campi in quel punto prima avvicinandoci dall' interno e poi dall esterno, misureremo quattro vettori:
	\[
		E^{in}\qquad B^{in}\qquad E^{out}\qquad B^{out}
	\]
	tra di essi intercorrono delle relazioni ben precise che ci dicono, fra l' altro, se ci sono e quanto valgono eventuali correnti, cariche superficiali nel punto $p$:
	\[
		E^{in}_\parallel=E^{out}_\parallel\qquad E^{out}_\nu-E^{in}_\nu=4\pi \sigma
	\]
	\[
		B^{out}_\parallel-B^{in}_\parallel=\frac{4\pi}{c}\vec\chi\qquad B^{out}_\nu=B^{in}_\nu
	\]
	dove le componenti parallele sono parallele al tangente ad S in $p$, e quelle marcate con $\nu$ sono quelle lungo la normale.
		\subsection{Equazioni di Maxwell}
			Le equazioni di Maxwell per l'elettrodinamica sono le seguenti:
			\begin{align}\displaystyle
				\div \vec E &= 4\pi \rho \label{eq:MaxGauss}\\ 
				\div \vec B &= 0 \label{eq:MaxMonopole} \\
				\rot \vec E &= -\frac1c \diffp {\vec B}t \label{eq:MaxFaraday}\\
				\rot \vec B &= \dfrac1c \diffp {\vec E}t + \dfrac{4\pi}c \vec J \label{eq:MaxAmpere}
			\end{align}
			Da osservare che gli operatori differenziali col $\vec \nabla$ agiscono solo sulle coordinate spaziali.
		\subsection{Potenziali}
			Possiamo definire un potenziale vettore $\vec A$ tale che
			\begin{equation}\label{eq:DefVecPot}
			 \rot \vec A = \vec B
			\end{equation}

			Prendo il rotore di \ref{eq:DefVecPot} e sostituisco nella \ref{eq:MaxAmpere} ottenendo
			\begin{equation}
				\rot (\rot \vec A) = \grad(\div \vec A) - \lapl \vec A = 
				%%TODO
			\end{equation}
			
		\subsection{Irraggiamento di un dipolo}
			Consideriamo un sistema di sorgenti localizzate (con scala $a$) monocromatiche, ossia la loro dipendenza dal tempo è periodica con frequenza $\omega$. Sia $\lambda=\frac{2\pi c}{\omega}$ la lunghezza d'onda.
			Supponiamo quindi che $\rho(\vec x, t) = \rho(x)e^{i\omega t}$, e sia $\vec d= \int \rho(x')\vec x' dx'^3$. Calcoliamo i campi in un punto $\vec R = R\cdot \hat n$.
			
			In \emph{zona d'onda}, ossia dove $R >> a$, $R >> \lambda$, vale la seguente formula a patto che valga anche $\lambda >> a$:
			\begin{equation}
				\begin{cases}
					\vec B = \frac{k^2}{R} e^{ikR} (\hat n \wedge \vec d)\\
					\vec E = \vec B \wedge \hat n
				\end{cases}
			\end{equation}
			
			Potrebbero esserci problemi di segno.
			%TODO Serivirà aggiungere come si ricava%

			Quindi $\vec E$ e $\vec B$ sono proporzionali a $\sin \theta$; $\vec E$ è diretto lungo $\hat \theta$, $\vec B$ è diretto lungo $\hat \phi$, quindi il vettore di Poynting è diretto lungo $\hat n$.
			
			Inoltre, se siamo interessati alla distribuzione angolare della potenza irradiata, vale
			\[
			 \de P = \frac c{8\pi} d^2k^4\sin^2\theta  \de\Omega
			\]
		\subsection{Energia elettromagnetica}
			Si può associare un'energia $u$ ai campi in modo che si conservi l'energia del sistema. Dobbiamo porre
			\begin{equation}
				u = \frac {E^2+B^2}{8\pi}
			\end{equation}
			La variazione dell'energia all'interno di un volumetto è data dal flusso del \emph{vettore di Poynting} dato da 
			\begin{equation}
				\vec S=\frac{c}{4\pi} \vec E \wedge \vec B
			\end{equation}


			
		\subsection{Quantità di moto}
			Consideriamo una superficie $S$ che racchiude un volume $V$, in cui ho cariche e correnti. I campi esercitano quindi delle forze, che possono produrre quantità di moto. Per scrivere la conservazione della quantità di moto, possiamo provare ad associare una quantità di moto ai campi.
			
			Da $\diff {\vec P}t = \vec F$  dopo un po' di conti si ottiene 
			\begin{equation}
					\diff {\vec P}t + \frac1{4\pi c}\diff{}t \int_V (\vec E \wedge \vec B) \de\vec r = \int_S T_{ij}n_j \de S
			\end{equation}
			dove $T_{ij}$ è il tensore degli sforzi di Maxwell, dato da 
			\begin{equation}
				T_{ij} = \frac1{4\pi} \left(E_iE_j+B_iB_j - \frac12 (E^2 + B^2) \delta_{ij}\right)
			\end{equation}
			
			Possiamo quindi associare una quantità di moto per unità di volume data da 
			\begin{equation}
				\vec g = \frac1{4	\pi c} \vec E \wedge \vec B.
			\end{equation}
			
			Notiamo inoltre che $\vec S = c^2 \vec g$.			

		\subsection{Circuiti}
			La caduta di potenziale associata agli elementi di un circuito è:
			\begin{itemize}
			 \item Resistenza: $V=RI$
			 \item Condensatore: $V=\dfrac QC$
			 \item Induttore: $V=L\diff It$
			\end{itemize}
			
			Dove $L$ di un solenoide è proporzionale a $\dfrac{n^2S}{l}$, con $n$ numero di spire per unità di lunghezza, l è la lunghezza del solenoide e $S$ la sua sezione. (Controllare che la formula sia vera)
			
			Inoltre l'enegia immagazzinata in un induttore vale $U_L=\dfrac12 LI^2$, mentre in un condensatore è $U_C=\dfrac12 \dfrac{Q^2}C$. La potenza dissipata dalle resistenze vale $P=RI^2$. 



\end{document}
