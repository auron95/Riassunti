\documentclass[a4paper,10pt,oneside]{math_article}

\newcommand{\rot}{\nabla \times}
\renewcommand{\div}{\nabla \cdot}
\newcommand{\grad}{\nabla}
\newcommand{\lapl}{\nabla^2}

\begin{document}
	\section{Teoria}
		\subsection{Equazioni di Maxwell}
			Le equazioni di Maxwell per l'elettrodinamica sono le seguenti:
			\begin{align}\displaystyle
				\div E &= 4\pi \rho \label{eq:MaxGauss}\\ 
				\div B &= 0 \label{eq:MaxMonopole} \\
				\rot E &= -\frac1c \diffp Bt \label{eq:MaxFaraday}\\
				\rot B &= \dfrac1c \diffp Et + \dfrac{4\pi}c \vec J \label{eq:MaxAmpere}
			\end{align}
			
		\subsection{Potenziali}
			Possiamo definire un potenziale vettore $\vec A$ tale che
			\begin{equation}\label{eq:DefVecPot}
			 \rot A = B
			\end{equation}

			Prendo il rotore di \ref{eq:DefVecPot} e sostituisco nella \ref{eq:MaxAmpere} ottenendo
			\begin{equation}
				\rot (\rot A) = \grad(\div \vec A) - \lapl \vec A = 
			\end{equation}

\end{document}
