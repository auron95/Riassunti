\documentclass[a4paper,10pt,oneside]{math_article}
\usepackage{bm}
\newcommand{\rot}{\vec{\nabla} \times}
\renewcommand{\div}{\vec{\nabla} \cdot}
\renewcommand{\phi}{\varphi }
\newcommand{\grad}{\vec{\nabla}}
\newcommand{\lapl}{{\nabla}^2}
\newcommand{\dalamb}{\square}

\begin{document}
	\section{Teoria}
	\subsection{Densità e correnti}
	La densità di carica $\rho(t,\vec r)$, è {\it l'unica} funzione scalare tale che, per ogni scelta di un volume chiuso $V$ si abbia:
	\[
		\int_V  \rho(t,\vec{r}\ ') \de^3r'=\mbox{carica netta contenuta dentro $V$ al tempo $t$}
	\]
	La definzione di $\vec J(t,\vec r)$ è simile: è {\it l'unico} campo vettoriale tale che, per ogni scelta di superficie $S$, orientata da un versore normale $\hat \nu(\vec r)$, con bordo $\partial S$, si abbia:
	\[
		\int_S \vec J(t,\vec{r}\ ')\cdot \hat\nu(\vec{r}\ ')=\mbox{corrente concatenata a $\partial S$ al tempo $t$}
	\]
	naturalmente tutte le quantità nell'ultima formula devono essere prese con orientazione destrorsa coerente.\\
	Dalle definzioni ricaviamo le unità di misura di $\rho$ e $\vec J$:
	\[
		[\rho]=[Q][L^{-3}]\qquad \qquad [J]=[Q][L^{-2}][T^{-1}]
	\]
	Naturalmente possiamo ``concentrare'' le nostre distribuzioni di sorgenti su insiemi bidimensionali o monodimensionali, aiutandoci con delle Delte. Ad esempio una densità lineare lungo $\hat z$ si può scrivere formalmente come (per quanto sia tipicamente inutile):
	\[
		\rho(x,y,z)=\delta(x)\delta(y)\lambda
	\]
	con $\lambda$ che ha dimensioni $[Q][L^{-1}]$ e le delta hanno dimensioni $[L^{-1}]$.\\
	Oppure una distribuzione di corrente uniforme sul piano $x=0$ si scrive come:
	\[
		\vec J(x,y,z)=\delta(x)\vec\chi
	\]
	con ovvie dimensioni.\\
	Tra $\rho$ e $J$ intercorre una fondamentale relazione di continuità che si esprime nella formula:
	\begin{equation}
		\diffp \rho t +\div \vec J=0\label{eq:continuita}
	\end{equation}
	
	Nel formalismo relativistico si ha che la seguente quantità è un quadrivettore:
	\[
		j^\mu=\left(c\rho,\vec J\right)
	\]
	e l'equazione di continuità si scrive compattamente come:
	\[
		\partial_\mu j^\mu=0
	\]
	Supponiamo ora $S$ superficie orientata da $\hat\nu$ e supponiamo di avere dei campi interni ed esterni. Fissato un punto $p$ della superficie operiamo una misura dei campi in quel punto prima avvicinandoci dall' interno e poi dall esterno, misureremo quattro vettori:
	\[
		E^{in}\qquad B^{in}\qquad E^{out}\qquad B^{out}
	\]
	tra di essi intercorrono delle relazioni ben precise che ci dicono, fra l' altro, se ci sono e quanto valgono eventuali correnti, cariche superficiali nel punto $p$:
	\[
		E^{in}_\parallel=E^{out}_\parallel\qquad E^{out}_\nu-E^{in}_\nu=4\pi \sigma
	\]
	\[
		B^{out}_\parallel-B^{in}_\parallel=\frac{4\pi}{c}\vec\chi\qquad B^{out}_\nu=B^{in}_\nu
	\]
	dove le componenti parallele sono parallele al tangente ad S in $p$, e quelle marcate con $\nu$ sono quelle lungo la normale.
		\subsection{Equazioni di Maxwell}
			Le equazioni di Maxwell per l'elettrodinamica sono le seguenti:
			\begin{align}\displaystyle
				\div \vec E &= 4\pi \rho \label{eq:MaxGauss}\\ 
				\div \vec B &= 0 \label{eq:MaxMonopole} \\
				\rot \vec E &= -\frac1c \diffp {\vec B}t \label{eq:MaxFaraday}\\
				\rot \vec B &= \dfrac1c \diffp {\vec E}t + \dfrac{4\pi}c \vec J \label{eq:MaxAmpere}
			\end{align}
			Da osservare che gli operatori differenziali col $\vec \nabla$ agiscono solo sulle coordinate spaziali.
		\subsection{Potenziali}
			Possiamo definire un potenziale vettore $\vec A$ tale che
			\begin{equation}\label{eq:DefVecPot}
			 \rot \vec A = \vec B
			\end{equation}

			Prendo il rotore di \ref{eq:DefVecPot} e sostituisco nella \ref{eq:MaxAmpere} ottenendo
			\begin{equation}
				\rot (\rot \vec A) = \grad(\div \vec A) - \lapl \vec A = 
				%%TODO
			\end{equation}
			
		\subsection{Irraggiamento di un dipolo}
			Consideriamo un sistema di sorgenti localizzate (con scala $a$) monocromatiche, ossia la loro dipendenza dal tempo è periodica con frequenza $\omega$. Sia $\lambda=\frac{2\pi c}{\omega}$ la lunghezza d'onda.
			Supponiamo quindi che $\rho(\vec x, t) = \rho(x)e^{i\omega t}$, e sia $\vec d= \int \rho(x')\vec x' dx'^3$. Calcoliamo i campi in un punto $\vec R = R\cdot \hat n$.
			
			In \emph{zona d'onda}, ossia dove $R >> a$, $R >> \lambda$, vale la seguente formula a patto che valga anche $\lambda >> a$:
			\begin{equation}
				\begin{cases}
					\vec B = \frac{k^2}{R} e^{ikR} (\hat n \wedge \vec d)\\
					\vec E = \vec B \wedge \hat n
				\end{cases}
			\end{equation}
			
			Potrebbero esserci problemi di segno.
			%TODO Serivirà aggiungere come si ricava%

			Quindi $\vec E$ e $\vec B$ sono proporzionali a $\sin \theta$; $\vec E$ è diretto lungo $\hat \theta$, $\vec B$ è diretto lungo $\hat \phi$, quindi il vettore di Poynting è diretto lungo $\hat n$.
			
			Inoltre, se siamo interessati alla distribuzione angolare della potenza irradiata, in media nel tempo vale:
			\[
			 \de P = \frac c{8\pi} d^2k^4\sin^2\theta  \de\Omega
			\]
		\subsection{Formula di Larmor} 
			Una carica $q$ con accelerazione $\vec a$ (e velocità non relativistica) emette nella direzione $\hat n$ una potenza per unità di angolo solido pari a:
			\[
				\de P=\frac{q^2}{4\pi c^3}a^2\sin^2 \theta \de \Omega
			\]
			dove $\theta$ è l'angolo tra $\hat n$ ed $\vec a$. Con un' integrazione si ottiene che la potenza totale emessa è:
			\[
				P=\frac{2}{3}\frac{q^2}{c^3} a^2
			\]
			
			Da notare che dalla formula di Larmor si può ricavare quella del dipolo imponendo $\vec a = - \omega^2 \vec r$ (compare un fattore $\frac12$ dato dalla media temporale del $\cos^2(\omega t)$.
		\subsection{Energia elettromagnetica}
			Si può associare un'energia $u$ ai campi in modo che si conservi l'energia del sistema. Dobbiamo porre
			\begin{equation}
				u = \frac {E^2+B^2}{8\pi}
			\end{equation}
			La variazione dell'energia all'interno di un volumetto è data dal flusso del \emph{vettore di Poynting} dato da 
			\begin{equation}
				\vec S=\frac{c}{4\pi} \vec E \wedge \vec B
			\end{equation}


			
		\subsection{Quantità di moto}
			Consideriamo una superficie $S$ che racchiude un volume $V$, in cui ho cariche e correnti. I campi esercitano quindi delle forze, che possono produrre quantità di moto. Per scrivere la conservazione della quantità di moto, possiamo provare ad associare una quantità di moto ai campi.
			
			Da $\diff {\vec P}t = \vec F$  dopo un po' di conti si ottiene 
			\begin{equation}
					\diff {\vec P}t + \frac1{4\pi c}\diff{}t \int_V (\vec E \wedge \vec B) \de\vec r = \int_S T_{ij}n_j \de S
			\end{equation}
			dove $T_{ij}$ è il tensore degli sforzi di Maxwell, dato da 
			\begin{equation}
				T_{ij} = \frac1{4\pi} \left(E_iE_j+B_iB_j - \frac12 (E^2 + B^2) \delta_{ij}\right)
			\end{equation}
			
			Possiamo quindi associare una quantità di moto per unità di volume data da 
			\begin{equation}
				\vec g = \frac1{4	\pi c} \vec E \wedge \vec B.
			\end{equation}
			
			Notiamo inoltre che $\vec S = c^2 \vec g$.			

		\subsection{Circuiti}
			La caduta di potenziale associata agli elementi di un circuito è:
			\begin{itemize}
			 \item Resistenza: $V=RI$
			 \item Condensatore: $V=\dfrac QC$
			 \item Induttore: $V=L\diff It$
			\end{itemize}
			per capire i segni (specie delle induttanze) bisogna inanzitutto fissare un verso positivo della corrente, poi si puo' utilizzare la legge di Faraday. Come check si può ricordare che l' equazione del circuito $LC$ è:
			\begin{equation}
				\ddot{Q}+\frac{1}{LC}Q=0
			\end{equation}
			che è il buon e vecchio oscillatore armonico della meccanica.\\			
			La $L$ di un solenoide è proporzionale a $\dfrac{n^2S}{l}$, con $n$ numero di spire per unità di lunghezza, l è la lunghezza del solenoide e $S$ la sua sezione. (Controllare che la formula sia vera)
			
			Inoltre l'enegia immagazzinata in un induttore vale $U_L=\dfrac12 LI^2$, mentre in un condensatore è $U_C=\dfrac12 \dfrac{Q^2}C$. La potenza dissipata dalle resistenze vale $P=RI^2$. 

		\subsection{Relatività}
					
			Le trasformazioni di Lorentz sono matrici $\Lambda_\mu^\nu$ tali che, detto $g_{\mu\nu}$ il tensore diagonale solito (spiegare meglio) vale
			\[
				\Lambda^\mu_\alpha \Lambda^\nu_\beta g_{\mu\nu} = g_{\alpha\beta}
			\]
			
			Possiamo alzare e abbassare gli indici nel seguente modo:
			\[
				X^\mu=\left(ct,\vec r\ \right)\qquad \Longleftrightarrow\qquad X_\mu=\left(ct,-\vec r\ \right)
			\]
			che equivale a dire che $X_\mu=g_{\nu\mu}X^\nu$.
						
			Dato un qualsiasi quadrivettore controvariante (indice in alto) $X^\mu$, si trasforma come $X'^\nu= \Lambda_\mu^\nu X^\mu$.
			%Se invece ha l'indice in basso? (Non mi è chiarissimo come scriverlo in forma tensoriale)
			
			
			Il quadrigradiente si scrive come:
			\[
				\partial_\mu =\left(\frac{\partial}{c\, \partial t}\ ,\ \grad\right)
			\]
			\begin{Achtung}
					Il quadrigradiente ha l'indice in basso.		
			\end{Achtung}
			
			Quindi se scrivo $\partial_\mu X^\mu = \dfrac1c \diffp {X^0}t + \div \vec x$ è la \emph{quadridivergenza}.
			
			Inoltre 
			\[
				\partial_\mu\partial^\mu = \left(\frac1{c^2}\diffp[2]{}{t}, -\lapl \right) 
			\]
			è il d'Alambertiano, si indica con $\dalamb$ ed è invariante (``non ha indici liberi'').
			
			Definiamo ora:
			\begin{itemize}
				\item Tempo proprio:
					\begin{equation}
						\gamma\, \de \tau=\de t
					\end{equation}
				\item Quadrivelocità:
					\begin{equation}
						u^\mu = \diff{x^\mu}\tau = \gamma(c, \vec v)
					\end{equation}
				\item Quadriaccelerazione:
					\begin{equation}
						a^\mu = \gamma^2\left( \frac{\gamma^2}c \vec v \cdot \vec a, \frac{\gamma^2\vec v \cdot \vec a}{c^2} \vec v + \vec a\right)
					\end{equation}
				\item Quadrimpulso:
					\begin{equation}
						P^\mu = mu^\mu = (mc\gamma, \vec P)
					\end{equation}
					dove $\vec P = m\gamma \vec v$ è l'impulso relativistico.	
			\end{itemize}
			
			Possiamo anche scrivere il quadrimpulso come $P^\mu = (\frac Ec, \vec P)$ con $E=m\gamma c^2$ un energia.\\
			Un importante invariante è:
			\[
				P_\mu P^\mu=P^\mu P_\mu=m^2 c^2
			\]
			Una particella ha {\it massa nulla} se e solo se viaggia alla velocità della luce se e solo se la quadrinorma del suo quadrimpulso è identicamente nulla.\\
			Un caso particolare di particelle a massa nulla sono i fotoni; il loro quadrimpulso è:
			\[	
				P^\mu=\left(\frac{\hbar \omega}{c},\frac{\hbar \omega}{c}\hat k\right)
			\]
			dove $\omega$ è la frequenza del fotone e $\hat k$ il versore della sua direzione di propagazione.
			In un certo senso se abbiamo un'onda piana standard essa si può pensare come tanti fotoni che viaggiano, e l'$\omega$ dei fotoni coincide con l'$\omega$ dell' onda, così some il $\hat k$. L'ampiezza dell'onda ha invece a che fare con ``quanti'' fotoni ci sono per unità di volume.
			
		\subsection{Elettrodinamica covariante}
			La carica elettrica è un invariante di Lorentz.
			
			La quadricorrente si indica con $J^\mu$ e vale $(c\rho,\vec J)$. Inoltre possiamo scrivere il quadripotenziale come $A^\mu= (V,\vec A)$
			
			Ora scriviamo un po' di equazioni utili in forma covariante:
			\begin{itemize}
				\item Equazione di continuità:
					\[
						\diffp \rho t + \div J = 0 \quad\Longrightarrow\quad \partial_\mu J^\mu = 0
					\]
				\item Equazioni di Maxwell sui potenziali:
					\[
						\dalamb A^\mu = 0
					\]
					posto che valga la gauge di Lorenz, data da $\partial_\mu A^\mu =0$.
			\end{itemize}
			
			Vogliamo ora indagare come variano i campi. Definiamo quindi il seguente tensore:
			\[
				F^{\mu\nu} = \partial^\mu A^\nu - \partial^\nu A^\mu
			\]
			Facendo il conto si scopre che 
			\[
				F^{\mu\nu} = \left(
				\begin{array}{c|ccc}
					0		&	-E_x	& -E_y	& -E_z \\
					\hline 
					E_x	&		0		& -B_z	&  B_y \\
					E_y &  B_z	& 	0		&	-B_x \\
					E_z & -B_y	&  B_x 	& 	0
				\end{array}
				\right)
			\]
			
			Quindi sapendo come si trasforma $F$ so anche come si trasformano $\vec E$ e $\vec B$. Per un boost lungo $x$ vale 
			\[
			 \begin{array}{cc}
				E_x \rar E_x												& B_x \rar B_x											\\
				E_y \rar \gamma (E_y - \beta B_z) 	& B_y \rar \gamma (B_y + \beta E_z) \\
				E_z \rar \gamma (E_z + \beta B_y) 	& B_z \rar \gamma (B_z - \beta E_x)
			 \end{array}
			\]
	\section{Esercizi}
	\subsection{Campi dentro a una corona cilindrica}
		Nella regione di spazio $a^2 < x^2 + y^2 < 9 a^2$, infinitamente estesa lungo $\hat z$, priva di cariche e correnti, è presente il seguente campo elettrico:
		\[
			\vec E(x,y,z,t)=\frac{x\hat x+y\hat y}{x^2+y^2}E_0ae^{i(kz-\omega t)}
		\]
		
		Si determini il campo magnetico ad esso associato e la relazione che lega $k$ e $\omega$. Sapendo che nelle
		regioni $a^2 > x^2 + y^2$ e $x^2 + y^2 > 9 a^2$ non vi sono sorgenti né campi, si determinino correnti e/o
		cariche superficiali presenti sulle superfici $x^2 + y^2 = a^2$ e $x^2 + y^2 = 9 a^2$. Si determini la media
		temporale della forza per unità di area esercitata dai campi su tali superfici.

	\subsection{Ruota collegata ad un circuito}
		Come non mostrato in figura, una ruota costituita da numerosi raggi conduttori può ruotare in modo tale che una spazzola mantenga sempre il contatto elettrico con un solo raggio alla volta. In presenza di un campo magnetico esterno $\vec B$ costante ed uniforme, diretto lungo l'asse della ruota, si chiude il circuito che collega la ruota in serie ad un induttanza $L$ e ad una pila di voltaggio $V$. La ruota, inizialmente ferma, ha raggio $R$ e momento d'inerzia $K$. Trascurando resistenze meccaniche ed elettriche, si calcolino in funzione del tempo la corrente nel circuito e la velocità angolare della ruota. Si discuta il bilancio energetico del sistema.
		
		\begin{sol}
		 Qui la soluzione.
		\end{sol}
	
	\subsection{Campo magnetico iperbolico}
		Nella regione $\abs z<a$ priva di cariche e correnti è presente il seguente campo magnetico rapidamente variabile:
		\[
		 \vec B = (0,B_0,0) \cosh(\kappa z)e^{i(qz-\omega t)}
		\]
		Si determinino il campo elettrico $\vec E$ associato a $\vec B$ e la relazione che lega le grandezze reali $\kappa, q$ e $\omega$. Si calcolino la media temporale del vettore di Poynting in questa striscia e la media temporale della pressione esercitata dai campi sulla superficie $z=a$.
		
	\subsection{Un dipolo lineare e uno circolare}
		Si consideri il sistema irraggiante non relativistico dato da una carica $q_1$ in moto non relativistico con legge ${\vec r}_1(t)=(0,0,a\cos(\omega t))$ e da una carica $q_2$ in moto con legge oraria ${\vec r}_2(t)=(l\cos(\omega t),l\sin(\omega t),0)$, con $q_1a=q_2l=d$. Si determinino polarizzazione ed intensità della luce emessa a grande distanza rispettivamente nelle direzioni $\hat x,\hat y,\hat z$.
	
	\subsection{Trasformazione di potenziali}
		Nel sistema di laboratorio sono presenti dei campi elettromagnetici indipendenti dal tempo descritti dai seguenti potenziali elettrodinamici:
		\[
		 V(\vec r)= -V_0 \frac{\abs y}l, \qquad \vec A(\vec r)=A_0(0,0,\frac{\abs y}l)
		\]
		
		Si discuta al variare del rapporto $\frac{V_0}{A_0}$ se esiste un sistema di riferimento in cui si annulli il potenziale scalare ovvero il potenziale vettore, e si determini il valore dei potenziali elettrodinamici in tale sistema.
		
	\subsection{Decadimento a tre}
		Una particella instabile di massa $M$ ferma decade in due particelle di massa $m_1$ e $m_2$ ovvero in tre particelle di massa $m_a,m_b$ e $m_c$. Nel primo caso, si dimostri che la cinematica del decadimento non ha gradi di libertà rilevanti (la scelta della direzione lungo la quale si allontanano i due prodotti è irrilevante data l'isotropia dello spazio). Quanti sono invece i gradi di liberta rilevanti per il decadimento a tre corpi? In questo secondo caso, si sospetta tuttavia che il decadimento avvenga in due stadi: prima la particella $M$ decade nella particella di massa $m_a$ e in una di massa $m_x$, quindi quest'ultima decade a sua volta nelle due particelle di massa $m_b$ e $m_c$, ma data la risoluzione spaziale e temporale del rilevatore non è possibile distinguere i due decadimenti successivi perché il tempo di vita della particella $x$ è troppo breve. Misurando solo i quadrivettori energia-impulso dei prodotti finali di massa $m_a,m_b$ e $m_c$, come fareste ad accertare se il decadimento avviene in due stadi con la produzione della particella di massa $m_x$ ovvero direttamente tramite un unico decadimento a tre corpi? 

\end{document}
