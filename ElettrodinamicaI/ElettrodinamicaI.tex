\documentclass[a4paper,10pt,oneside]{math_article}

\newcommand{\rot}{\nabla \times}
\renewcommand{\div}{\nabla \cdot}
\newcommand{\grad}{\nabla}
\newcommand{\lapl}{\nabla^2}

\begin{document}
	\section{Teoria}
		\subsection{Equazioni di Maxwell}
			Le equazioni di Maxwell per l'elettrodinamica sono le seguenti:
			\begin{align}\displaystyle
				\div E &= 4\pi \rho \label{eq:MaxGauss}\\ 
				\div B &= 0 \label{eq:MaxMonopole} \\
				\rot E &= -\frac1c \diffp Bt \label{eq:MaxFaraday}\\
				\rot B &= \dfrac1c \diffp Et + \dfrac{4\pi}c \vec J \label{eq:MaxAmpere}
			\end{align}
			
		\subsection{Potenziali}
			Possiamo definire un potenziale vettore $\vec A$ tale che
			\begin{equation}\label{eq:DefVecPot}
			 \rot A = B
			\end{equation}

			Prendo il rotore di \ref{eq:DefVecPot} e sostituisco nella \ref{eq:MaxAmpere} ottenendo
			\begin{equation}
				\rot (\rot A) = \grad(\div \vec A) - \lapl \vec A = 
				%%TODO
			\end{equation}
			
		\subsection{Irraggiamento di un dipolo}
			Consideriamo un sistema di sorgenti localizzate (con scala $a$) monocromatiche, ossia la loro dipendenza dal tempo è periodica con frequenza $\omega$. Sia $\lambda=\frac{2\pi c}{\omega}$ la lunghezza d'onda.
			Supponiamo quindi che $\rho(\vec x, t) = \rho(x)e^{i\omega t}$, e sia $\vec d= \int \rho(x')\vec x' dx'^3$. Calcoliamo i campi in un punto $\vec R = R\cdot \hat n$.
			
			In \emph{zona d'onda}, ossia dove $R >> a$, $R >> \lambda$, vale la seguente formula a patto che valga anche $\lambda >> a$:
			\begin{equation}
				\begin{cases}
					\vec B = \frac{k^2}{R} e^{ikR} (\hat n \wedge \vec d)\\
					\vec E = \vec B \wedge \hat n
				\end{cases}
			\end{equation}
			
			Potrebbero esserci problemi di segno.
			%TODO Serivirà aggiungere come si ricava%

			Quindi $\vec E$ e $\vec B$ sono proporzionali a $\sin \theta$; $\vec E$ è diretto lungo $\hat \theta$, $\vec B$ è diretto lungo $\hat \phi$, quindi il vettore di Poynting è diretto lungo $\hat n$.
			
			
		

\end{document}
