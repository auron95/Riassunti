\documentclass[a4paper,10pt,oneside]{math_article}

\usepackage[nohug,small]{diagrams}


\newcommand{\Cyc}{\mathbb Z}
\renewcommand{\phi}{\varphi}
%\renewcommand{\Phi}{\varPhi}
\newcommand{\herm}[2]{\left(#1 | #2\right)}
\newcommand{\id}{I}
\newcommand{\class}[2][\bC]{#1^{#2}_\#}
\newcommand{\func}[2][\bC]{#1^{#2}}
\newcommand{\tensor}{\otimes}
\renewcommand{\bar}{\overline}


\newcommand{\Reg}{\cR}
\DeclareMathOperator{\BReg}{Reg}
\DeclareMathOperator{\Ind}{Ind}
\DeclareMathOperator{\Res}{Res}
\DeclareMathOperator{\std}{std}

\let\conj\overline
\title{Riassunto di Elementi di Teoria delle Rappresentazioni}
 \author{Matteo Migliorini}
 
\date{}
 
 
\begin{document}
 
 
 \maketitle
 
 \cleardoublepage	
 \section*{Introduzione}
  Questo file è un riassunto del corso dei proff. Angelo Vistoli e Jacopo Gandini. Tuttavia tenete presente che tutte le cose di Algebra 1 o inferiori verranno date per note (magari verranno \TeX ate in seguito, ma non sperateci troppo).
  
  Inoltre la struttura degli argomenti è ripresa dal Serre, e integrata con gli appunti del corso (per i quali ringrazio Balbo e Clara).
	Grazie anche a Dario per il supporto nello studio.
  \subsection*{DISCLAIMER}
    Il file non è ancora completo, usatelo a vostro rischio e pericolo.

 \cleardoublepage
 \tableofcontents
 \cleardoublepage
 
 \section{Precorso}

\subsection{Base di uno spazio vettoriale}
  Oltre alla definizione solita data in algebra lineare, esiste una definizione più astratta di base di uno spazio vettoriale.
  \begin{mydef}
    Sia $V$ uno spazio vettoriale. Si dice \emph{base} di $V$ un insieme di indici $I$ con un'applicazione $e: I \rar V$ se per ogni $W$ spazio vettroriale e per ogni $f: I \rar W$, esiste un unica $\phi: V \rar W$ lineare tale che il seguente diagramma commuta

    	\tridiag IeV\phi Wf

  \end{mydef}

  \begin{myprop}
    Sono fatti equivalenti:
    \begin{enumerate}
    \item $e: I \rar V$ è una base;
    \item Ogni $v \in V$ si scrive come $\sum_i a_ie_i$ con $a: I\rar V$ a supporto finito. 
    \end{enumerate}
  \end{myprop}
  \begin{proof}
    Dimostriamo le due implicazioni.
    \begin{itemize}
      \item $2\Rar 1$
      
      Se una tale $\phi$ esiste, allora deve essere $\phi(v)=\phi(\sum_i a_i e_i) = \sum a_i \phi(e_i) = \sum a_i f_i$. D'altra parte questa $\phi$ funziona.
      
      \item $1 \Rar 2$
      
      Considero $W = \{\sum_i a_ie_i : a:I\rar \bC \mbox{ a supporto finito}\}$. $W$ è un sottospazio di $V$ e vale ovviamente che $\im e \subseteq W$. Sia ora $\pi$ una proiezione sul quoziente $V/W$, e sia $f: I \rar V/W$ l'applicazione nulla.
	
	\tridiag IeV\pi{V/W}f
      
      Il diagramma sopra commuta, dato che $\im e \subseteq W$. Ma commuta anche se al posto di $\pi$ metto l'applicazione nulla. Quindi per l'unicità data dalla definizione di base $V/W$ deve essere banale, cioè $V=W$.
    \end{itemize}
  \end{proof}
  
\subsection{Moduli}
  \begin{mydef}
    Sia $A$ un anello commutativo con identità. Si dice $A$-modulo un gruppo abeliano additivo $(M,+)$ con un operazione $\cdot: A\times M \rar M$ distributiva da entrambe le parti, associativa e tale che $1_A \cdot x = x$.
  \end{mydef}
  
  Un modulo non è altro che la generalizzazione di uno spazio vettroriale, dove gli scalari non sono più un campo ma bensì un anello.
  
  In questo corso la nozione di modulo sarà data solo su anelli della seguente forma.
  \begin{mydef}
    Sia $G$ un gruppo. Si indica con $\bC[G]$ l'anello formato dalle combinazioni lineari formali a coefficienti complessi degli elementi di $G$, ossia gli elementi della forma $\sum_i a_ig_i$ dove la somma è deifinita in modo ovvio e il prodotto coincide con quello del gruppo, esteso in modo che sia distributivo. 
  \end{mydef}
  
  Un $\bC[G]$-modulo lo possiamo quindi pensare moralmente come un $\bC$-spazio vettoriale dove oltre a una moltiplicazione per uno scalare complesso posso anche moltiplicare per un elemento del gruppo $G$ (sarà tutto più chiaro -- forse -- quando introduceremo il concetto di rappresentazione).

\subsection{Prodotto tensore}
  \begin{mydef}[Prodotto tensore]
    Siano $V, W$ due $\bK$-spazi vettoriali. Si dice \emph{prodotto tensore} di $V$ e $W$, e si indica come $V\tensor W$, uno spazio vettoriale con una funzione bilineare $\tensor: V \times W \rar V\tensor W$ tale che per ogni applicazione bilineare $f: V\times W \rar Z$ esiste un unica applicazione lineare $\phi: V\tensor W \rar Z$ che fa commutare il seguente diagramma:
    
      \tridiag{V\times W}{ \tensor }{V\tensor W}{\phi}{Z}{f}
  
  \end{mydef}
  
  Bisognerebbe dimostrare che sta roba esiste, ma adesso non ho sbatty.

  Si può dimostrare che se $I \xrar v V$ è una base di $V$, e $J \xrar w W$ è una base di $W$, allora $(i,j) \mapsto v_i \tensor w_i$ è una base di $V\tensor W$.

  \begin{mydef}[Prodotto tensore di applicazioni lineari]
    Si può riportare la definizione di prodotto tensore alle applicazioni lineari. Date $f:V \rar V', g: W \rar W'$, si definisce $f\tensor g: V\tensor W \rar V'\tensor W'$ come
      \[
	(f \otimes g)(v\otimes w)=f(v)\otimes g(w) 
      \]
    
    Il prodotto tensore tra applicazioni fa commutare il seguente diagramma:
    
      \quaddiag{V\times W}{f\times g}{V'\times W'}{}{V'\tensor W'}{}{V \tensor W}{f\tensor g}   
   
   \end{mydef}

  \begin{myobs}
   Nella costruzione del prodotto tensore, noi vorremmo costruire uno spazio che ha per base il prodotto cartesiano delle basi. Tuttavia, non riesco a dotare il prodotto cartesiano della struttura di spazio vettoriale.
   
   Infatti se voglio che l'applicazione $\tensor$ sia bilineare, allora vale che $0 \tensor w = v \tensor 0 = 0$, quindi l'applicazione non è iniettiva, ma non è neanche suriettiva e in generale esistono elementi che non sono della forma $v \tensor w$, ma sono combinazioni lineari di elementi di questa forma.
  \end{myobs}
  \begin{myexample}
  
   Ad esempio, in $\bC \tensor \bC$ come spazi vettoriali su $\bR$, l'elemento $1 \tensor i + i \tensor 1$ non è della forma $v\tensor w$: altrimenti innanzitutto avremmo $v=w$ per simmetria, e scrivendo tutto nella base canonica ottengo, ponendo $v=a+bi$
   \[
    v\tensor v= (a+bi)\tensor (a+bi) = a^2(1\tensor 1) + b^2 (i\tensor i) + ab(i\tensor 1 +1\tensor i)
   \]
   da cui per le proprietà della base su $\bC\tensor \bC$ ottengo $a^2=b^2=0, ab=1$ assurdo.

  \end{myexample}
  
  La potenza tensoriale $V^{\tensor n}$ si definisce nel modo ovvio. Introduciamo ora invece la potenza simmetrica e la potenza esterna.
   
  \begin{mydef}[Potenza simmetrica]
   Indichiamo con $V^n$ il prodotto cartesiano $\overbrace{V \times V \times \dots \times V}^{n \textrm{ volte}}$.
   
   Si dice \emph{potenza simmetrica} $n$-esima di uno spazio vettoriale $V$ uno spazio vettroriale $S^nV$ con un'applicazione multilineare $\sigma$ da $V^n$ a $S^nV$ che sia simmetrica, ossia invariante per permutazione delle entrate, e tale che per ogni altro $Z$ con un'applicazione lineare $\theta: V^n \rar Z$ simmetrica esiste un unica $\phi: S^n \rar Z$ che fa commutare 
   
      \tridiag{V^n}{\sigma}{S^nV}{\exists!\phi}{Z}{\theta}
  
  \end{mydef}
  
  \begin{mydef}[Potenza alternante]
   Si dice \emph{potenza esterna} $n$-esima di uno spazio vettoriale $V$ uno spazio vettroriale $\Lambda^nV$ con un'applicazione multilineare $\wedge$ da $V^n$ a $\Lambda^nV$ che sia stavolta alternante, che significa che $\sigma$ deve cambiare segno se scambio due entrate, e tale che per ogni altro $Z$ con un'applicazione lineare $\theta: V^n \rar Z$ alternante esiste un unica $\phi: \Lambda^n \rar Z$ che fa commutare 
   
      \tridiag{V^n}{\wedge}{\Lambda^nV}{\exists!\phi}{Z}{\theta}
  
  \end{mydef}
  
  \begin{myobs}
   Possiamo pensare la potenza simmetrica come un sottospazio di $V^{\tensor n}$, e più precisamente come il sottospazio invariante per permutazione delle coordinate. Detto meglio, su $V^{\tensor n}$, su cui possiamo far agire il gruppo simmetrico $S_n$ per permitazione, ossia in modo che $v_1 \tensor \dots \tensor v_n \xmapsto{\sigma} v_{\sigma(1)} \dots v_{\sigma(n)}$. Non l'ho ancora definita ovunque (vedi sopra) ma si estende in un unico modo per linearità. La potenza simmetrica è lo spazio lasciato fisso da $S_n$.
  \end{myobs}

  \begin{proof}
   Basta considerare la mappa $(v_1,\dots,v_n) \mapsto \sum_{\sigma \in S_n} v_{\sigma(1)} \dots v_{\sigma(n)}$, e notare che l'immagine è proprio quella cercata dentro $S_n$ e usare la definizione di prodotto tensore per mostrare che quella è $S^nV$
  \end{proof}



 \section{Rappresentazioni}
	\begin{mydef}
		Sia $G$ un gruppo finito, e sia $V$ un $\mathbb C$-spazio vettoriale. Si dice \myname{rappresentazione} di $G$ su $V$ un'azione lineare di $G$ su $V$, ovvero un omomorfismo da $G$ in $GL(V)$.
	\end{mydef}

	Le rappresentazioni si possono indicare equivalentemente come delle applicazioni $G \times V \rightarrow V$ con $(g,v)\mapsto gv$, con $gv$ lineare, oppure come mappe $g \mapsto \rho(g)$, dove $\rho(g) \in GL(V)$. In generale mi piace di più il secondo modo. A volte userò anche la notazione $g \mapsto \rho_g$, per evitare quintali di parentesi.

	In alternativa, si possono vedere le rappresentazioni come $\bC[G]$-moduli, dove $\bC[G]$ è l'anello delle combinazioni $\bC$-lineari di elementi di $G$.

	\begin{myobs}
		Nello studio delle rappresentazioni si usano diverse notazioni. In particolare spesso si chiama rappresentazione sia l'omomorfismo, sia lo spazio vettoriale. Il motivo dietro a questa usanza è che la rappresentazione non è altro che un $\bC[G]$-modulo, che significa moralmente uno ``spazio vettoriale'' dove negli scalari ci sta anche $G$: così come quando pensiamo uno spazio vettoriale non ci riferiamo alla funzione $\bK \times V \rar V$, ma all'insieme in cui è intrinsecamente definita la moltiplicazione per scalari, lo stesso faremo per le rappresentazioni.
		
		Quindi quello che faremo in questo riassunto è la seguente cosa: pensando le rappresentazioni come dei $\bC[G]$-moduli, quando non ci sarà ambiguità, chiameremo rappresentazione uno spazio vettoriale $V$ su cui è definita una moltiplicazione a sinistra per gli elementi del gruppo, che quindi indicherò spesso, invece di usare la notazione $\rho_g(v)$, semplicemente con $g\cdot v$. 
	\end{myobs}

	\begin{mydef}
		Si definisce \myname{grado} di una rappresentazione $\rho$
		\[\deg \rho := \dim V_\rho\]
	\end{mydef}


	Sia ora $[\rho_g]_\cB$ la \myname{matrice associata} all'applicazione lineare $\rho_g$ in base $\cB$. Allora $[\rho_g]$ è quadrata, di ordine $\deg \rho$, invertibile ($\det [\rho_g]\ne 0$) e vale $[\rho_g] [\rho_h]=[\rho_{gh}]$. Inoltre, se $[\rho_g]_i^j$ è il coefficiente di $[\rho_g]$ di riga $i$ e colonna $j$, allora vale
	\[
		[\rho_{gh}]_i^k=\sum_j [\rho_g]_i^j [\rho_h]_j^k
	\]

	Dato che per gruppi finiti $[\rho_g]^n$ deve fare l'identità per $n=\abs G$, allora il suo polinomio minimo divide $x^n-1$, e visto che non ha fattori ripetuti $[\rho_g]$ è diagonalizzabile, con solo radici $n$-esime dell'unità sulla diagonale. 

	\begin{mydef}
		Date $\rho,\sigma$ due rappresentazioni di $G$ su $V_\rho,V_\sigma$ rispettivamente, si dice \myname{omomorfismo di rappresentazioni} un omomorfismo $\phi$ di spazi vettoriali $V_\rho \rightarrow V_\sigma$ tale che $\rho(g) \circ \phi = \phi \circ \sigma(g)$. 
		In altre parole, deve far commutare il seguente diagramma
		\[
			\begin{diagram}
	V_\rho         & \rTo^{\phi}  & V_\sigma\\
	\dTo<{\rho(g)} &           	 & \dTo>{\sigma(g)}\\
	V_\rho         & \rTo^{\phi}  & V_\sigma
			\end{diagram}
		\]
		Nella notazione dei $\bC[G]$-moduli, la stessa condizione si scriverebbe come $g\cdot \phi(v)=\phi(g\cdot v)$, dove il prima moltiplicazione per $g$ è quella su $V_\rho$, la seconda è quella su $V_\sigma$.
			
	Analogamente, si definisce \myname{endomorfismo} di $\rho$ un omomorfismo da $\rho$ in $\rho$, e \myname{isomorfismo di rappresentazioni} un omomorfismo che è un isomorfismo di spazi vettoriali.
	\end{mydef}

	\begin{myobs}
		La definizione di omomorfismo coincide con quella per moduli generici.
	\end{myobs}

	\begin{myobs}
		Con $\Hom(V_\rho,V_\sigma)$ indicheremo gli omomorfismi tra $V_\rho$ e $V_\sigma$ come $\bC$-spazi vettoriali. Quando parliamo di omomorfismi di rappresentazione, scriveremo invece indifferentemente $\Hom(\rho,\sigma)$ oppure $\Hom_G(V_\rho,V_\sigma)$.
	\end{myobs}

	Come al solito, spesso identificheremo le rappresentazioni isomorfe, senza scrivere tutte le volte \emph{a meno di isomorfismo}.

	Date due rappresentazioni $\rho,\sigma$, con due basi $\cB$ e $cS$ sui relativi spazi vettoriali, se le due rappresentazioni sono isomorfe, allora esiste una matrice $T$ invertibile tale che
	\[
		[\rho_g]_\cB = T\inv [\sigma_g]_cS T \qquad \forall g\in G
	\]
	La matrice $T$ è quella che induce l'isomorfismo tra $V_\rho$ e $V_\sigma$.
	Quindi le matrici analoghe (associate allo stesso $g$) sono coniugate attraverso un'unica matrice invertibile.
			
	Vediamo ora qualche esempio di rappresentazione.

	\begin{myexample}[Rappresentazione banale]
		Dato un qualsiasi gruppo $G$ e un qualsiasi spazio vettoriale $V$, una sua rappresentazione possibile è quella banale, ossia  $g\cdot v = v$ per ogni $g\in G$ e per ogni $v\in V$.
	\end{myexample}
	\begin{myexample}[Rappresentazione regolare]
		Così come dato un qualsiasi campo $\bK$ possiamo considerare $\bK$ come spazio vettoriale su se stesso, pensando il prodotto in $\bK$ come prodotto per scalare, lo stesso si può fare con un modulo.
		
		Mettiamo quindi su $\bC[G]$ la struttura di $\bC$-modulo. Possiamo vedere $C[G]$ come spazio vettoriale su $\bC$ in due modi:
		\begin{itemize}
		 \item $\bC[G]$ è uno spazio vettoriale con una base indicizzata da $G$. Chiameremo i vettori della base $e_g$, oppure semplicemente $g$ (per evitare notazioni come $e_e$). $G$ agisce su $\bC[G]$ per moltiplicazione a sinistra: $g\cdot e_h = e_{gh}$.
		 \item $\bC[G]$ è lo spazio vettoriale delle funzioni da $G$ in $\bC$. $G$ agisce in questo modo: l'elemento $g$ manda $f(x)$ in $f(g\inv x)$.
		\end{itemize}
		
		\begin{myprop}
		 Le due rappresentazioni elencate sopra sono effettivamente isomorfe.
		\end{myprop}
		\begin{proof}
		 L'isomorfismo è dato da \[\phi\left(\sum_{g\in G} a_g\cdot g\right)= \left(g\mapsto a_g\right)\]
		 La verifica che è effettivamente un isomorfismo è lasciata per esercizio.
		\end{proof}

		Anche se sono effettivamente la stessa cosa, quando pensiamo $\bC[G]$ come funzioni da $G$ in $\bC$, scriveremo $\func \bC G$.
		
		\iffalse
				Sia $G$ un gruppo finito, e sia $V$ uno spazio vettoriale con base indicizzata da $G$ (ci starebbe una digressione sulla definizione di base). La rappresentazione regolare $\Reg$ è quella rappresentazione che associa a ogni elemento del gruppo l'azione di \emph{moltiplicazione a sinistra} sulla base, ossia $\Reg_g: e_h \mapsto e_{gh}$.
				
				Alternativamente, posso vedere $V$ come lo spazio $\mathbb C[G]$ delle funzioni da $G$ in $\mathbb C$ (l'indentificazione è quella che manda il vettore $\sum_i a_ie_{g_i}$ nella mappa $g_i \mapsto a_i$). In tal caso la rappresentazione regolare diventa
			\[
				\Reg_g(f): x \mapsto  f ( g\inv x) 
				\]
				
				Questa rappresentazione è particolarmente importante perché coincide con $\bC[G]$ visto come $\bC[G]$-modulo, che è sostanzialmente la prima definizione.
				
				Vedremo l'importanza di questa rappresentazione.

		\fi
	\end{myexample}
	\begin{myexample}[Rappresentazione per permutazione di un insieme]
		Se ho un'azione di $G$ su un insieme $X$, posso considerare $V$ con $X\rightarrow V$ una base, e costruire la rappresentazione che permuta la base tramite l'azione su $X$. Anch'essa è una rappresentazione.
		
		Notiamo che se $X=G$ con l'azione di moltiplicazione a sinistra, otteniamo la rappresntazione regolare.
	\end{myexample}

	\subsection{Operazioni tra rappresentazioni}
	Vediamo quali operazioni si possono definire tra le rappresentazioni.

	\begin{mydef}[Somma di rappresentazioni]
		Siano $\rho, \sigma$ due rappresentazioni di $G$ in $V_\rho,V_\sigma$ rispettivamente. Si definisce la \myname{somma di rappresentazioni} $\rho+\sigma$ come la rappresentazione di $G$ in $V_\rho\oplus V_\sigma$ tale che 
		\[
		(\rho+\sigma)_g(u+v)=\rho_g(u)+\sigma_g(v)\qquad \forall u\in V_\rho, v\in V_\sigma
		\]
		
		In forma più snella, potremmo anche scrivere
		\[
		 g\cdot(u+v) = g\cdot u + g\cdot v
		\]

	\end{mydef}

	Matricialmente $(\rho+\sigma)_g$ si rappresenta come
	\[
		[(\rho+\sigma)_g]=\left[
			\begin{array}{c|c}
				[\rho_g] & 0 \\
				\hline
				0 & [\sigma_g]
			\end{array}
		\right]
	\]
	dove ovviamente si intende che la base ha i primi vettori in $V_\rho$, i secondi in $V_\sigma$.

	Inoltre per come è definito il grado vale $\deg(\rho+\sigma)=\deg\rho+\deg\sigma$.

	\begin{mydef}
		Un sottospazio $W$ di $V_\rho$ si dice $G$-invariante se per ogni $g\in G$ vale $\rho_g(W)\subseteq W$.
	\end{mydef}

	\begin{mydef}
		Si dice \myname{sottorappresentazione} di $\rho$ la restrizione dei $\rho_g$ a un sottospazio vettoriale $G$-invariante. Nel linguaggio dei moduli si parla di sottomodulo.
	\end{mydef}

	\begin{myexample}
		Data la rappresentazione regolare $\Reg$, allora il sottospazio generato da $\sum_{g\in G} e_g$ è $G$-invariante, e la sottorappresentazione indotta è quella banale.
		
		Lo stesso vale per una qualsiasi rappresentazione per permutazione.
	\end{myexample}
			
	Ora invece, data una rappresentazione su $V$, vediamo come costruirne una sul duale $V^*$.
	\begin{mydef}
		Sia $\rho$ una rappresentazione di $G$ su $V$. Dato $f\in V^*$ e $v\in V$, sia ora $\scalar fv$ la dualità (è solo un modo figo per chiamare l'applicazione $\scalar fv \mapsto f(v)$). Definiamo la \myname{rappresentazione duale} $\rho^*$ come l'unica rappresentazione tale che:
		\[
		\scalar{\rho^*_g(f)}{\rho_g (v)}=\scalar fv
		\]
	\end{mydef}

	In altre parole, $\rho^*_g$ è l'applicazione \myname{trasposta} di $\rho_g\inv$.

	Il fatto che io usi l'inverso è perché l'applicazione trasposta inverte l'ordine delle composizioni, mentre noi vogliamo un isomorfismo: quindi mettendo l'inverso l'ordine torna magicamente a essere quello giusto.

	Oltre alla somma, abbiamo anche un prodotto tra rappresentazioni. Esso è definito come segue. 

	Ora siamo pronti a dare la definizione di prodotto di rappresentazioni.

	\begin{mydef}[Prodotto di rappresentazioni]\label{def:RapprProd}
		Siano $\rho,\sigma$ due rappresentazioni su $V_\rho,V_\sigma$. Si definisce il prodotto $\rho\sigma$ come la rappresentazione su $V_\rho \otimes V_\sigma$ che soddisfa
		\[
			(\rho\sigma)_g = \rho_g \otimes \sigma_g
		\]
		
		Ancora, pensando in termini di moduli scriveremmo, se $v\in V_\rho, w\in V_\sigma$, che
		\[
		 g\cdot (v\tensor w) = (g\cdot v) \tensor (g\cdot w)
		\]

	\end{mydef}
	




	\subsection{Rappresentazioni irriducibili e Schur}

	Iniziamo con un risultato prelimnare.

	\begin{mytheorem}[Maschke]\label{Th:SupplInv}
		Sia $\rho$ una rappresentazione su $V$, e sia $W$ un sottospazio $G$-invariante. Allora esiste un supplementare $W_0$ anch'esso $G$-invariante.
	\end{mytheorem}
	
	Ci sono molte dimostrazioni di questo teorema. Dato che il campo è $\bC$, possiamo utilizzare la seguente:
	\begin{proof}
		Dato un qualsiasi prodotto hermitiano su $V$, allora il prodotto hermitiano dato da $\sum_{g\in G}\herm {\rho_g(x)}{\rho_g(y)}$ è invariante per $G$ (da notare che sto usando fortemente la finitezza di $G$), quindi si può facilmente verificare che $W^\perp$ è $G$-invariante.
		
		Inoltre, dato che rispetto al prodotto hermitiano $G$ è invariante, significa che i $\rho_g$ sono ortogonali rispetto a una base ortonormale, e quindi sono applicazioni unitarie.
	\end{proof}

	In un campo generico, vale la dimostrazione seguente.
	\begin{proof}
		Sia $\pi$ una qualsiasi proiezione di $V$ su $W$, e sia $\pi_0$ la proiezione pesata data da
		\[
		\pi_0=\frac1{\abs G} \sum_{g\in G} \rho_g \circ \pi \circ \rho_g\inv
		\]
		Dato che $\pi$ lascia fisso $W$ allora anche $\pi_0$ lo lascia fisso (sto usando che $\rho_g$ stabilizza $W$). Inoltre $\pi_0(V) \subseteq W$. Quindi anch'essa è una proiezione con $\Ker \pi_0=W_0$ . Inoltre $\pi_0$ commuta con i $\rho_g$, come si vede calcolando $\rho_g \circ \pi_0 \circ \rho_g\inv$.
		
		Quindi se $w\in W_0$, allora $\pi_0 (\rho_g(w))=\rho_g(\pi_0(w))=0$, quindi $W_0$ è $G$-invariante ed è facile verificare che è anche un supplementare.
	\end{proof}

	Ogniqualvolta abbiamo una sottorappresentazione di $\rho$, siamo quindi in grado di spezzare $\rho$ in somma di due sottorappresentazioni più piccole. 

	Data una rappresentazione, possiamo reiterare il procediemnto finché non è più possibile trovare sottorappresentazioni non banali.

	\begin{mydef}
		Una rappresentazione che non ammette sottorappresentazioni non banali si dice irriducibile.
	\end{mydef}

	Banalmente, tutte le rappresentazioni di grado 1 sono irriducibili, ma in generale non sono le uniche.

	Possiamo quindi decomporre una rappresentazione fino a quando gli addendi non sono tutti irriducibili.

	Ci piacerebbe affermare che la decomposizione in irriducibili è unica (dove unica è inteso come al solito a meno dell'ordine). Tuttavia non è così banale dimostrarlo. Per farlo ci serve che se possiamo immergere una rappresentazione irriducibile in una somma, allora possiamo immergerla in almeno uno dei fattori. (Sì, vogliamo in un qualche senso che irriducibile $\Rightarrow$ primo)

	Ci viene in aiuto il Lemma di Schur.

	\begin{mytheorem}[Lemma di Schur]
		Siano $\rho,\sigma$ due rappresentazioni irriducibili, e sia $\Phi$ un omomorfismo. Allora o $\Phi \equiv 0$ oppure $\Phi$ è un isomorfismo.
	\end{mytheorem}
	\begin{proof}
		Notiamo che $\Ker \Phi$ è banale oppure è tutto $V_\rho$, dato che $\rho$ è irriducibile. Nel secondo caso quindi $\Phi\equiv 0$, altrimenti si vede che analogamente $\Im \Phi$ è tutto $V_\sigma$. Quindi $\Phi$ è un isomorfismo.
	\end{proof}

	Questo vale per qualsiasi campo.
	
	Ora vediamo due facili corollari, che però dipendono fortemente dalle caratteristiche di $\bC$.

	\begin{mycor}
		Sia $\rho$ irriducibile, e sia $\Phi: \rho \rar \rho$ un endomorfismo di rappresentazione. Allora $\Phi$ è una moltiplicazione per uno scalare. 
	\end{mycor}
	\begin{proof}
		Sia $\lambda$ un autovalore di $\Phi$ (esiste sempre perché siamo su $\bC$: allora $\Phi-\lambda\id$ ha $\Ker$ non banale e quindi è identicamente nulla.
	\end{proof}

	\begin{mycor}
		Date $\rho,\sigma$ irriducibili, e $\Phi: V_\rho \rightarrow V_\sigma$ lineare, sia $\bar \Phi = \frac 1{\abs G} \sum_{g\in G} \sigma_g\inv \Phi \rho_g$. Allora
		\begin{itemize}
		\item Se $\rho \not \isom \sigma$ allora $\bar \Phi\equiv 0$;
		\item Se $\rho = \sigma$, allora $\bar \Phi = \frac{\Tr(\Phi)}{\deg \rho}\id$.
		\end{itemize}
	\end{mycor}
	\begin{proof}
		Deriva tutto dal lemma di Schur. Il fatto che lo scalare si $\frac{\Tr \Phi}{\deg \rho}$ viene dal fatto che $\Tr \bar \Phi = \Tr \Phi$ (basta fare il conto) e che $\Tr \bar \Phi = \lambda \deg \rho$.
	\end{proof}

	Adesso possiamo concludere il claim precedente.
	\begin{myprop}
		Ogni rappresentazione si può scrivere in maniera unica come somma di rappresentazioni irriducibili.
	\end{myprop}
	\begin{proof}
		Siano $\rho = \rho_1 + \rho_2 + \dots + \rho_n = \sigma_1 + \dots +\sigma_m$ due decomposizioni in irriducibili della stessa rappresentazione. Prendiamo $\sigma_1$ e la immergiamo in modo canonico in $\rho$. A questo punto, restringendosi ai $\rho_i$, abbiamo degli omomorfismi da $\sigma_1$ in $\rho_i$, che non possono essere tutti banali. Quindi per il lemma di Schur $\sigma_1\isom \rho_i$ per qualche $i$. La conclusione per induzione è immediata. 
	\end{proof}

	\begin{mydef}
		Data una rappresentazione $\rho$, si dice componente isotopica una sottorappresentazione della forma $n_i\rho_i$, con $n_i$ massimo possibile e $\rho_i$ irriducibile.
	\end{mydef}

	Le componenti isotopiche sono importanti perché sono caratteristiche.

	\begin{myprop}
		Sia $\rho$ una rappresentazione, e sia $\sigma$ una sua componente isotopica. Sia inoltre $\phi$ un automorfismo di $\rho$. Allora $\phi(\sigma)=\sigma$.
	\end{myprop}

	\begin{proof}
		Considero $\phi(\sigma)$. Poiché sulle altre rappresentazioni irriducubili deve essere l'applicazione nulla per il lemma di Schur, allora per questioni di dimensione deve essere mandata in $\sigma$.
	\end{proof}
  

 \section{Caratteri}

\begin{mydef}
  Si dice \myname{carattere} di una rappresentazione $\rho$ l'applicazione $\chi_\rho: G \rightarrow C^*$ definita da 
  \[
  \chi_\rho(g)=\Tr \rho(g)
  \]
\end{mydef}

\begin{Achtung}
  Il carattere \underline{NON} è un omomorfismo da $G$ in $\bC^*$! Lo è se e solo se $\deg \rho = 1$. 
\end{Achtung}

\begin{myprop}
  Il carattere soddisfa le seguenti:
  \begin{enumerate}
    \item $\chi_\rho(e)=\deg\rho$
    \item $\chi_\rho(g\inv)=\conj{\chi_\rho(g)}$
    \item $\chi_\rho(hgh\inv)=\chi_\rho(g)$
  \end{enumerate}
\end{myprop}

\begin{proof}
  La 1 è ovvia, visto che $[\rho_e]= \id_n$, dove $n=\deg \rho$.
  
  La 2 viene dal fatto che, essendo $\rho_g$ diagonalizzabile con autovalori di norma $1$, e dato che la traccia è la somma degli autovalori, allora
  \[
		\Tr\rho_g\inv = \sum_i \lambda_i\inv = \sum_i \conj{\lambda_i} = \conj{\sum_i \lambda_i} = \conj{\Tr\rho_g} 
  \]
  da cui quello che volevamo.
  
La 3 invece deriva dal fatto che se $g$ e $g'$ sono coniugati, allora anche $\rho_g$ e $\rho_{g'}$ lo sono, e la traccia è invariante per coniugio. Ogni funzione che soddisfa la 3 si chiama \myname{funzione di classe}. Lo spazio delle funzioni di classe è un sottospazio di $\func G$ e lo chiameremo $\class G$.
\end{proof}

Ci chiediamo se il carattere identifica le rappresentazioni, ossia se rappresentazioni diverse hanno caratteri diversi.
Intanto però il carattere distingue il grado di una rappresentazione, che è dato da $\deg \rho = \chi_\rho(e)$.

\begin{mylemma}
I caratteri godono delle seguenti proprietà:
  \begin{itemize}
  \item $\chi_{\rho+\sigma}=\chi_\rho+\chi_\sigma$     
  \item $\chi_{\rho\otimes\sigma}=\chi_\rho\cdot\chi_\sigma$
  \item $\chi_{\rho^*} = \conj{\chi_\rho}$
  \end{itemize}
\end{mylemma}
\begin{proof}
	Per le prime due proprietà basta scrivere tutto rispetto alla base canonica.
	
  Per la rappresentazione duale invece vale 
  \[
		\chi_{\rho^*}(s)=\Tr (\trasp\rho_{s\inv})=\Tr (\rho_s\inv)
  \]
  Dato che la traccia è la somma degli inversi degli autovalori, e questi hanno modulo 1, allora otteniamo $\conj{\chi_\rho}$. 

\end{proof}

\subsection{Prodotto hermitiano}
Introduciamo un prodotto interno su $\func G$, dato da 
\[
  \herm fg = \frac 1{\abs G} \sum_{s\in G} f(s)\conj{g(s)}
\]

Grazie a questo prodotto possiamo effettuare delle proiezioni.

\begin{mylemma}
  Consideriamo $\rho$ una G-rappresentazione su V, e sia $\id$ la rappresentazione banale di grado 1 (irriducibile). Allora 
  \[
  \herm \rho\id = \frac 1{\abs G}\sum_{g\in G} \chi_\rho(g) = \dim V^G
  \]    
  dove con $V^G$ intendiamo il sottospazio dei vettori lasciati fissi da $G$.
\end{mylemma}
\begin{proof}
  Sappiamo che per definizione
  \[
  \herm \rho\id = \frac 1{\abs G}\sum_{g\in G} \chi_\rho(g) = \frac 1{\abs G}\sum_{g\in G} \Tr(\rho_g) 
  \]
  Per la linearità otteniamo $\Tr\left(\frac 1{\abs G}\sum_{g\in G}\rho_g\right)$. Sia quindi $T=\frac 1{\abs G}\sum_{g\in G}\rho_g$.
  
  Sia $v \in V^G$, allora $T(v)=v$. Inoltre per ogni $v$ vale che $T(v) \in V^G$ (si vede dal conto). Quindi $T$ è un proiettore su $V^G$, e quindi ha la traccia voluta.
  
\end{proof}

Costruiamo ora un modo alternativo per esprimere questo prodotto hermitiano. Ci serve la costruzione della rappresentazione sugli omomorfismi.

\begin{mydef}\label{def:HomRepr}
  Siano $\rho, \sigma$ due rappresentazioni di $G$ su $V_\rho,V_\sigma$. Allora costruiamo una rappresentazione su $\Hom(V_\rho,V_\sigma)$ (come spazio vettoriale) con l'azione definita da 
  \[
  \tau_g (\phi) = \sigma_g \circ \phi \circ \rho_g\inv
  \]
\end{mydef}

Per definizione, la rappresentazione su $\Hom(V_\rho,V_\sigma)$ fa commutare il seguente diagramma:
  \quaddiag{V_\rho}{\phi}{V_\sigma}{\sigma_g}{V_\sigma}{\rho_g}{V_\rho}{\tau_g(\phi)}
  

\begin{myprop}
  Esiste un isomorfismo canonico tra $\Hom (V_\rho,V_\sigma)$ e $V_\rho^* \tensor V_\sigma$ (se $V_\rho, V_\sigma$ hanno dimensione finita, altrimenti è falso).
\end{myprop}

\begin{proof}
  La verifica è lasciata per esercizio.
\end{proof}

Noi abbiamo costruito una rappresentazione su $\Hom(V_\rho,V_\sigma)$, ma non tutti sono omomorfismi di rappresentazioni: lo sono infatti solo quelli fissati da $G$ (basta guardare il diagramma). Dato che $\chi_\tau = \conj{\chi_\rho}\chi_\sigma$ per le proprietà del prodotto tensore e della rappresentazione duale, allora vale per il lemma precedente
\[
  \dim \Hom(\rho,\sigma) = \dim \Hom(V_\rho,V_\sigma)^G = \herm {\chi_\tau}\id = \herm{\chi_\sigma}{\chi_\rho}
\]

Notiamo che visto che è un numero intero, allora posso scambiare i due fattori nel prodotto hermitiano, e ottengo
\[
  \herm{\chi_\rho}{\chi_\sigma} = \dim \Hom(\rho,\sigma)
\]

Questo è un risultato \emph{molto importante}, che dice tra l'altro che il prodotto tra caratteri dà un numero naturale. Lo applichiamo subito.





\subsection{Relazioni di ortogonalità}
\begin{mytheorem}[Teorema di ortogonalità dei caratteri]
  Siano $\rho,\sigma$ due rappresentazioni irriducibili. Allora
  \[
  \herm {\chi_\rho} {\chi_\sigma} = \begin{cases}
			1 \mbox{ se } \rho \isom \sigma \\
			0 \mbox{ se } \rho \not\isom \sigma
		      \end{cases}
  \]
\end{mytheorem}
\begin{proof}
  \[
  \herm {\chi_\rho}{\chi_\sigma}=\dim \Hom(\rho,\sigma)
  \]
  e quindi con il lemma di Schur ho la tesi.
\end{proof}

\begin{myprop}\label{pr:IrrCount}
  Sia $\rho$ una rappresentazione decomposta in irriducibili. Allora il numero di addendi isomorfi a $\sigma$, con $\sigma$ irriducibile, è pari a $\herm{\chi_\rho}{\chi_\sigma}$.
\end{myprop}

In particolare, da questo deriva che due rappresentazioni con lo stesso carattere sono isomorfe.

\begin{mytheorem}[Criterio di irriducibilità]
  Una rappresentazione $\sigma$ è irriducibile se e solo se $\herm\sigma\sigma=1$.
\end{mytheorem}

\subsection{Carattere della rappresentazione regolare}
Sia $G$ un gruppo e sia $\Reg$ la sua rappresentazione regolare. Vogliamo indagare come si decompone $\Reg$ in somma di irriducibili. Sia quindi
\[
  \Reg = \sum_i n_i \rho_i
\]
con $n_i \in \mathbb N$. Sia inoltre $\chi_\Reg$ il carattere di $\Reg$ e $\chi_i$ il carattere dei $\rho_i$.
  
\begin{myprop}
  Il carattere della rappresentazione regolare è
  \[
  \chi_\Reg(g) = \abs G \cdot \delta_{ge}
  \]
  dove $\delta_{ij}$ è la \myname{delta di Kronecker}.
\end{myprop}

\begin{proof}
  Sappiamo che la matrice associata a $\Reg_g$ è una matrice di permutazione, e gli elementi sulla diagonale sono $1$ se $g=e$, $0$ altrimenti (la moltiplicazione a sinistra non lascia elementi fissi). Visto che il carattere è la somma degli elementi sulla diagonale, segue la tesi.
\end{proof}

Sorprendentemente, siamo in grado di determinare esplicitamente gli $n_i$.
\begin{myprop}
  Ogni rappresentazione irriducibile $\rho_i$ di $G$ è contenuta $\deg \rho_i$ volte in $\Reg$.
\end{myprop}

\begin{proof}
  Basta applicare la Proposizione \ref{pr:IrrCount}:
  \[
  n_i=\herm{\chi_R}{\chi_i} = \frac1{\abs G}\sum_{g\in G} \chi_R(g)\conj{\chi_i(g)} = \frac1{\abs G}\chi_R(e)\conj{\chi_i(e)} = \deg \rho_i
  \]

\end{proof}

Vediamo anche una dimostrazione alternativa, che si basa sul seguente lemma:
\begin{mylemma}
  Sia $G$ un gruppo, e sia $\rho$ una rappresentazione, e $\Reg$ la sua rappresentazione regolare. Allora, fissato $v\in V_\rho$, esiste un unico omomorfismo di rappresentazioni $\phi: \Reg \rightarrow \rho$ tale che $\phi(e_1)=v$.
\end{mylemma}
\begin{proof}
  Se $\phi$ è un omomorfismo di rappresentazione, allora deve valere 
  \[
  \phi(e_g)=\phi \circ \Reg_g (e_1) = \rho_g \circ \phi (e_1) = \rho_g(v)
  \]
  e quindi se esiste è unica. \`E facile verificare che questo omomorfismo rispetta le ipotesi.
\end{proof}

La dimostazione diventa ora immediata:
\begin{proof}
  Per quanto visto nel lemma, vale $\dim \Hom(R,\rho_i) = \dim V_{\rho_i}$. Quindi
  \[
		n_i = \herm{\chi_R}{\chi_i}= \dim \Hom(R,\rho_i) = \dim V_{\rho_i}
  \]
  e ottengo la tesi.
\end{proof}

Questo teorema ha importantissime conseguenze, la cui più evidente è la seguente:
\begin{myprop}
  Con le notazioni precedenti vale $\sum_i n_i^2 = \abs G$ e, se $g\ne e$, $\sum_i n_i \chi_i(g)=0$
\end{myprop}
\begin{proof}
  Per definizione vale $\sum n_i \chi_i(g) = \chi_R(g)$. Prendendo $g=e$ ottengo la prima proposizione, per $g\ne e$ ottengo invece la seconda.
\end{proof}

\begin{mytheorem}
  I caratteri delle rappresentazioni irriducibili di $G$ formano una base ortonormale delle funzioni classe $\class G$.
\end{mytheorem}

\begin{proof}
  Sappiamo che i caratteri sono ortonormali, ci manca da dimostrare che generano $\class G$. Bisogna far vedere che se $f$ è ortogonale a tutti i caratteri $\chi_i$ allora è nullo.
  
  Ometterò la dimostrazione di questo fatto anche perché verrà dimostato nella sezione $\ref{sec:MatEl}$ quando introdurremo i coefficienti matriciali.
\end{proof}


\begin{mytheorem}
  Le rappresentazioni irriducibili di $G$ sono tante quante le classi di coniugio.
\end{mytheorem}

\begin{proof}
  Visto che sulle funzioni di classe ho la base canonica costruita sulle classi di coniugio, ma ho anche la base formata dalle rappresentazioni irriducibili, come basi di uno spazio vettoriale devono avere la stessa cardinalità.
\end{proof}







 \section{Tabella dei caratteri}

Dato un gruppo $G$, possiamo costruire la sua \myname{tabella dei caratteri} fatta come segue:
\begin{itemize}
  \item Su ogni colonna mettiamo una classe di coniugio del gruppo;
  \item su ogni riga mettiamo una rappresentazione irriducibile del gruppo;
  \item all'incrocio tra la rappresentazione $\rho$ e la classe di coniugio $C$ inseriamo il valore di $\chi_\rho(g)$ con $g\in C$ (sto usando il fatto che il carattere è invariante di coniugio).
\end{itemize}

Per quanto abbiamo visto la tabella è \emph{quadrata} (ho tante classi di coniugio quante rappresentazioni irriducibili).

Per le relazioni di ortogonalità dei caratteri, sappiamo che le righe sono tutte \emph{ortonormali} (dove bisogna sempre ricordarsi di fare la media pesata sulla cardinalità della classe di coniugio quando si effettua il prodotto hermitiano).

Inoltre vale anche la seguente:
\begin{myprop}[Ortogonalità delle colonne]
  Se $\chi_i$ sono le rappresentazioni irriducibili di $G$, e $g,h\in G$ non coniugati, e $c(g)$ la cardinalità della classe di coniugio di $g$, allora 
  \begin{align*}
    \sum_i \abs{\chi_i(g)}^2 = \frac {\abs G}{c(g)}\\
    \sum_i \chi_i(g)\conj{\chi_i(h)} = 0
  \end{align*}
\end{myprop}

\begin{myexample} [Tabella dei caratteri di $Q_8$]
  \[
  \begin{array}{|c|ccccc|}
  \hline
    Q_8    & \{1\} & \{-1\} & \{\pm i\} & \{\pm j\} & \{\pm k\} \\ \hline
    Id     &   1   &    1   &     1     &     1     &     1     \\ 
    \rho_i &   1   &    1   &     1     &    -1     &    -1     \\
    \rho_j &   1   &    1   &    -1     &     1     &    -1     \\
    \rho_k &   1   &    1   &    -1     &    -1     &     1     \\
    \rho_2 &   2   &   -2   &     0     &     0     &     0     \\ \hline
  \end{array}
  \]
  
  
\end{myexample}


\iffalse 
\section{Forma hermitiana, ortogonalità, rappresentazioni irriducibili}
Sia $h_0$ un prodotto interno (forma hermitiana definita positiva) in $V\times V$, e definiamo 
\[
  h(v,w)=\frac1{\abs G}\sum_{g\in G}h_0\left(\rho_g(v),\rho_g(w)\right) 
\]
In questo modo in $\rho_g$ sono tutte applicazioni lineari unitarie.

\begin{myexample}
  Data una rappresentazione $\rho$ su $V$, indichiamo con $V^G$ il sottospazio dei punti lasciati fissi da tutti i $\rho_g$. In particolare $V^G$ è $G$-invariante e quindi ho ottenuto una sottorappresentazione (banale). 
\end{myexample}
\begin{myexample}
  Sia $\lambda$ un omomorfismo da $G$ in $C^*$. Possiamo definire, generalizzando la nozione di autospazio vista ad Algebra lineare, l'autospazio relativo a $\lambda$ come il sottospazio di $V_\rho$ costituito dai vettori $v$ che soddisfano 
  \[
  \rho_g(v)=\lambda(g)\cdot v
  \]

  Notiamo che se $\lambda \equiv 1$, allora $V_\lambda=V^G$. Inoltre, se pensiamo a $\lambda$ come una rappresentazione di grado 1, allora 
  \[	
  \restr\rho{V_\lambda}= \overbrace{\lambda + \lambda + \lambda + \dots + \lambda}^{\mbox{\tiny come somme di rappresentazioni!}}=\dim V_\lambda \cdot \lambda
  \]     
\end{myexample}
\fi

 \section{Potenza simmetrica ed esterna}

\begin{mydef}[Quadrato simmetrico e alternante]
  Sia $\rho$ una rappresentazione su $V$. Prendiamo lo spazio $V\otimes V$, e consideriamo l'automorfismo $\theta$ tale che 
  \[
  \theta(e_i \tensor e_j) = e_j \tensor e_i
  \]
  
  $V\otimes V$ si decompone quindi nella somma del quadrato simmetrico $S^2V$ dei vettori fissati da $\theta$, e nel quadrato alternante $\Lambda^2V$ dei vettori tali che $\theta(z) = -z$. Poichè essi sono $G$-invarianti (tramite la rappresentazione $\sigma^2$) allora essi definiscono due sottorappresentazioni.
\end{mydef}

\begin{myexample}
  Sia $V=\mathbb C^n$. Possiamo pensare $V\otimes V$ come $\mathcal M_n(\mathbb C)$, dove $v \tensor w = v\trasp w$. In questo caso, i quadrati simmetrici e alternanti coincidono con le matrici simmetriche e antisimmetriche rispettivamente.
\end{myexample}

\begin{mydef}[Potenza esterna]
  Si dice potenza esterna $n$-esima $\Lambda^nV$  di uno spazio vettoriale $V$ lo spazio $V^{\tensor n}$ quozientato per il sottospazio generato dai $v_1 \tensor v_2 \tensor \dots \tensor v_n$ con $v_i=v_j$ per qualche $i\ne j$.  
\end{mydef}

Detto in maniera più comprensibile, è l'insieme generato dagli elementi scritti come $v_1 \wedge v_2 \wedge \dots \wedge v_n$ dove scambiando una coppia di componenti l'intero prodotto cambia segno.

\begin{Achtung}
  Il prodotto $v_1 \wedge v_2 \wedge v_3$ non va pensato in maniera ``associativa'' (come $(v_1 \wedge v_2) \wedge v_3$), perché non ha senso!
\end{Achtung}

Data una base $v_1,v_2,\dots,v_n$ di $V$, possiamo prender come base di $\Lambda^mV$ gli elementi della forma $v_{i_1},v_{i_2},\dots,v_{i_m}$ con $i_1<i_2<\dots<i_m$. \`E quindi evidente che $\dim \Lambda^mV=\binom nm$.    
  
\begin{mydef}[Potenza simmetrica]
  Si dice potenza simmetrica $n$-esima $S^nV$  di uno spazio vettoriale $V$ lo spazio $V^{\tensor n}$ quozientato per il sottospazio generato dai $v_1 \tensor v_2 \tensor \dots \tensor v_n - v_{\sigma(1)} \tensor v_{\sigma(2)} \tensor \dots \tensor v_{\sigma(n)}$, dove $\sigma$ è una qualsiasi permutazione degli indici $\{1,2,\dots,n\}$.  
\end{mydef}

Anche qui, possiamo pensare la potenza simmetrica come il prodotto tensore in cui scambiando le componenti il prodotto rimane invariato.

Analogamente a prima, una base per $S^mV$, con $\dim V=n$, sono i vettori della forma  $v_{i_1},v_{i_2},\dots,v_{i_m}$ con $i_1\le i_2\le \dots \le i_m$. Il conto combinatorico ci dice che $\dim S^mV = \binom{n+m-1}m$.


Ora siamo pronti e mettere una rappresentazione su potenza esterna e simmetrica.



\begin{mydef}
  Si dice potenza esterna di una rappresentazione $\rho$ la rappresentazione su $\Lambda^m V_\rho$ dove
  \[
  \Lambda^m\rho_g: v_1 \wedge v_2 \wedge \dots \wedge v_n \mapsto \rho_g(v_1) \wedge \rho_g(v_2) \wedge \dots \wedge \rho_g(v_n)
  \]
  
  In modo completamente analogo si definisce la potenza simmetrica.

\end{mydef}

Precedentemente abbiamo definito potenza esterna e simmetrica in modo astratto, ma si possono anche vedere come sottospazi della potenza tensoriale, e le relative rappresentazioni come sue sottorappresentazioni.

\begin{myprop}
  Sia $\rho$ una rappresentazione su $V_\rho$, e sia $\rho^m$ la rappresentazione su $V_\rho^{\tensor m}$. Allora valgono le seguenti: 
  \begin{eqnarray*}
  &S^m\rho \isom \Span{\left\{\sum_\sigma v_{\sigma(1)}\tensor v_{\sigma(2)} \tensor \dots \tensor v_{\sigma(m)}:v_i \in V\right\}}\\
  &\Lambda^m\rho \isom \Span{\left\{\sum_\sigma \sgn(\sigma) v_{\sigma(1)}\tensor v_{\sigma(2)} \tensor \dots \tensor v_{\sigma(m)}:v_i \in V\right\}} 
  \end{eqnarray*}

\end{myprop}

\begin{proof}
  Basta far vedere che, per $S^m\rho$, l'applicazione
  \[
  v_1 \tensor \dots \tensor v_m \mapsto \frac1{m!} \sum_\sigma v_{\sigma(1)}\tensor v_{\sigma(2)} \tensor \dots \tensor v_{\sigma(m)}
  \]
  e, per $\Lambda^m\rho$, l'applicazione
  \[
  v_1 \tensor \dots \tensor v_m \mapsto \frac1{m!} \sum_\sigma \sgn(\sigma) v_{\sigma(1)}\tensor v_{\sigma(2)} \tensor \dots \tensor v_{\sigma(m)}
  \] sono ben definite, sono suriettive e hanno il nucleo giusto, e poi si può concludere per il teorema di omomorfismo. 
\end{proof}

\begin{myprop}
 Sia $\rho$ una rappresentazione su $V$. Allora valgono le seguenti:
 \begin{itemize}
  \item $\chi_{\Lambda^2 \rho} = \frac12 (\chi_\rho(g)^2 - \chi_\rho(g^2)) $
  \item $\chi_{S^2 \rho} = \frac12 (\chi_\rho(g)^2 + \chi_\rho(g^2)) $
 \end{itemize}
 
 \begin{proof}
  Fissiamo un $g \in G$, e sia $v_i$ una base di $V$ di autovettori di $\rho_g$, ossia tali che $\rho_g(v_i)=\lambda_iv_i$. Sappiamo che una base di $\Lambda^2 V$ è fatta dai $v_i \wedge v_j$ per $i<j$. D'altra parte $\Lambda^2\rho_g (v_i\wedge v_j) = \lambda_i\lambda_j (v_i \wedge v_j)$, da cui segue
  \[
   \Tr \Lambda^2\rho_g = \sum_{i<j}\lambda_i\lambda_j = \frac12\left(\left(\sum_i \lambda_i\right)^2 - \sum_i \lambda_i^2\right) = \frac12 (\chi_\rho(g)^2 - \chi_\rho(g^2))
  \]
  Per $S^2\rho$ la dimostrazione è analoga, oppure si può sfruttare anche che $\chi_{\Lambda^2\rho}+\chi_{S^2\rho}=\chi_{\rho^2}$ perchè $\Lambda^2\rho$ e $S^2\rho$ decompongono il prodotto tensore.

 \end{proof}


\end{myprop}

 \section{Rappresentazioni di gruppi particolari}
	Per costruire la tabella dei caratteri, dobbiamo costruire delle rappresentazioni da cui estrarre le irriducibili. Ci sono sostanzialmente due modi: il primo è inferendo sulla struttura del gruppo, la seconda è costruendo rappresentazioni su spazi vettoriali diversi (come le potenze esterne). In questa sezione ci occuperemo del primo metodo.
	
	\subsection{Gruppi abeliani}
		Le rappresentazioni dei gruppi abeliani sono particolarmente semplici. Vale infatti la seguente proposizione:
		\begin{myprop}
			Sia $G$ un gruppo, e siano $\rho_i$ le sue rappresentazioni irriducibili. Allora G è abeliano se e solo se tutte le $\rho_i$ hanno grado 1.
		\end{myprop}
		
		\begin{proof}
			Supponiamo che $G$ sia abeliano. Allora ho $\abs G$ classi di coniugio e pertanto altrettante rappresentazioni irriducibili. Dato che $\sum_i n_i^2 = \abs G$ segue $n_i = 1$.
			
			Viceversa, supponiamo che $G$ sia un gruppo le cui rappresentazioni irriducibili hanno grado 1. Consideriamo una rappresentazione $\rho$ fedele di $G$ (ad esempio quella regolare), quindi possiamo decomporla in irriducibili di grado 1. Questo significa che esiste una base in cui tutte le $\rho_g$ sono diagonali, e pertanto commutano. Dunque $\rho_g\rho_h = \rho_h\rho_g \Rightarrow \rho_{gh} = \rho_{hg}$ e per l'iniettività di $\rho$ ho finito.
		\end{proof}

	\subsection{Sottogruppi normali}
		Se il gruppo ha sottogruppi normali, possiamo ottenere gratuitamente delle rappresentazioni del gruppo conoscendo quelle di un gruppo più piccolo.
		
		\begin{myprop}
			Sia $N$ sottogruppo normale di $G$, e sia $\sigma$ una rappresentazione del quoziente $G/N$. Allora $\sigma$ si può estendere in modo naturale a rappresentazione $\rho$ di $G$, e in particolare le rappresentazioni del quoziente sono in relazione biunivoca con quelle di $G$ il cui $\Ker$ contiene $N$. Inoltre le irriducibili in $G/N$ sono irriducibili anche in $G$.
		\end{myprop}
		\begin{proof}
			Basta mandare mappare l'applicazione $\sigma_{gN}$ in $\rho_g$ assicurandosi che sia una buona definizione. E' simile alla corrispondenza biunivoca tra sottogruppi.
			Per quanto riguarda l'irriducibilità, basta vedere che, se $R\subset G$ è un insieme di rappresentanti delle classi laterali di $N$, allora 
			\[
				\frac{\abs{N}}{\abs{G}} \sum_{g\in R}\Tr \sigma_{gN}\conj{\Tr \sigma_{gN}} = \frac1{\abs G} \sum_{g\in R} \abs N \Tr \sigma_{gN}\conj{\Tr \sigma_{gN}} = \frac1{\abs G} \sum_{g\in G} \conj{\Tr \rho_{g}}
			\]
			quindi i due caratteri hanno la stessa norma.
		\end{proof}
		
		\begin{myexample} [Tabella dei caratteri di $Q_8$]
		Voglio costrure la tabella dei caratteri di $Q_8$. Ho 5 classi di coniugio, quindi 5 rappresentazioni. Una è l'identità. Per trovarne altre 3, posso pensare che $\{1,-1\}$ è normale, e quozientando ottengo $\Cyc_2 \times \Cyc_2$, che ha 3 rappresentazioni irriducibili non banali (scelgo quale dei 3 sottogruppi mandare nell'identità, e il resto andrà in $-1$). Sollevandole in $Q_8$, ottengo $\rho_i,\rho_j,\rho_k$. L'ultima rappresentazione, di grado 2, la ottengo per ortogonalità.
		\[
			\begin{array}{|c|ccccc|}
			\hline
			Q_8    & \{1\} & \{-1\} & \{\pm i\} & \{\pm j\} & \{\pm k\} \\ \hline
			Id     &   1   &    1   &     1     &     1     &     1     \\ 
			\rho_i &   1   &    1   &     1     &    -1     &    -1     \\
			\rho_j &   1   &    1   &    -1     &     1     &    -1     \\
			\rho_k &   1   &    1   &    -1     &    -1     &     1     \\
			\rho_2 &   2   &   -2   &     0     &     0     &     0     \\ \hline
			\end{array}
		\]
		
		
		\end{myexample}


	
	\subsection{Prodotto diretto}
	
	Vogliamo ora indagare come sono fatte le rappresentazioni di un prodotto diretto di gruppi, a partire dalle rappresentazioni sui singoli gruppi.
	
	Date due rappresentazioni $\rho$ di $G_1$, $\sigma$ di $G_2$, possiamo generalizzare il prodotto di rappresentazioni.
	\begin{mydef}\label{def:TensProdRepr}
		Si definisce il \myname{prodotto tensoriale} di rappresentazioni la rappresentazione di $G_1\times G_2$ data da
		\[
		(\rho \tensor \sigma)(g_1,g_2) = \rho_{g_1} \tensor \rho_{g_2} 
		\]
	\end{mydef}

	\begin{myprop}
		Il carattere del prodotto tensoriale vale
		\[
		\chi_{\rho\tensor\sigma}(g_1,g_2) = \chi_\rho(g_1)\chi_\sigma(g_2)
		\] 
	\end{myprop}

	\begin{proof}
		Per le proprietà del prodotto tensore di spazi vettoriali, vale $\Tr(\rho_{g_1}\tensor \sigma_{g_2})= \Tr{\rho_{g_1}}\Tr{\sigma_{g_2}}$
	\end{proof}
	
	Abbiamo quindi ottenuto una rappresentazione di $G_1\times G_2$, ci chiediamo quando è irriducibile. 
	\begin{myprop}
		Le rappresentazioni $\sigma$ e $\rho$ sono irriducibili se e solo se $\rho \tensor \sigma$ è irriducibile.
		
		Inoltre tutte le rappresentazioni irriducibili di $G_1\times G_2$ sono della forma $\rho \tensor \sigma$, con $\rho,\sigma$ irriducibili.
	\end{myprop}
	\begin{proof}
		Facciamo il conto con il carattere di $\rho \tensor \sigma$.
		\begin{align*}
		\herm{\chi_{\rho\tensor\sigma}}{\chi_{\rho\tensor\sigma}} &= \frac1{\abs {G_1}\abs{G_2}}\sum_{g_1 \in G_1}\sum_{g_2\in G_2} \chi_\rho(g_1)\chi_\sigma(g_2)\conj{\chi_\rho(g_1)\chi_\sigma(g_2)} = \\
		&= \frac1{\abs {G_1}}\sum_{g_1 \in G_1} \chi_\rho(g_1)\conj{\chi_\rho(g_1)}\cdot\frac1{\abs {G_2}}\sum_{g_2\in G_2}\chi_\sigma(g_2)\conj{\chi_\sigma(g_2)}\\
		&= \herm{\chi_\rho}{\chi_\rho}\herm{\chi_\sigma}{\chi_\sigma}
		\end{align*}
		e ora si ha la tesi per il criterio di irriducibilità.
		
		Per il secondo punto invece, mostriamo che un qualsiasi carattere $f$ (funzione di classe) su $G_1\times G_2$, ortogonale a tutti i caratteri della forma $\chi_\rho(g_1)\chi_\sigma(g_2)$, è nullo. Quindi sia
		\[
		\sum_{g_1,g_2} f(g_1,g_2) \conj{\chi_\rho(g_1),\chi_\sigma(g_2)}=0
		\]
		
		Fissiamo $\sigma$, e sia $h(g_1)=\sum_{g_2} f(g_1,g_2)\conj{\chi_\sigma(g_2)}$. Allora
		\[
		\sum_{g_1}h(g_1)\conj{\chi_\rho(g_1)}=0 \qquad\qquad \forall \rho
		\]
		Poichè deve valere per ogni rappresentazione $\rho$, e $h$ è una funzione classe, allora deve essere $h\equiv 0$. Analogamente si conclude che $f \equiv 0$.
	\end{proof}
	
	\subsection{Rappresentazione indotta}
		Sia $H<G$ un sottogruppo. Supponiamo di avere una $H$-rappresentazione $\sigma$, su uno spazio vettoriale $W$. Noi vorremmo costruire una $G$-rappresentazione. Iniziamo con una definizione.
		
		\begin{mydef}
		 Supponiamo di avere una $G$-rappresentazione $\rho$ su $V$, e sia $W$ un suo sottospazio $H$-invariante (ossia su $W$ abbiamo una $H$-rappresentazione $\sigma$). Chiamiamo $G/H$ l'insieme delle classi laterali sinistre
		 
		 La rappresentazione $\rho$ si dice indotta da $\sigma$ se 
		 \[
		  V = \bigoplus_{\bar g \in G/H} \rho_g'(W) 
		 \]
		 dove si intende che $g'$ è un rappresentante di $\bar g$.
		\end{mydef}
		
		L'idea è, dato uno spazio $W$ con una $H$-rappresentazione sopra, di costruire $V$ come somma diretta di tante copie di $W$ indicizzate dai $\bar g$, che indicheremo come $W_{\bar g}$, e dato un vettore $w\in W$, indicheremo con $w_{\bar g}$ la sua copia in $W_{\bar g}$. 
		
		Vediamo ora come deve agire $G$. Consideriamo un elemento $g\in G$, che possiamo scrivere come $g'h$ con $h\in H$, e definiamo
		\[
			\rho_g(w_{\bar s}) = \rho_{g'h}(w_{\bar s})= \sigma_h(w_{\bar{g's}})
		\]
		
		Detto a parole, io mando $W_{\bar s}$ nella sua copia $W_{\bar {g's}}$, e poi all'interno faccio agire $h$ tramite $\sigma$. 
		
		Rimarrebbe da verificare che questa è una buona definizione, e che due rappresentazioni indotte dalla stessa rappresentazione sono isomorfe, e questo viene lasciato come esercizio al volenteroso lettore.
		
		D'ora in poi, data una $H$-rappresentazione $\rho$, indicheremo come $\Ind_H^G(\rho)$ la sua indotta.
		
		Per le rappresentazioni indotte vale il seguente importante teorema.
		\begin{mytheorem}[di Frobenius]
			Sia $\rho$ una $H$-rappresentazione. Sia $\sigma$ una rappresentazione qualunque di $G$, e sia inoltre $\sigma_H$ la sua restrizione a $H$-rappresentazione (ossia $\bC[G]$ con l'azione di $H$). Allora ogni omomorfismo di rappresentazione $\phi$ da $\rho$ a $\sigma_H$ si può estendere in modo unico a un omomorfismo da $\tilde\phi$ da $\Ind_H^G(\rho)$ a $\bC[G]$. In particolare vale quindi
			\[
				\Hom_H(\rho, \sigma_H) \isom \Hom_G(\Ind_H^G(\rho), \sigma)
			\]
		\end{mytheorem}
		\begin{proof}
			Prendiamo $\phi \in \Hom(\rho, \restr{\bC[G]}{H})$. Allora deve valere 
			 \[
				\tilde\phi (w_{\bar g}) = \tilde\phi(g \cdot w_{\bar e}) = g \cdot \tilde\phi(w) = g\cdot \phi (w)			  
			 \]
			dove la prima uguaglianza vale per definizione, la seconda per le proprietà di omomorfismo di rappresentazioni e la terza perchè deve essere un'estensione (sto identificando ovviamente $W$ con $W_{\bar e}$. 
		\end{proof}
		
		Un utile corollario è il seguente.
		\begin{mycor}
		 Vale 
		 \[
		  \herm {\chi_\rho}{\chi_{\sigma_H}} = \herm {\chi_{\Ind_H^G(\rho)}}{\chi_\sigma}
		 \]

		\end{mycor}
		\begin{proof}
			Ovvio per il fatto che $\herm{\chi_\rho}{\chi_\sigma}=\dim \Hom (\rho,\sigma)$.
		\end{proof}
		
		Questo corollario serve per calcolare la composizione della rappresentazione indotta senza conoscerne il carattere.
		

	
	\subsection{Rappresentazioni di $S_n$}
		Indaghiamo le rappresentazioni del gruppo simmetrico $S_n$, che è complicato dato che ha pochi sottogruppi normali.
		
		\begin{myprop}
		Le uniche rappresentazioni di grado 1 di $S_n$ sono quella banale e il segno.
		\end{myprop}
		\begin{proof}
		Dato che devo immergere $S_n$ in $C^*$ che è abeliano, allora i commutatori devono stare nel $\Ker$. Dato che i commutatori sono $A_n$, le uniche possibilità sono $\sigma \cdot v = v$ o $\sigma \cdot v=\sgn(\sigma)v$.
		\end{proof}
		
		Una rappresentazione naturale per $S_n$ è quella data da $\bC^n$ dove $\sigma \cdot v$ permuta le componenti di $v$. Chiaramente il sottospazio $\Span(1,1,\dots,1)$ è $G$-invariante. Vediamo che la sottorappresentazione supplementare (che chiameremo $\std$) è irriducibile.
		
		\begin{myprop}
		La rappresentazione per permutazioni di $\bC^n$ è somma di due irriducibili.
		\end{myprop}
		\begin{proof}
		Calcoliamo la norma del carattere. Sia quindi $[\sigma]$ la matrice associata all'azione di $\sigma$ rispetto alla base canonica.
		Dobbiamo quindi calcolare
		\[
			\frac1{n!}\sum_{\sigma\in S_n} \left(\sum_i [\sigma]_i^i\right)\conj{\left(\sum_i[\sigma]_i^i\right)}
		\]
		Dato che $[\sigma]$ è una matrice di permutazione, in particolare è reale e possiamo scordarci di coniugare. Apro il prodotto e ottengo
		\[
			\frac1{n!}\sum_{\sigma\in S_n} \sum_i \sum_j [\sigma]_i^i[\sigma]_j^j
		\]
		
		Adesso posso fare \emph{double-counting} (o forse dovremmo dire \emph{triple-counting}) per scambiare i simboli di sommatoria e ottenere
		\[
			\frac1{n!}\sum_i \sum_j\sum_{\sigma\in S_n} [\sigma]_i^i[\sigma]_j^j      
		\]
		A $i,j$ fissati, $\sum_{\sigma\in S_n} [\sigma]_i^i[\sigma]_j^j$ conta le permutazioni che fissano sia $i$ che $j$.
		
		Quindi divido nei due casi $i=j$ e $i\ne j$ e ottengo
		\[
			\frac1{n!}\sum_i (n-1)! +  \frac1{n!}\sum_{i\ne j} (n-2)! = 2
		\]
		
		L'unica possibilità è dunque che la rappresentazione sia somma di due irriducibili distinte.
		\end{proof}
	






 \section{Coefficienti matriciali}\label{sec:MatEl}
	In questa sezione ridimostreremo molti fatti senza utilizzare la teoria dei caratteri, ma utilizzando i coefficienti matriciali.
	
	\begin{mydef}
		Sia $\rho$ una rappresentazione di $G$ su $V_\rho$. Sia $\mathcal B$ una base di $V_\rho$, e sia $[\rho_g]$ la matrice associata a $\rho_g$ in base $\cB$.
		
		Si dice \myname{coefficiente matriciale} di posto $i,j$ l'applicazione da $G$ in $\mathbb C$ che associa a un elemento del gruppo la componente $[\rho_g]_i^j$.
		
		Lo spazio generato da queste funzioni si dice \myname{spazio dei coefficienti matriciali} e si indica con $\mathcal M(\rho)$.
	\end{mydef}
	
	Per come sono definiti, lo spazio dei coefficienti matriciali dipende dalla base, ma vedremo che in realtà non è così.
	
	Innanzitutto $\cM(\rho) \subseteq \func G$, quindi potremmo farci agire $G$ come per la rappresentazione regolare. Quello che invece faremo sarà usare una rappresentazione di $G\times G$:
	
	\begin{mydef}
		Si dice rappresentazione regolare \emph{bilatera} di $G$ la rappresentazione di $G \times G$ su $\func G$ data da
		\[
			\BReg_{(h_1,h_2)}f = g \mapsto f(h_2\inv g h_1)
		\]
		
		Alternativamente, se pensassimo $\func G$ come $\bC[G]$, scriveremmo
		\[
			(h_1,h_2)\cdot g = h_2gh_1\inv
		\]

	\end{mydef}
	Per convincersi che bisogna definirla così basta prendere $f\in\func G$ che vale $1$ applicata a $g$ e $0$ altrove, applicare $(h_1,h_2)$, e scoprire che ho ottenuto una funzione che vale $1$ solo in $h_2gh_1\inv$.
	
	Quindi, sappiamo che $\cM(\rho)$ è un sottospazio di $\func G$: sarà anche $G\times G$-invariante (ossia una sottorappresentazione)? La risposta sarà sì, e lo dimostreremo tra poco

	Sia ora $\rho$ una rappresentazione su $V$, e consideriamo lo spazio degli endomorfismi da $V$ in $V$. Su di esso sappiamo che possiamo far agire $G$ come abbiamo fatto sugli omomorfismi, ossia moltiplicando per $g\inv$ sullo spazio di partenza e per $g$ in quello di arrivo. Tuttavia anche qui ci facciamo agire $G\times G$ e otteniamo una rappresentazione $\tau$ data da
	\[
		\tau_{(g,h)} (\phi) = \rho_g \circ \phi \circ \rho_h\inv
	\]
	Cioè moltiplico per $h\inv$ in partenza e per $g$ in arrivo. \`E un caso particolare del prodotto tensore tra rappresentazioni (definizione \ref{def:TensProdRepr}).
	
	Definiamo l'applicazione $\mu: \End \rar \func G$ che manda un endomorfismo $\xi$ nella mappa $g\mapsto \Tr(\xi\circ\rho_g)$.

	Chi è l'immagine? Scrivendo in base ottengo che sono gli elementi della forma
	\[
		\sum_{i,j} [C]_i^j[\rho_g]_j^i
	\]
	che non sono nient'altro che le combinazioni lineari dei $[\rho_g]_i^j$, cioè $\cM(\rho)$. Quindi $\mu$ è un omomorfismo suriettivo tra $\End(V)$ e $\cM(\rho)$.
	
	\begin{myprop}
		$\cM(\rho)$ è una sottorappresentazione di $\BReg$ (ossia è stabile per $G\times G$). Inoltre $\mu$ è un omomorfismo di rappresentazioni tra $\End (V)$ e $\cM(\rho)$ (sempre con l'azione bilatera).		 
	\end{myprop}

	\begin{proof}
		Prendiamo un generico elemento di $\cM(\rho)$, che sappiamo che si può scirvere come $\mu(\xi)$ per qualche $\xi\in\End(V)$. Applico la definizione e ottengo
		\begin{align*}
			(h_1,h_2)\cdot \mu (\xi) &= \Tr(\xi \circ \rho_{h_2\inv gh_1})\\
															&= \Tr(\rho_{h_1} \circ \xi \circ \rho_{h_2} \circ \rho_g)\\
															&= \Tr((h_1,h_2)\cdot \xi)\rho)
															&= \mu((h_1,h_2)\cdot \xi)
		\end{align*}
		Dove come al solito la prima moltiplicazione per $(h_1,h_2)$ è quella data dall'azione di $\BReg$, mentre quella finale è data dalla rappresentazione bilatera sugli endomorfismi. Nel conto ho anche usato che $\Tr(AB)=\Tr(BA)$.
		
		Quindi abbiamo scoperto che ogni elemento della forma $\mu(\xi)$ viene mandato in un altro elemento della medesima forma, che significa per quanto abbiamo appena visto che $\cM(\rho)$ è $G\times G$-invariante. Inoltre la formula che abbiamo ottenuto ci dice esattamente che $\mu$ è un omomorfismo di rappresentazioni tra $\End$ e $\cM(\rho)$.
	\end{proof}
	
	\begin{mycor}
		Supponiamo che $\rho$ sia irriducibile. Allora $\mu$ è un isomorfismo tra $\End(V)$ e $\cM(\rho)$, e inoltre $\cM(\rho)$ è anch'essa irriducibile. 
	\end{mycor}
	\begin{proof}
		Sappiamo che $\End (V)$ è il prodotto tensoriale di due rappresentazioni irriducibili ($\rho \tensor \rho^*$), e quindi è irriducibile. Quindi per un lemma di Schur indebolito $\mu$ deve essere iniettiva (dato che non può essere nulla perché è anche suriettiva).
		
		La seconda parte della tesi è ora ovvia.
	\end{proof}
	
	Quindi ora conosciamo un po' di rappresentazioni irriducibili di $\BReg$. Vediamo di indagare meglio come sono fatte.
	
	\begin{myprop}
		Sia $\rho$ una rappresentazione su $V$, e sia $\rho^*$ la corrispondente rappresentazione su $V^*$. Allora
		\begin{itemize}
		 \item $\Res_{G\times \{e\}}^{G\times G}\cM(\rho) \isom n\rho$
		 \item $\Res_{\{e\}\times G}^{G\times G}\cM(\rho) \isom n\rho^*$
		\end{itemize}

	 
	\end{myprop}

	\begin{proof}
		Sia $I$ la rappresentazione banale su $V$, e sia $I^*$ la sua duale. 

		Consideriamo il prodotto di rappresentazioni (non quello tensoriale, ma quello definito in \ref{def:RapprProd}) dato da $\rho I^*$. Per definizione, questa è la $G$-rappresentazione su $V\tensor V*$ dove $G$ agisce solo su $V$ (perché su $V^*$ l'azione è quella banale). Quindi se noi restringiamo l'azione bilatera di $\rho \tensor \rho^*$ a $G \times \{e\}$, otteniamo precisamente $\rho I^*$, che è ovviamente isomorfa a $n\rho$. Per l'isomorfismo tra $\rho \tensor \rho^*$ e $\cM(\rho)$ concludiamo.
		
		L'altro punto è analogo.
	\end{proof}		
	
	\begin{myprop}
	 Siano $\rho$, $\sigma$ due $G$-rappresentazioni irriducibili non isomorfe. Allora le $G\times G$-rappresentazioni $\cM(\rho)$ e $\cM(\sigma)$ sono anch'esse non isomorfe.
	\end{myprop}
	
	\begin{proof}
	 Se fossero isomorfe, allora lo stesso varrebbe per le loro restrizioni a $G \times \{e\}$. Ma abbiamo visto che queste sono $n\rho \not\isom n\sigma$, assurdo.
	\end{proof}
	
	Quindi sappiamo che gli $\cM(\rho_i)$ al variare di $\rho_i$ tra le rappresentazioni irriducibili di $G$ sono linearmente indipendenti, ci rimane da sapere che generano effettivamente $\func G$. Possiamo procedere in due modi:
	
	\begin{itemize}
	 \item Nell'ordine seguito da questo riassunto, ora il fatto è banale: sappiamo che $\cM(\rho_i)$ ha dimensione $n_i^2$ (per l'isomorfismo con $\rho_i \tensor \rho_i^*$), e sappiamo che $\sum_i n_i^2 = \abs G$, e quindi per dimensioni abbiamo concluso.
	 \item Se invece non vogliamo usare la teoria sui caratteri, si può dare una dimostrazione alternativa.
	\end{itemize}
	
	\begin{myprop}\label{prop:MatElDirSum}
		Vale
		\[
			\bigoplus_i \cM(\rho_i) = \func G
		\]
		al variare di $\rho_i$ tra le rappresentazioni irriducibili di $G$.
	\end{myprop}
	\begin{proof}
		L'inclusione $\subseteq$ è ovvia.
		
		Consideriamo la rappresentazione regolare unilatera $\Reg$ (insomma, la $G$-rappresentazione solita). Consideriamo la base canonica di $\func G$ fatta dalle funzioni $f_g$ che valgono $1$ valutate in $g$ e $0$ altrove. Com'è fatto $[\Reg_h]_i^j$? Questo vale $1$ se e solo se $g_i\cdot h = g_j$: se fissiamo $g_i=e$ allora scopriamo che è esattamente $f_h$, ma per definizione questo è un coefficiente matriciale. Quindi $f_h\in \cM(\Reg)$. Ora chiaramente $\cM(\Reg) \subseteq \bigoplus_i \cM(\rho_i)$ (sto usando $\cM(\rho)+\cM(\sigma)=\cM(\rho+\sigma)$), e poiché abbiamo fatto federe che gli elementi della base di $\func G$ stanno nella somma, allora abbiamo dimostrato anche la $\supseteq$.
	\end{proof}
	
	\subsection{Funzioni classe}
		\begin{mydef}
			Chiamiamo $\cD$ la rappresentazione regolare diagonale, ossia la restrizione della regolare bilatera agli elementi di $G\times G$ della forma $(g,g)$.
		\end{mydef}
		Per definizione quindi le funzioni classe $\class G$ sono quelle funzioni in $\func G$ che vengono lasciate fisse dall'azione della \emph{diagonale}. 
		
		Siamo ora pronti per dimostrare che le rappresentazioni irriducibili sono tante quante le classi di coniugio.
		
		\begin{myprop}
			Sia $\class G$ lo spazio vettoriale delle funzioni classe, e sia $n$ il numero di rappresentazioni irriducibili di $G$. Allora $\dim \class G = n$.
		\end{myprop}
		\begin{proof}
			
			Prendiamo ora $f\in \class G$. In particolare $f$ sta in $\func G$, e quindi per la proposizione \ref{prop:MatElDirSum} possiamo scriverla come
			\[
			f= \sum_i f_i\qquad f_i \in \cM(\rho_i)
			\]
			dove al solito le $\rho_i$ sono le rappresentazioni irriducibili di $G$. Poiché $\cM(\rho_i)$ è stabile per $\BReg$, a maggior ragione lo sarà per $\cD$, e quindi $f$ è stabile per l'azione diagonale se e solo se sono stabili le $f_i$. Se indichiamo con $\cM(\rho)^\cD$ il sottospazio di $\cM(\rho)$ stabile per $\cD$, abbiamo quindi che
			\[
				\dim \class G = \sum_i \dim \cM(\rho_i)^\cD
			\]
			
			Indaghiamo ora come è fatto $\cM(\rho_i)^\cD$. Noi abbiamo dimostrato che $\cM(\rho_i)\isom \End(V_i)$ come spazio vettoriale: se ora mettiamo l'azione diagonale, gli endomorfismi stabili per $\cD$ sono precisamente gli endomorfismi di rappresentazione! Infatti, la rappresentazione $\cD$ su $\End(V_i)$ coincide precisamente con quella definita in $\ref{def:HomRepr}$, e avevamo visto che gli omomorfismi lasciati fissi da $G$ sono quelli di rappresentazione. Quindi, per il lemma di Schur, $\cM(\rho_i)^\cD$ ha dimensione $1$ (gli endomorfismi di rappresentazione di un'irriducibile differiscono per la moltiplicazione per uno scalare).
			
			Dunque
			\[
				\dim \class G = \sum_i 1 = n
			\]

		\end{proof}
		
		Dato che chiaramente la dimensione di $\class G$ è data dal numero di classi di coniugio di $G$, deduciamo che il numero di classi di coniugio di $G$ è uguale al numero di rapresentazioni irriducibili.
		
		\begin{myobs}
			Notiamo che il carattere $\chi_i$ di $\rho_i$ appartiene evidentemente a $\cM(\rho_i)$, dato che $\chi_i = \Tr(\Id\rho_i)$ per definizione. Quindi la retta di $\cM(\rho_i)$ stabile per $\cD$ è formata dai multipli di $\chi_i$.
		\end{myobs}


		
		



	
	




 \section{Rappresentazioni su $\bR$}
	Vediamo ora come sono fatte le rappresentazioni sul campo reale $\bR$. Per la struttura di $\bR$, non è detto che le $\rho_g$ siano diagonalizzabili. Vediamo un esempio.
	
	\begin{myexample}
		Sia $\rho: \Cyc_4 \rar GL(\bR^2)$, data da 
			\[
				\rho_g = \left(
					\begin{matrix}
						0 	& 	1 	\\
						-1 	& 	0
					\end{matrix}
				\right)
			\]
		dove $g$ è un generatore di $\Cyc_4$. Questa applicazione non è diagonalizzabile. Inoltre è irriducibile, dato che se esistesse un sottospazio invariante non banale sarebbe una retta, ma non ho rette invarianti per $\rho_g$.
	\end{myexample}
	
	Questo esempio mostra che molte delle proposizioni vere in $\bC$ non valgono più quando la rappresentazione è su uno spazio vettoriale reale.
	
	Il lemma di Schur continua a valere, ma non valgono più i suoi corollari (ovvero che un isomorfismo è dato da una moltiplicazione per scalare). Infatti sempre nell'esempio di prima, 
	$
		\left(
			\begin{matrix}
				0 	& 	1 	\\
				-1 	& 	0
			\end{matrix}
		\right)
	$ è un omomorfismo da $\rho$ in se stessa.
	
	Ora cercheremo di confrontare le rappresentazioni su $\bC$ con quelle su $\bR$. 
	
	\begin{mydef}
	 Una rappresentazione complessa $\rho$ si dice reale se esiste una base di $V$ rispetto alla quale tutte le matrici $[\rho_g]$ sono reali. Equivalentemente, si dice reale se è isomorfa a una rappresentazione su uno spazio vettoriale reale.
	\end{mydef}

	È importante notare che se $\rho$ è una rappresentazione reale, anche $\chi_\rho$ è reale.
	
	In una generica rappresentazione complessa sappiamo trovare una forma hermitiana $G$-invariante. Purtroppo non sempre riusciamo a trovare anche una forma quadratica invariante non degenere.
	
	Prendiamo ad esempio una rappresentazione $\rho$ di grado 1: allora una forma quadratica sarà della forma $\scalar xy = axy$, quindi dalla condizione di invarianza si ottiene che $\rho_g z = \pm z$, che in generale è falso. Il motivo per cui non funziona la dimostrazione usata nel caso delle forme hermitiane è che quando faccio la media non sono sicuro di ottenere una forma non degenere.
	
	Se però $\rho$ è una rappresentazione reale, questa ammette una forma quadratica invariante non degenere: basta infatti definirne una su $\bR$ e poi estenderla al caso complesso. %TODO Spiegare bene questa parte
	
	Grazie a questa forma, possiamo ridimostrare il teorema di Masche nel caso reale, con la stessa dimostrazione del caso complesso.
	
	È chiaro che ammettere una forma quadratica di questo tipo (mi sono stancato di scrivere tutta la pappardella ogni volta) è un fatto importante, e perciò vogliamo cercare condizioni per cui siamo possibile trovarne una.
	
	Sia quindi $\rho$ una $G$-rappresentazione su $V$, $\scalar \cdot\cdot: V \times V \rar \bC$ una forma bilineare (non necessariamente simmetrica!) su $V$. 
	Questa può essere vista anche come una $\phi: V \rar V^*$ tale che $\phi(v) = \scalar v\cdot$. 
	
	\begin{myprop}
		La forma bilineare $\scalar \cdot\cdot$ è non degenere se e solo se $\phi$ è un isomorfismo. Inoltre $\scalar \cdot\cdot $ è invariante se e solo se $\phi$ è un omomorfismo di rappresentazioni tra $\rho$ e $\rho^*$.
	\end{myprop}
	
	Quindi $\rho$ ammette forme bilineari invarianti non degeneri se e solo se $\rho \isom \rho^*$, il che equivale a dire che il carattere $\chi_\rho$ è reale.
	
	Vale inoltre la seguente proposizione:
	
	\begin{myprop}
		Esiste una forma bilineare invariante non degenere $\scalar \cdot\cdot $ se e solo se esiste una funzione lineare non nulla $h: V \tensor V \rar \bC$ omomorfismo di rappresentazioni tra $\rho^2$ e $1$
	\end{myprop}
	
	\begin{proof}
		Sia $\scalar \cdot\cdot $ una tale forma. Allora, per definizione di prodotto tensore, esiste unica $h$ lineare che fa commutare il diagramma.
		\begin{itemize}
		 \item $h$ è non nulla: siano $v,w$ tali che $\scalar vw\neq 0$. Allora $h(v \tensor w )=\scalar vw \neq 0$.
		 \item $h$ è omomorfismo di rappresentazioni: si ha infatti 
		 \[
			h(\rho_g^2(v \tensor w))=h(\rho_g(v) \tensor \rho_g(w))=\scalar{\rho_g(v)}{\rho_g(w)}=\scalar vw=1\cdot h(v,w)
		 \]
		\end{itemize}
		
		L'altra freccia è lasciata come esercizio al (più volenteroso di me) lettore.
	\end{proof}

	
	Sia ora $\rho$ irriducibile. Grazie alla caratterizzazione con la $\phi$, per Schur vale che ogni forma bilineare $G$-invariante non nulla è non degenere.
	
	Se in più è vero che $\rho\isom\rho^*$, si ha anche 
	\[
		\dim\Hom(\rho^2,1)=\herm{\chi_{\rho^2}}{1} = \herm{\chi_\rho\chi_\rho}{1} = \herm{\chi_\rho}{\conj{\chi_\rho}} = 1
	\]
	per cui esiste un omomorfismo di rappresentazioni non nullo, dunque esiste una forma bilineare non degenere invariante. Inoltre, visto che $\rho^2 = S^2\rho \oplus \Lambda^2\rho$, si ottiene
	\[
		\Hom(\rho^2,1)=\Hom(S^2\rho,1) \oplus \Hom(\Lambda^2\rho,1)
	\]
	
	e poiché il termine a sinistra ha dimensione $1$, uno dei due pezzi a destra ha dimensione $1$ e l'altro $0$. Questo, ritornando alle forme bilineari, significa che una rappresentazione irriducibile isomorfa alla duale ammette o una forma simmetrica oppure una forma alternante.
	
	\begin{myobs}
		Per distinguere se una rappresentazione irriducibile ammetta una forma quadratica, una forma alternante o nessuna delle due, si può usare il cosiddetto \emph{indicatore di Schur}, cioè il numero
		\[
			\dim\Hom(S^2\rho,1)-\dim\Hom(\Lambda^2\rho,1) = \frac{1}{\abs G}\sum_{g\in G}\chi(g^2)
		\]
		Questo vale $1$ se e solo se $\rho$ ammette una forma simmetrica, $-1$ se e solo se ammette una forma alternante e $0$ se e solo se non ammette nessuna delle due, cioè se e solo se $\rho\not\isom\rho^*$.

	\end{myobs}
	
	
	
	
 \section{Rappresentazioni di gruppi compatti}
	La maggior parte dei teoremi che abbiamo dimostrato non vale più se non si assume la finitezza di $G$. Il problema fondamentale è che non riesco più a fare la media pesata sugli elementi di $G$, operazione che spesso ci veniva in aiuto.
	
	Tuttavia, è possibile riciclare buona parte delle dimostrazioni assumendo che $G$ sia compatto. Iniziamo con un po' di definizioni.
	
	\begin{mydef}
		Un \emph{gruppo topologico} è un gruppo $G$ su cui ho una topologia tale che le mappe:
		\begin{itemize}
		 \item $(g,h) \mapsto gh$;
		 \item $g \mapsto g\inv$
		\end{itemize}
		sono continue (dove su $G\times G$ ho la topologia prodotto).
	\end{mydef}
	
	\begin{mydef}
		Un gruppo topologico si dice \emph{compatto} se è uno spazio compatto rispetto alla sua topologia.
	\end{mydef}
	
	\begin{mydef}
		Si dice \emph{omomorfismo continuo} una funzione $\phi: G\rar H$ che è un'omomorfismo di gruppi e una mappa continua tra spazi topologici.  
	\end{mydef}

	Vorremmo ora, dato che non possiamo più fare la somma sugli elementi del gruppo, definire una misura che sia $G$-invariante. 
	\begin{mydef}
		Si dice \emph{integrale invariante} una mappa che associa a ogni funzione continua $f: G \rar \bR$ un numero reale $\int_G f(x) \de x$ tale che 
		\begin{itemize}
		 \item L'integrazione è lineare;
		 \item $f(x)\ge 0 \Rar \haar{f(x)} \ge 0$;
		 \item $f(x)\ge 0 \wedge \haar{f(x)} = 0 \Rar f(x)=0$; 
		 \item L'integrazione è $G$-invariante, ossia $\haar{f(gx)}=\haar{f(xg)}=\haar{f(x)}$.
		\end{itemize}
	\end{mydef}
	
	Spesso si rinormalizza l'integrale in modo che $\haar 1=1$. Un integrazione $G$-invariante si dice integrazione di Haar.
	
	\`E un fatto noto\footnote{Dicesi anche cannone.} che ogni gruppo topologico compatto ammette un integrazione di Haar.
	
	In realtà per i gruppi di cui ci interesseremo, l'integrale sarà definito in maniera abbastanza ovvia.
	
	\begin{myexample}
		Consideriamo il gruppo unitario $S^1$, dato dal sottogruppo moltiplicativo di $\bC^*$ dato dai complessi di norma $1$, che possiamo anche pensare come il gruppo delle rotazioni del piano.
		
		L'integrale di Haar di $f$ viene definito come $\haar{f(z)} = \frac1{2\pi}\int_0^{2\pi} f\left(e^{i\theta}\right) \de \theta$.
	\end{myexample}

	Vediamo ora come si può riadattare il teorema di Maschke per un gruppo compatto.
	
	\begin{mytheorem}
		Sia $G$ un gruppo compatto, e sia $\rho$ una sua rappresentazione di dimensione finita su uno spazio vettoriale complesso $V$. Sia inoltre $U \subseteq V$ un suo sottospazio $G$-invariante. Allora $U$ ammette un supplementare $W$ che sia $G$-invariante.
	\end{mytheorem}
	\begin{proof}
		Sia $h(u,v)$ una forma hermitiana su $V$. Allora $h_0=\int_{S^1} h(g\cdot u, g\cdot v)\de g$ è una forma $G$-invariante, e adesso possiamo concludere facilmente come nel teorema \ref{Th:SupplInv}, ottenendo così anche che le $\rho(g)$ sono unitarie.  
	\end{proof}
	
	Grazie all'integrale di Haar, possiamo mettere anche una forma hermitiana su $\func G$, come per i gruppi finiti. La forma sarà data da
	\[
		\herm {f_1}{f_2} = \int_G f_1(g)\conj{f_2(g)} \de g 
	\]
	
	Vorremmo ora, come abbiamo fatto per i gruppi finiti, trovare tutte le rappresentazioni irriducibili. Il problema è che se noi consideriamo la rappresentazione regolare $\Reg$ su $\bC[G]$, questa ha dimensione infinita, quindi bisogna essere più cauti. Urgono delle definizioni.
	
	\begin{Achtung}
		Non fidatevi troppo di quello che segue, ché non sono sicurissimo di quello che dico.
	\end{Achtung}

	\begin{mydef}
		Sia $V$ uno spazio vettoriale, e sia $(V_i)_{i\in I}$ una sua famiglia di sottospazi non necessariamente finita.
		Si dice che i $V_i$ sono in \emph{somma diretta} se ogni $v\in V$ si può scrivere in modo unico come somma \underline{finita} di $v_{i_1} + \cdots + v_{i_n}$ con gli $i_n$ distinti e con $v_{i_k} \in V_{i_k}$ non nulli.
		
		Sia ora invece $V$ uno spazio vettoriale in cui ho una nozione di convergenza (quindi almeno una topologia), e i $(V_i)_{i\in I}$ come sopra. Allora si dice che i $V_i$ sono in somma diretta \emph{topologica} se per ogni $v\in V$ esiste una successione $i_k$ di elementi distinti di $I$ tali che $v=\sum_{k=0}^{+\infty} v_{i_k}$ con $v_{i_k} \in V_{i_k}$.
	\end{mydef}






\section{Appunti sparsi}


\begin{myprop}
 $SO_3$ è isomorfo a $SU_2$ quozientato per $\{\pm I\}$.
\end{myprop}
\begin{proof}
	Sia $\bE=\left\{\left(
		\begin{matrix}
			x_1	& x_2+ix_3 \\
			x_2-ix_3 & -x_1 
		\end{matrix}
	\right): (x_1,x_2,x_3) \in \bR^3\right\}$.

	Considero l'azione di coniugio di $SU_2$ su $\bE$. 
\end{proof}

Consideriamo il sottoinsieme di $SU_2$ delle matrici diagonali.
\[
	A(z) = \left(
		\begin{matrix}
		z & 0 \\
		0 & -z
		\end{matrix}
	\right), \qquad \abs z=1
\]

Se $X$ è una matrice, ottengo
\[
	p(A(z))\cdot X = \left( 
	\begin{matrix}
		x_1 & z^2(x_2+ix_3) \\
		z^{-2}(x_2-ix_3) & -x_1
	\end{matrix}
	\right)
\]

Pensando $\bE = \bR \oplus \bC$ vengono cose belle.

\subsection{Rappresentazioni di $SU_2$}
	Sia $g \in SU_2$. $g$ agisce naturalmente su $\bC^2$.
	
	Considero la sua azione su $\bC[x,y]_n$ (polinomi di grado $n$). Sia $u \in \bC^2$, $f \in \bC[x,y]_n$, allora 
	\[
	 (gf)(u) = f(g\inv u)
	\]
	
	Chiamiamo $V_n$ la rappresentazione così definita.

	Studiamo innanzitutto $V_m$ come rappresentazione sul sottogruppo diagonale $T$.
	Ottengo che 
	\[
		A(z) \cdot (x^i y^{n-i}) = z^{m-2i}x^iy^{m-i}
	\]
	quindi i polinomi della forma $x^iy^{n-i}$ sono tutti autovettori. Quindi ho ottenuto la decomposizione in irriducibili, e inoltre sono tutti non isomorfi.
	
	\begin{mylemma}
	 Sia $W\subseteq V_n$ un sottospazio $T$-invariante. Allora è somma di rette della forma $\Span(x^iy^{n-i})$
	\end{mylemma}
	\begin{proof}
		Viene dal fatto che le sottorappresentazioni sono tutte non isomorfe.
	\end{proof}

	Dimostriamo ora la seguente proposizione.
	\begin{myprop}
	 $V_n$ è irriducibile.
	\end{myprop}
	\begin{proof}
	 Se $W$ è stabile per $SU_2$, allora è stabile per $T$, e quindi sarà somma diretta di rette della base dei monomi. Ora dal conto si vede che l'azione è transitiva sui monomi. 
	\end{proof}


	





  \section{Esercizi}
  \begin{myex}
   Consideriamo un $n$-agono regolare, con un numero complesso scritto su ogni vertice. A ogni passo sositiuisco ogni vertice con la media degli adiacenti. Come si comporta il problema asintoticamente?
   
   Consideriamo lo spazio vettoriale $\mathbb C^n$ dei numeri presenti sui vertici, e consideriamo la rappresentazione regolare di $\Cyc_n$ su $\mathbb C^n$ (che agisce sui vettori della base per rotazione). Sia inoltre $T$ l'applicazione lineare che manda ogni vertice nella media degli adiacenti.
   
   Visto che $T$ commuta con tutti i $\rho_g$, allora $T$ è anche un \myname{endomorfismo di rappresentazione}. DA FINIRE
  \end{myex}

\end{document}
