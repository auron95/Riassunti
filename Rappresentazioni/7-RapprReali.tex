\section{Rappresentazioni su $\bR$}
	Vediamo ora come sono fatte le rappresentazioni sul campo reale $\bR$. Per la struttura di $\bR$, non è detto che le $\rho_g$ siano diagonalizzabili. Vediamo un esempio.
	
	\begin{myexample}
		Sia $\rho: \Cyc_4 \rar GL(\bR^2)$, data da 
			\[
				\rho_g = \left(
					\begin{matrix}
						0 	& 	1 	\\
						-1 	& 	0
					\end{matrix}
				\right)
			\]
		dove $g$ è un generatore di $\Cyc_4$. Questa applicazione non è diagonalizzabile. Inoltre è irriducibile, dato che se esistesse un sottospazio invariante non banale sarebbe una retta, ma non ho rette invarianti per $\rho_g$.
	\end{myexample}
	
	Questo esempio mostra che molte delle proposizioni vere in $\bC$ non valgono più quando la rappresentazione è su uno spazio vettoriale reale.
	
	Il lemma di Schur continua a valere, ma non valgono più i suoi corollari (ovvero che un isomorfismo è dato da una moltiplicazione per scalare). Infatti sempre nell'esempio di prima, 
	$
		\left(
			\begin{matrix}
				0 	& 	1 	\\
				-1 	& 	0
			\end{matrix}
		\right)
	$ è un omomorfismo da $\rho$ in se stessa.
	
	Ora cercheremo di confrontare le rappresentazioni su $\bC$ con quelle su $\bR$.
	
	\begin{mydef}
	 Una rappresentazione complessa $\rho$ si dice reale se esiste una base di $V$ rispetto alla quale le matrici $[\rho_g]$ sono reali. Equivalentemente, si dice reale se è isomorfa a una rappresentazione su uno spazio vettoriale reale.
	\end{mydef}

	In una generica rappresentazione complessa sappiamo trovare una forma hermitiana $G$-invariante. Purtroppo non sempre riusciamo a trovare anche una forma quadratica invariante non degenere. Prendiamo ad esempio una rappresentazione $\rho$ di grado 1: allora una forma quadratica sarà della forma $\scalar xy = axy$. Quindi dalla condizione di invarianza si ottiene che $\rho_g z = \pm z$, che in generale è falso.
	
	Il problema che quando faccio la media non sono sicuro di ottenere una forma non degenere.
	
	Sia quindi $\rho$ una $G$-rappresentazione su $V$ e consideriamo allora una forma quadratica $\scalar \cdot\cdot: V \times V \rar \bC$.
	Possiamo anche vederla come una $\phi: V \rar V^*$ tale che $\phi(v) = \scalar v\cdot$. 
	
	\begin{myprop}
		La forma quadratica $\scalar \cdot\cdot$ è non degenere se e solo se $\phi$ è un isomorfismo. Inoltre $\scalar \cdot\cdot $ è invariante se e solo se $\phi$ è un omomorfismo di rappresentazioni tra $\rho$ e $\rho^*$.
	\end{myprop}
	
	Quindi $\rho$ ammette forme quadratiche invarianti non degeneri se e solo se $\rho \isom \rho^*$, il che equivale a dire che il carattere $\chi_\rho$ è reale.
	
	Inoltre, se $\rho$ è irriducibile, ogni forma quadratica non nulla è non degenere (per Schur).

	
	
	
	