\section{Rappresentazioni su $\bR$}
	Vediamo ora come sono fatte le rappresentazioni sul campo reale $\bR$. Per la struttura di $\bR$, non è detto che le $\rho_g$ siano diagonalizzabili. Vediamo un esempio.
	
	\begin{myexample}
		Sia $\rho: \Cyc_4 \rar GL(\bR^2)$, data da 
			\[
				\rho_g = \left(
					\begin{matrix}
						0 	& 	1 	\\
						-1 	& 	0
					\end{matrix}
				\right)
			\]
		dove $g$ è un generatore di $\Cyc_4$. Questa applicazione non è diagonalizzabile. Inoltre è irriducibile, dato che se esistesse un sottospazio invariante non banale sarebbe una retta, ma non ho rette invarianti per $\rho_g$.
	\end{myexample}
	
	Questo esempio mostra che molte delle proposizioni vere in $\bC$ non valgono più quando la rappresentazione è su uno spazio vettroriale reale.
	
	Il lemma di Schur continua a valere, ma non valgono più i suoi corollari (ovvero che un isomorfismo è dato da una moltiplicazione per scalare). Infatti sempre nell'esempio di prima, 
	$
		\left(
			\begin{matrix}
				0 	& 	1 	\\
				-1 	& 	0
			\end{matrix}
		\right)
	$ è un omomorfismo da $\rho$ in se stessa.
	
	Ora cercheremo di confrontare le rappresentazioni su $\bC$ con quelle su $\bR$.
	
	\begin{mydef}
	 Una rappresentazione complessa $\rho$ si dice reale se esiste una base di $V$ rispetto alla quale le matrici $[\rho_g]$ sono reali. Equivalentemente, si dice reale se è isomorfa a una rappresentazione su uno spazio vettoriale reale.
	\end{mydef}

	