\section{Rappresentazioni su $\bR$}
	Vediamo ora come sono fatte le rappresentazioni sul campo reale $\bR$. Per la struttura di $\bR$, non è detto che le $\rho_g$ siano diagonalizzabili. Vediamo un esempio.
	
	\begin{myexample}
		Sia $\rho: \Cyc_4 \rar GL(\bR^2)$, data da 
			\[
				\rho_g = \left(
					\begin{matrix}
						0 	& 	1 	\\
						-1 	& 	0
					\end{matrix}
				\right)
			\]
		dove $g$ è un generatore di $\Cyc_4$. Questa applicazione non è diagonalizzabile. Inoltre è irriducibile, dato che se esistesse un sottospazio invariante non banale sarebbe una retta, ma non ho rette invarianti per $\rho_g$.
	\end{myexample}
	
	Questo esempio mostra che molte delle proposizioni vere in $\bC$ non valgono più quando la rappresentazione è su uno spazio vettoriale reale.
	
	Il lemma di Schur continua a valere, ma non valgono più i suoi corollari (ovvero che un isomorfismo è dato da una moltiplicazione per scalare). Infatti sempre nell'esempio di prima, 
	$
		\left(
			\begin{matrix}
				0 	& 	1 	\\
				-1 	& 	0
			\end{matrix}
		\right)
	$ è un omomorfismo da $\rho$ in se stessa.
	
	Ora cercheremo di confrontare le rappresentazioni su $\bC$ con quelle su $\bR$. 
	
	\begin{mydef}
	 Una rappresentazione complessa $\rho$ si dice reale se esiste una base di $V$ rispetto alla quale tutte le matrici $[\rho_g]$ sono reali. Equivalentemente, si dice reale se è isomorfa a una rappresentazione su uno spazio vettoriale reale.
	\end{mydef}

	È importante notare che se $\rho$ è una rappresentazione reale, anche $\chi_\rho$ è reale.
	
	In una generica rappresentazione complessa sappiamo trovare una forma hermitiana $G$-invariante. Purtroppo non sempre riusciamo a trovare anche una forma quadratica invariante non degenere.
	
	Prendiamo ad esempio una rappresentazione $\rho$ di grado 1: allora una forma quadratica sarà della forma $\scalar xy = axy$, quindi dalla condizione di invarianza si ottiene che $\rho_g z = \pm z$, che in generale è falso. Il motivo per cui non funziona la dimostrazione usata nel caso delle forme hermitiane è che quando faccio la media non sono sicuro di ottenere una forma non degenere.
	
	Se però $\rho$ è una rappresentazione reale, questa ammette una forma quadratica invariante non degenere: basta infatti definirne una su $\bR$ e poi estenderla al caso complesso. %TODO Spiegare bene questa parte
	
	Grazie a questa forma, possiamo ridimostrare il teorema di Masche nel caso reale, con la stessa dimostrazione del caso complesso.
	
	È chiaro che ammettere una forma quadratica di questo tipo (mi sono stancato di scrivere tutta la pappardella ogni volta) è un fatto importante, e perciò vogliamo cercare condizioni per cui siamo possibile trovarne una.
	
	Sia quindi $\rho$ una $G$-rappresentazione su $V$, $\scalar \cdot\cdot: V \times V \rar \bC$ una forma bilineare (non necessariamente simmetrica!) su $V$. 
	Questa può essere vista anche come una $\phi: V \rar V^*$ tale che $\phi(v) = \scalar v\cdot$. 
	
	\begin{myprop}
		La forma bilineare $\scalar \cdot\cdot$ è non degenere se e solo se $\phi$ è un isomorfismo. Inoltre $\scalar \cdot\cdot $ è invariante se e solo se $\phi$ è un omomorfismo di rappresentazioni tra $\rho$ e $\rho^*$.
	\end{myprop}
	
	Quindi $\rho$ ammette forme bilineari invarianti non degeneri se e solo se $\rho \isom \rho^*$, il che equivale a dire che il carattere $\chi_\rho$ è reale.
	
	Vale inoltre la seguente proposizione:
	
	\begin{myprop}
		Esiste una forma bilineare invariante non degenere $\scalar \cdot\cdot $ se e solo se esiste una funzione lineare non nulla $h: V \tensor V \rar \bC$ omomorfismo di rappresentazioni tra $\rho^2$ e $1$
	\end{myprop}
	
	\begin{proof}
		Sia $\scalar \cdot\cdot $ una tale forma. Allora, per definizione di prodotto tensore, esiste unica $h$ lineare che fa commutare il diagramma.
		\begin{itemize}
		 \item $h$ è non nulla: siano $v,w$ tali che $\scalar vw\neq 0$. Allora $h(v \tensor w )=\scalar vw \neq 0$.
		 \item $h$ è omomorfismo di rappresentazioni: si ha infatti 
		 \[
			h(\rho_g^2(v \tensor w))=h(\rho_g(v) \tensor \rho_g(w))=\scalar{\rho_g(v)}{\rho_g(w)}=\scalar vw=1\cdot h(v,w)
		 \]
		\end{itemize}
		
		L'altra freccia è lasciata come esercizio al (più volenteroso di me) lettore.
	\end{proof}

	
	Sia ora $\rho$ irriducibile. Grazie alla caratterizzazione con la $\phi$, per Schur vale che ogni forma bilineare $G$-invariante non nulla è non degenere.
	
	Se in più è vero che $\rho\isom\rho^*$, si ha anche 
	\[
		\dim\Hom(\rho^2,1)=\herm{\chi_{\rho^2}}{1} = \herm{\chi_\rho\chi_\rho}{1} = \herm{\chi_\rho}{\conj{\chi_\rho}} = 1
	\]
	per cui esiste un omomorfismo di rappresentazioni non nullo, dunque esiste una forma bilineare non degenere invariante. Inoltre, visto che $\rho^2 = S^2\rho \oplus \Lambda^2\rho$, si ottiene
	\[
		\Hom(\rho^2,1)=\Hom(S^2\rho,1) \oplus \Hom(\Lambda^2\rho,1)
	\]
	
	e poiché il termine a sinistra ha dimensione $1$, uno dei due pezzi a destra ha dimensione $1$ e l'altro $0$. Questo, ritornando alle forme bilineari, significa che una rappresentazione irriducibile isomorfa alla duale ammette o una forma simmetrica oppure una forma alternante.
	
	\begin{myobs}
		Per distinguere se una rappresentazione irriducibile ammetta una forma quadratica, una forma alternante o nessuna delle due, si può usare il cosiddetto \myname{indicatore di Schur}, cioè il numero
		\[
			\dim\Hom(S^2\rho,1)-\dim\Hom(\Lambda^2\rho,1) = \frac{1}{\abs G}\sum_{g\in G}\chi(g^2)
		\]
		Questo vale $1$ se e solo se $\rho$ ammette una forma simmetrica, $-1$ se e solo se ammette una forma alternante e $0$ se e solo se non ammette nessuna delle due, cioè se e solo se $\rho\not\isom\rho^*$.

	\end{myobs}
	
	
	
\subsection{Realificazione e complessificazione}
	
	\begin{mydef}
		Sia $W$ uno spazio vettoriale complesso. Si dice \myname{realificazione} di $W$, e si indica con $W_\bR$, $W$ considerato come spazio vettoriale reale. In altre parole, se $\{w_1,\ldots,w_n\}$ è una base complessa di $W$, $W_\bR$ è lo spazio vettoriale reale con base $\{w_1,i w_1,\ldots,w_n,i w_n\}$.
	\end{mydef}
	
	Ovviamente si ha $\dim(W_\bR)=2\dim(W)$. Inoltre $GL(W)\embed GL(W_\bR)$
	
	\begin{mydef}
		Sia $f:W \rar W$ una funzione $\bC$-lineare. Si dice \myname{realificazione} di $f$, e si indica con $f_\bR:W_\bR \rar W_\bR$, la stessa funzione, ma considerata $\bR$-lineare.
	\end{mydef}
	
	\begin{myexample}
		Prendiamo ad esempio la moltiplicazione per un numero complesso $\alpha=a+ib$ in $\bC$. Allora $\alpha_\bR$ si identifica con la matrice $2\times2$
		$
			\left(
				\begin{matrix}
					a  & -b  \\
					b  &  a
				\end{matrix}
			\right)
		$.
	\end{myexample}
	
	Facendo i conti si può osservare che $\Tr(f_\bR)=2\Re(\Tr(f))=\Tr(f)+\conj{\Tr(f)}$
	
	\begin{mydef}
		Sia $\sigma$ una rappresentazione complessa di $G$. Si dice \myname{realificazione} di $\sigma$ la rappresentazione reale $\sigma_\bR$ definita da $\sigma_\bR(g)=(\sigma(g))_\bR$
	\end{mydef}
	
	Si ha $\chi_{\sigma_\bR}=\chi_\sigma + \conj{\chi_\sigma} = \chi_{\sigma+\sigma^*}$
	
	\begin{mydef}
		Sia $V$ uno spazio vettoriale reale. Si dice \myname{complessificazione} di $V$ lo spazio vettoriale complesso $V_\bC=V\tensor_\bR \bC$ con la moltiplicazione per scalari $\alpha \cdot (v\tensor a)= v\tensor (\alpha a)$
	\end{mydef}
	
	Se per ogni $v \in V$ identifico $v\tensor 1$ con $v$, è facile notare che una base di $V$ come spazio vettoriale reale è a tutti gli effetti anche una base $V_\bC$.
	
	\begin{mydef}
		Sia $f:V\rar V$ una funzione $\bR$-lineare. Si dice \myname{complessificazione} di $f$ la funzione $f_\bC:V_\bC \rar V_\bC$ definita da $f_\bC=f\tensor id_\bC$
	\end{mydef}
	
	Se $\cB$ è una base di $V$, si ha che le due matrici $[f]_\cB$ e $[f_\bC]_\cB$ coincidono, dunque in particolare $\Tr(f)=\Tr(f_\bC)$.
	
	Analogamente a quanto fatto per le rappresentazioni complesse, definiamo una rappresentazione complessa a partire da una reale:
	
	\begin{mydef}
		Sia $\rho$ una rappresentazione reale di $G$. Si dice \myname{realificazione} di $\rho$ la rappresentazione complessa $\rho_\bC$ definita da $\rho_\bC(g)=(\rho(g))_\bC$.
	\end{mydef}

	Analogamente a prima (è tutto un dejà-vu ormai) ne sappiamo calcolare il carattere: infatti si ha $\chi_{\rho_\bC}=\chi_\rho$, e in particolare è una funzione a valori reali. Inoltre si ha $\chi_{\rho_{\bC\bR}}=\chi_\rho+\chi_\rho$, $\chi_{\sigma_{\bR\bC}}=\chi_\sigma+\chi_{\sigma^*}$, da cui si ottiene $\rho_{\bC\bR}\isom \rho+\rho$, $\sigma_{\bR\bC}\isom \sigma+\sigma^*$.
	
	Il nostro obbiettivo adesso è di studiare le rappresentazioni reali attraverso le rappresentazioni complesse, di cui sappiamo tutto. Per fare ciò abbiamo bisogno di alcuni lemmi, che presentiamo qui sotto.
	
	\begin{myprop}
		Siano $V$ uno spazio vettoriale reale, $W$ uno spazio vettoriale complesso, $\rho$ una rappresentazione reale e $\sigma$ una rappresentazione complessa di $G$.
		
		Allora gli spazi $\Hom_\bR(V,W_\bR)$, $\Hom_\bC(V_\bC,W)$ sono isomorfi. Analogamente sono isomorfi gli spazi $\Hom(\rho,\sigma_\bR)$, $\Hom(\rho_\bC,\sigma)$.
		\label{prop:RealCompl}
	\end{myprop}

	\begin{proof}
		Siano $f:\Hom_\bR(V,W_\bR) \rar \Hom_\bC(V_\bC,W)$, $f(\phi)=\phi|_V$ e $g:\Hom_\bC(V_\bC,W) \rar \Hom_\bR(V,W_\bR)$, $g(\psi)=\psi_\bC$. Si mostra facendo i conti che $f$ e $g$ sono una l'inversa dell'altra e che sono omomorfismi di spazi vettoriali o di rappresentazioni (io come al solito non ne ho alcuna voglia).
	\end{proof}
	
	Da questa proposizione se ne ricava un'altra, che ha una conseguenza molto importante sul piano pratico.
	
	\begin{mylemma}
	 	Siano $\rho$ e $\rho'$ due rappresentazioni irriducibili reali tali che $\rho\not\isom\rho'$. Allora $\rho_\bC$ e $\rho'_\bC$ non hanno componenti irriducibili comuni.
	\end{mylemma}

	\begin{proof}
	 	Si ha $\Hom(\rho_\bC,\rho'_\bC)\isom\Hom(\rho_{\bC\bR},\rho')\isom\Hom(\rho+\rho,\rho')=\{0\}$ per la proposizione \ref{prop:RealCompl}.
	\end{proof}

	Conseguenza di questo lemma è che il numero di rappresentazioni reali di un gruppo $G$ è minore o uguale del numero delle sue rappresentazioni complesse. In particolare, poiché $G$ ha un numero finito di rappresentazioni complesse, ha anche un numero finito di rappresentazioni reali.
	
	\begin{mylemma}
	 	Sia $\rho$ una rappresentazione irriducibile reale, e $\sigma$ una componente irriducibile complessa di $\rho_\bC$. Allora $\rho$ è una componente irriducibile reale di $\sigma_\bR$.
	\end{mylemma}

	\begin{proof}
	 	Si ha $\Hom(\rho,\sigma_\bR)\isom\Hom(\rho_\bC,\sigma)$ per la proposizione \ref{prop:RealCompl}
	\end{proof}

	
	
	
	
	
	
	
	
	
	
	
	
	
	
	
	