	\begin{myexample}[Rappresentazioni di $\Cyc_n$]
		Ho esattamente $n$ possibilità per $\chi_\rho$. Infatti, se $g$ genera $\Cyc_n$, posso scegliere $\chi_\rho(g)=\zeta^i$ per $i=0,1\dots n-1$, dove $\zeta$ è una radice primitiva dell'unità.
		
		Posso ora scrivere
		\[
			V=\bigoplus_{\lambda=\zeta^i}V_\lambda
		\]
		dove i $V_\lambda$ sono gli autospazi dell'applicazione $\rho_(g)$ relativi all'autovalore $\lambda$.
		
		Se $\rho$ è una rappresentazione in $V$, e $\sigma$ in W, allora $\phi$ è un omomorfismo da $\rho$ in $\sigma$ se e solo se $\phi(V_\lambda)\subseteq \phi(W_\lambda)$ per ogni $\lambda$.
		
		\begin{proof}
			Perchè sia un omomorfismo di rappresentazione, chiaramente gli autospazi di $V$ devono andare negli autospazi di $W$. Inoltre, visto che posso scegliere una base di autovettori (le immagini di una rappresentazione di un gruppo finito sono diagonalizzabili), mi basta che soddisfi le proprietà su una base.
		\end{proof}


	\end{myexample}
