\section{Rappresentazioni di gruppi compatti (bozza)}

\begin{myprop}
 $SO_3$ è isomorfo a $SU_2$ quozientato per $\{\pm I\}$.
\end{myprop}
\begin{proof}
	Sia $\bE=\left\{\left(
		\begin{matrix}
			x_1	& x_2+ix_3 \\
			x_2-ix_3 & -x_1 
		\end{matrix}
	\right): (x_1,x_2,x_3) \in \bR^3\right\}$.

	Considero l'azione di coniugio di $SU_2$ su $\bE$. 
\end{proof}

Consideriamo il sottoinsieme di $SU_2$ delle matrici diagonali.
\[
	A(z) = \left(
		\begin{matrix}
		z & 0 \\
		0 & -z
		\end{matrix}
	\right), \qquad \abs z=1
\]

Se $X$ è una matrice, ottengo
\[
	p(A(z))\cdot X = \left( 
	\begin{matrix}
		x_1 & z^2(x_2+ix_3) \\
		z^{-2}(x_2-ix_3) & -x_1
	\end{matrix}
	\right)
\]

Pensando $\bE = \bR \oplus \bC$ vengono cose belle.

\subsection{Rappresentazioni di $SU_2$}
	Sia $g \in SU_2$. $g$ agisce naturalmente su $\bC^2$.
	
	Considero la sua azione su $\bC[x,y]_n$ (polinomi di grado $n$). Sia $u \in \bC^2$, $f \in \bC[x,y]_n$, allora 
	\[
	 (gf)(u) = f(g\inv u)
	\]
	
	Chiamiamo $V_n$ la rappresentazione così definita.

	Studiamo innanzitutto $V_m$ come rappresentazione sul sottogruppo diagonale $T$.
	Ottengo che 
	\[
		A(z) \cdot (x^i y^{n-i}) = z^{m-2i}x^iy^{m-i}
	\]
	quindi i polinomi della forma $x^iy^{n-i}$ sono tutti autovettori. Quindi ho ottenuto la decomposizione in irriducibili, e inoltre sono tutti non isomorfi.
	
	\begin{mylemma}
	 Sia $W\subseteq V_n$ un sottospazio $T$-invariante. Allora è somma di rette della forma $\Span(x^iy^{n-i})$
	\end{mylemma}
	\begin{proof}
		Viene dal fatto che le sottorappresentazioni sono tutte non isomorfe.
	\end{proof}

	Dimostriamo ora la seguente proposizione.
	\begin{myprop}
	 $V_n$ è irriducibile.
	\end{myprop}
	\begin{proof}
	 Se $W$ è stabile per $SU_2$, allora è stabile per $T$, e quindi sarà somma diretta di rette della base dei monomi. Ora dal conto si vede che l'azione è transitiva sui monomi. 
	\end{proof}


	



