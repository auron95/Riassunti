\section{Rappresentazioni di gruppi compatti}
	La maggior parte dei teoremi che abbiamo dimostrato non vale più se non si assume la finitezza di $G$. Il problema fondamentale è che non riesco più a fare la media pesata sugli elementi di $G$, operazione che spesso ci veniva in aiuto.
	
	Tuttavia, è possibile riciclare buona parte delle dimostrazioni assumendo che $G$ sia compatto. Iniziamo con un po' di definizioni.
	
	\begin{mydef}
		Un \emph{gruppo topologico} è un gruppo $G$ su cui ho una topologia tale che le mappe:
		\begin{itemize}
		 \item $(g,h) \mapsto gh$;
		 \item $g \mapsto g\inv$
		\end{itemize}
		sono continue (dove su $G\times G$ ho la topologia prodotto).
	\end{mydef}
	
	\begin{mydef}
		Un gruppo topologico si dice \emph{compatto} se è uno spazio compatto rispetto alla sua topologia.
	\end{mydef}
	
	\begin{mydef}
		Si dice \emph{omomorfismo continuo} una funzione $\phi: G\rar H$ che è un'omomorfismo di gruppi e una mappa continua tra spazi topologici.  
	\end{mydef}

	Vorremmo ora, dato che non possiamo più fare la somma sugli elementi del gruppo, definire una misura che sia $G$-invariante. 
	\begin{mydef}
		Si dice \emph{integrale invariante} una mappa che associa a ogni funzione continua $f: G \rar \bR$ un numero reale $\int_G f(x) \de x$ tale che 
		\begin{itemize}
		 \item L'integrazione è lineare;
		 \item $f(x)\ge 0 \Rar \haar{f(x)} \ge 0$;
		 \item $f(x)\ge 0 \wedge \haar{f(x)} = 0 \Rar f(x)=0$; 
		 \item L'integrazione è $G$-invariante, ossia $\haar{f(gx)}=\haar{f(xg)}=\haar{f(x)}$.
		\end{itemize}
	\end{mydef}
	
	Spesso si rinormalizza l'integrale in modo che $\haar 1=1$. Un integrazione $G$-invariante si dice integrazione di Haar.
	
	\`E un fatto noto\footnote{Dicesi anche cannone.} che ogni gruppo topologico compatto ammette un integrazione di Haar.
	
	In realtà per i gruppi di cui ci interesseremo, l'integrale sarà definito in maniera abbastanza ovvia.
	
	\begin{myexample}
		Consideriamo il gruppo unitario $S^1$, dato dal sottogruppo moltiplicativo di $\bC^*$ dato dai complessi di norma $1$, che possiamo anche pensare come il gruppo delle rotazioni del piano.
		
		L'integrale di Haar di $f$ viene definito come $\haar{f(z)} = \frac1{2\pi}\int_0^{2\pi} f\left(e^{i\theta}\right) \de \theta$.
	\end{myexample}

	Vediamo ora come si può riadattare il teorema di Maschke per un gruppo compatto.
	
	\begin{mytheorem}
		Sia $G$ un gruppo compatto, e sia $\rho$ una sua rappresentazione di dimensione finita su uno spazio vettoriale complesso $V$. Sia inoltre $U \subseteq V$ un suo sottospazio $G$-invariante. Allora $U$ ammette un supplementare $W$ che sia $G$-invariante.
	\end{mytheorem}
	\begin{proof}
		Sia $h(u,v)$ una forma hermitiana su $V$. Allora $h_0=\int_{S^1} h(g\cdot u, g\cdot v)\de g$ è una forma $G$-invariante, e adesso possiamo concludere facilmente come nel teorema \ref{Th:SupplInv}, ottenendo così anche che le $\rho(g)$ sono unitarie.  
	\end{proof}
	
	Grazie all'integrale di Haar, possiamo mettere anche una forma hermitiana su $\func G$, come per i gruppi finiti. La forma sarà data da
	\[
		\herm {f_1}{f_2} = \int_G f_1(g)\conj{f_2(g)} \de g 
	\]
	
	Vorremmo ora, come abbiamo fatto per i gruppi finiti, trovare tutte le rappresentazioni irriducibili. Il problema è che se noi consideriamo la rappresentazione regolare $\Reg$ su $\bC[G]$, questa ha dimensione infinita, quindi bisogna essere più cauti. Urgono delle definizioni.
	
	\begin{mydef}
		Sia $V$ uno spazio vettoriale generalmente con dimensione infinita su cui è definita una forma hermitiana $\scalar\cdot\cdot$. Consideriamo un insieme al più numerabile di vettori $v_1, v_2, \dots$ ortogonali tra loro. Questo insieme si dice \emph{completo} se le combinazioni lineari finite sono dense in $V$. In tal caso, ogni elemento $w\in V$ si può decomporre in serie di Fourier come
		\[
			w = \sum_{i=1}^{+\infty} \frac{\scalar w{v_i}}{\scalar{v_i}{v_i}}v_i
		\]
	\end{mydef}

	Ora, si potrebbe sviluppare la teoria delle rappresentazioni dei gruppi compatti con i caratteri, ma bisogna essere cauti perché questi non sono definiti per rappresentazioni di grado infinito. Come possiamo ovviare?
	
	\subsection{Coefficienti matriciali -- il ritorno}
		Prendiamo un generico spazio topologico compatto $X$ (che poi sarà il nostro gruppo), su cui abbiamo definito un integrazione. Più precisamente, noi sappiamo integrare una funzione definita su $X$ a valori in $\bR$, e possiamo definire l'integrazioni di funzioni complesse come
		\[
		 \int_X f+ig = \int_X f +i\int_X g
		\]
		dove $f,g$ sono funzioni reali. Ora possiamo definire la forma hermitiana nel modo solito:
		\[
			\herm {f_1}{f_2} = \int_X f_1 \conj{f_2}  
		\]
		
		Abbiamo definito in generale una forma hermitiana su $\func X$. Ora, se prendiamo $X$ un gruppo topologico compatto, e consideriamo l'integrazione di Haar, quello che otteniamo è che $\func G$ ha definita una forma hermitiana invariante per l'azione della rappresentazione regolare bilatera (vedi definizione \ref{def:BilReg}).
		

	



\section{Appunti sparsi}


\begin{myprop}
 $SO_3$ è isomorfo a $SU_2$ quozientato per $\{\pm I\}$.
\end{myprop}
\begin{proof}
	Sia $\bE=\left\{\left(
		\begin{matrix}
			x_1	& x_2+ix_3 \\
			x_2-ix_3 & -x_1 
		\end{matrix}
	\right): (x_1,x_2,x_3) \in \bR^3\right\}$.

	Considero l'azione di coniugio di $SU_2$ su $\bE$. 
\end{proof}

Consideriamo il sottoinsieme di $SU_2$ delle matrici diagonali.
\[
	A(z) = \left(
		\begin{matrix}
		z & 0 \\
		0 & -z
		\end{matrix}
	\right), \qquad \abs z=1
\]

Se $X$ è una matrice, ottengo
\[
	p(A(z))\cdot X = \left( 
	\begin{matrix}
		x_1 & z^2(x_2+ix_3) \\
		z^{-2}(x_2-ix_3) & -x_1
	\end{matrix}
	\right)
\]

Pensando $\bE = \bR \oplus \bC$ vengono cose belle.

\subsection{Rappresentazioni di $SU_2$}
	Sia $g \in SU_2$. $g$ agisce naturalmente su $\bC^2$.
	
	Considero la sua azione su $\bC[x,y]_n$ (polinomi di grado $n$). Sia $u \in \bC^2$, $f \in \bC[x,y]_n$, allora 
	\[
	 (gf)(u) = f(g\inv u)
	\]
	
	Chiamiamo $V_n$ la rappresentazione così definita.

	Studiamo innanzitutto $V_m$ come rappresentazione sul sottogruppo diagonale $T$.
	Ottengo che 
	\[
		A(z) \cdot (x^i y^{n-i}) = z^{m-2i}x^iy^{m-i}
	\]
	quindi i polinomi della forma $x^iy^{n-i}$ sono tutti autovettori. Quindi ho ottenuto la decomposizione in irriducibili, e inoltre sono tutti non isomorfi.
	
	\begin{mylemma}
	 Sia $W\subseteq V_n$ un sottospazio $T$-invariante. Allora è somma di rette della forma $\Span(x^iy^{n-i})$
	\end{mylemma}
	\begin{proof}
		Viene dal fatto che le sottorappresentazioni sono tutte non isomorfe.
	\end{proof}

	Dimostriamo ora la seguente proposizione.
	\begin{myprop}
	 $V_n$ è irriducibile.
	\end{myprop}
	\begin{proof}
	 Se $W$ è stabile per $SU_2$, allora è stabile per $T$, e quindi sarà somma diretta di rette della base dei monomi. Ora dal conto si vede che l'azione è transitiva sui monomi. 
	\end{proof}


	



