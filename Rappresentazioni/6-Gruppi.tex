\section{Rappresentazioni di gruppi particolari}
	Per costruire la tabella dei caratteri, dobbiamo costruire delle rappresentazioni da cui estrarre le irriducibili. Ci sono sostanzialmente due modi: il primo è inferendo sulla struttura del gruppo, la seconda è costruendo rappresentazioni su spazi vettoriali diversi (come le potenze esterne). In questa sezione ci occuperemo del primo metodo.
	
	\subsection{Gruppi abeliani}
		Le rappresentazioni dei gruppi abeliani sono particolarmente semplici. Vale infatti la seguente proposizione:
		\begin{myprop}
			Sia $G$ un gruppo, e siano $\rho_i$ le sue rappresentazioni irriducibili. Allora G è abeliano se e solo se tutte le $\rho_i$ hanno grado 1.
		\end{myprop}
		
		\begin{proof}
			Supponiamo che $G$ sia abeliano. Allora ho $\abs G$ classi di coniugio e pertanto altrettante rappresentazioni irriducibili. Dato che $\sum_i n_i^2 = \abs G$ segue $n_i = 1$.
			
			Viceversa, supponiamo che $G$ sia un gruppo le cui rappresentazioni irriducibili hanno grado 1. Consideriamo una rappresentazione $\rho$ fedele (ossia iniettiva) di $G$ (ad esempio quella regolare), quindi possiamo decomporla in irriducibili di grado 1. Questo significa che esiste una base in cui tutte le $\rho_g$ sono diagonali, e pertanto commutano. Dunque $\rho_g\rho_h = \rho_h\rho_g \Rightarrow \rho_{gh} = \rho_{hg}$ e per l'iniettività di $\rho$ ho finito.
		\end{proof}
%%%%%%%%%%%%%%%%%%%%%%%%%%%%%%%%%%%%%%%%%%%%%%%%%% Non mi pare tu abbia mai definito fedele G.I. (TODO rispondermi)
%% Per adesso lascio così, poi magari cercherò un posto sensato per dare la definizione
	\subsection{Sottogruppi normali}
		Se il gruppo ha sottogruppi normali, possiamo ottenere gratuitamente delle rappresentazioni del gruppo conoscendo quelle di un gruppo più piccolo.
		
		\begin{myprop}
			Sia $N$ sottogruppo normale di $G$, e sia $\sigma$ una rappresentazione del quoziente $G/N$. Allora $\sigma$ si può estendere in modo naturale a rappresentazione $\rho$ di $G$, e in particolare le rappresentazioni del quoziente sono in relazione biunivoca con quelle di $G$ il cui $\Ker$ contiene $N$. Inoltre le irriducibili in $G/N$ sono irriducibili anche in $G$.
		\end{myprop}
		\begin{proof}
			Basta mandare mappare l'applicazione $\sigma_{gN}$ in $\rho_g$ assicurandosi che sia una buona definizione. E' simile alla corrispondenza biunivoca tra sottogruppi.
			Per quanto riguarda l'irriducibilità, basta vedere che, se $R\subset G$ è un insieme di rappresentanti delle classi laterali di $N$, allora 
			\[
				\frac{\abs{N}}{\abs{G}} \sum_{g\in R}\Tr \sigma_{gN}\conj{\Tr \sigma_{gN}} = \frac1{\abs G} \sum_{g\in R} \abs N \Tr \sigma_{gN}\conj{\Tr \sigma_{gN}} = \frac1{\abs G} \sum_{g\in G} \Tr\rho_g\conj{\Tr \rho_{g}}
			\]
			quindi i due caratteri hanno la stessa norma.
		\end{proof}
		
		\begin{myexample} [Tabella dei caratteri di $Q_8$]
		Voglio costrure la tabella dei caratteri di $Q_8$. Ho 5 classi di coniugio, quindi 5 rappresentazioni. Una è l'identità. Per trovarne altre 3, posso pensare che $\{1,-1\}$ è normale, e quozientando ottengo $\Cyc_2 \times \Cyc_2$, che ha 3 rappresentazioni irriducibili non banali (scelgo quale dei 3 sottogruppi mandare nell'identità, e il resto andrà in $-1$). Sollevandole in $Q_8$, ottengo $\rho_i,\rho_j,\rho_k$. L'ultima rappresentazione, di grado 2, la ottengo per ortogonalità.
		\[
			\begin{array}{|c|ccccc|}
			\hline
			Q_8    & \{1\} & \{-1\} & \{\pm i\} & \{\pm j\} & \{\pm k\} \\ \hline
			Id     &   1   &    1   &     1     &     1     &     1     \\ 
			\rho_i &   1   &    1   &     1     &    -1     &    -1     \\
			\rho_j &   1   &    1   &    -1     &     1     &    -1     \\
			\rho_k &   1   &    1   &    -1     &    -1     &     1     \\
			\rho_2 &   2   &   -2   &     0     &     0     &     0     \\ \hline
			\end{array}
		\]
		
		
		\end{myexample}


	
	\subsection{Prodotto diretto}
	
	Vogliamo ora indagare come sono fatte le rappresentazioni di un prodotto diretto di gruppi, a partire dalle rappresentazioni sui singoli gruppi.
	
	Date due rappresentazioni $\rho$ di $G_1$, $\sigma$ di $G_2$, possiamo generalizzare il prodotto di rappresentazioni.
	\begin{mydef}\label{def:TensProdRepr}
		Si definisce il \myname{prodotto tensoriale} di rappresentazioni la rappresentazione di $G_1\times G_2$ data da
		\[
		(\rho \tensor \sigma)(g_1,g_2) = \rho_{g_1} \tensor \rho_{g_2} 
		\]
	\end{mydef}

	\begin{myprop}
		Il carattere del prodotto tensoriale vale
		\[
		\chi_{\rho\tensor\sigma}(g_1,g_2) = \chi_\rho(g_1)\chi_\sigma(g_2)
		\] 
	\end{myprop}

	\begin{proof}
		Per le proprietà del prodotto tensore di spazi vettoriali, vale $\Tr(\rho_{g_1}\tensor \sigma_{g_2})= \Tr{\rho_{g_1}}\Tr{\sigma_{g_2}}$
	\end{proof}
	
	Abbiamo quindi ottenuto una rappresentazione di $G_1\times G_2$, ci chiediamo quando è irriducibile. 
	\begin{myprop}
		Le rappresentazioni $\sigma$ e $\rho$ sono irriducibili se e solo se $\rho \tensor \sigma$ è irriducibile.
		
		Inoltre tutte le rappresentazioni irriducibili di $G_1\times G_2$ sono della forma $\rho \tensor \sigma$, con $\rho,\sigma$ irriducibili.
	\end{myprop}
	\begin{proof}
		Facciamo il conto con il carattere di $\rho \tensor \sigma$.
		\begin{align*}
		\herm{\chi_{\rho\tensor\sigma}}{\chi_{\rho\tensor\sigma}} &= \frac1{\abs {G_1}\abs{G_2}}\sum_{g_1 \in G_1}\sum_{g_2\in G_2} \chi_\rho(g_1)\chi_\sigma(g_2)\conj{\chi_\rho(g_1)\chi_\sigma(g_2)} = \\
		&= \frac1{\abs {G_1}}\sum_{g_1 \in G_1} \chi_\rho(g_1)\conj{\chi_\rho(g_1)}\cdot\frac1{\abs {G_2}}\sum_{g_2\in G_2}\chi_\sigma(g_2)\conj{\chi_\sigma(g_2)}\\
		&= \herm{\chi_\rho}{\chi_\rho}\herm{\chi_\sigma}{\chi_\sigma}
		\end{align*}
		e ora si ha la tesi per il criterio di irriducibilità.
		
		Per il secondo punto invece, mostriamo che un qualsiasi carattere $f$ (funzione di classe) su $G_1\times G_2$, ortogonale a tutti i caratteri della forma $\chi_\rho(g_1)\chi_\sigma(g_2)$, è nullo. Quindi sia
		\[
		\sum_{g_1,g_2} f(g_1,g_2) \conj{\chi_\rho(g_1),\chi_\sigma(g_2)}=0
		\]
		
		Fissiamo $\sigma$, e sia $h(g_1)=\sum_{g_2} f(g_1,g_2)\conj{\chi_\sigma(g_2)}$. Allora
		\[
		\sum_{g_1}h(g_1)\conj{\chi_\rho(g_1)}=0 \qquad\qquad \forall \rho
		\]
		Poichè deve valere per ogni rappresentazione $\rho$, e $h$ è una funzione classe, allora deve essere $h\equiv 0$. Analogamente si conclude che $f \equiv 0$.
	\end{proof}
	
	\subsection{Rappresentazione indotta}
		Sia $H<G$ un sottogruppo. Supponiamo di avere una $H$-rappresentazione $\sigma$, su uno spazio vettoriale $W$. Noi vorremmo costruire una $G$-rappresentazione. Iniziamo con una definizione.
		
		\begin{mydef}
		 Supponiamo di avere una $G$-rappresentazione $\rho$ su $V$, e sia $W$ un suo sottospazio $H$-invariante (ossia su $W$ abbiamo una $H$-rappresentazione $\sigma$). Chiamiamo $G/H$ l'insieme delle classi laterali sinistre
		 
		 La rappresentazione $\rho$ si dice indotta da $\sigma$ se 
		 \[
		  V = \bigoplus_{\bar g \in G/H} \rho_g'(W) 
		 \]
		 dove si intende che $g'$ è un rappresentante di $\bar g$.
		\end{mydef}
		
		L'idea è, dato uno spazio $W$ con una $H$-rappresentazione sopra, di costruire $V$ come somma diretta di tante copie di $W$ indicizzate dai $\bar g$, che indicheremo come $W_{\bar g}$, e dato un vettore $w\in W$, indicheremo con $w_{\bar g}$ la sua copia in $W_{\bar g}$. 
		
		Vediamo ora come deve agire $G$. Consideriamo un elemento $g\in G$, che possiamo scrivere come $g'h$ con $h\in H$, e definiamo
		\[
			\rho_g(w_{\bar s}) = \rho_{g'h}(w_{\bar s})= \sigma_h(w_{\bar{g's}})
		\]
		
		Detto a parole, io mando $W_{\bar s}$ nella sua copia $W_{\bar {g's}}$, e poi all'interno faccio agire $h$ tramite $\sigma$. 
		
		Rimarrebbe da verificare che questa è una buona definizione, e che due rappresentazioni indotte dalla stessa rappresentazione sono isomorfe, e questo viene lasciato come esercizio al volenteroso lettore.
		
		D'ora in poi, data una $H$-rappresentazione $\rho$, indicheremo come $\Ind_H^G(\rho)$ la sua indotta.
		
		Per le rappresentazioni indotte vale il seguente importante teorema.
		\begin{mytheorem}[di reciprocità di Frobenius]
			Sia $\rho$ una $H$-rappresentazione. Sia $\sigma$ una rappresentazione qualunque di $G$, e sia inoltre $\sigma_H$ la sua restrizione a $H$-rappresentazione (ossia $\bC[G]$ con l'azione di $H$). Allora ogni omomorfismo di rappresentazione $\phi$ da $\rho$ a $\sigma_H$ si può estendere in modo unico a un omomorfismo da $\tilde\phi$ da $\Ind_H^G(\rho)$ a $\bC[G]$. In particolare vale quindi
			\[
				\Hom_H(\rho, \sigma_H) \isom \Hom_G(\Ind_H^G(\rho), \sigma)
			\]
		\end{mytheorem}
		\begin{proof}
			Prendiamo $\phi \in \Hom(\rho, \restr{\bC[G]}{H})$. Allora deve valere 
			 \[
				\tilde\phi (w_{\bar g}) = \tilde\phi(g \cdot w_{\bar e}) = g \cdot \tilde\phi(w) = g\cdot \phi (w)			  
			 \]
			dove la prima uguaglianza vale per definizione, la seconda per le proprietà di omomorfismo di rappresentazioni e la terza perchè deve essere un'estensione (sto identificando ovviamente $W$ con $W_{\bar e})$. 
		\end{proof}
		
		Un utile corollario è il seguente.
		\begin{mycor}
		 Vale 
		 \[
		  \herm {\chi_\rho}{\chi_{\sigma_H}} = \herm {\chi_{\Ind_H^G(\rho)}}{\chi_\sigma}
		 \]

		\end{mycor}
		\begin{proof}
			Ovvio per il fatto che $\herm{\chi_\rho}{\chi_\sigma}=\dim \Hom (\rho,\sigma)$.
		\end{proof}
		
		Questo corollario serve per calcolare la composizione della rappresentazione indotta senza conoscerne il carattere.
		

	
	\subsection{Rappresentazioni di $S_n$}
		Indaghiamo le rappresentazioni del gruppo simmetrico $S_n$, che è complicato dato che ha pochi sottogruppi normali.
		
		\begin{myprop}
		Le uniche rappresentazioni di grado 1 di $S_n$ sono quella banale e il segno.
		\end{myprop}
		\begin{proof}
		Dato che devo immergere $S_n$ in $\bC^*$ che è abeliano, allora i commutatori devono stare nel $\Ker$. Dato che i commutatori sono $A_n$, le uniche possibilità sono $\sigma \cdot v = v$ o $\sigma \cdot v=\sgn(\sigma)v$.
		\end{proof}
		
		Una rappresentazione naturale per $S_n$ è quella data da $\bC^n$ dove $\sigma \cdot v$ permuta le componenti di $v$. Chiaramente il sottospazio $\Span(1,1,\dots,1)$ è $G$-invariante. Vediamo che la sottorappresentazione supplementare (che chiameremo $\std$) è irriducibile.
		
		\begin{myprop}
		La rappresentazione per permutazioni di $\bC^n$ è somma di due irriducibili.
		\end{myprop}
		\begin{proof}
		Calcoliamo la norma del carattere. Sia quindi $[\sigma]$ la matrice associata all'azione di $\sigma$ rispetto alla base canonica.
		Dobbiamo quindi calcolare
		\[
			\frac1{n!}\sum_{\sigma\in S_n} \left(\sum_i [\sigma]_i^i\right)\conj{\left(\sum_i[\sigma]_i^i\right)}
		\]
		Dato che $[\sigma]$ è una matrice di permutazione, in particolare è reale e possiamo scordarci di coniugare. Apro il prodotto e ottengo
		\[
			\frac1{n!}\sum_{\sigma\in S_n} \sum_i \sum_j [\sigma]_i^i[\sigma]_j^j
		\]
		
		Adesso posso fare \emph{double-counting} (o forse dovremmo dire \emph{triple-counting}) per scambiare i simboli di sommatoria e ottenere
		\[
			\frac1{n!}\sum_i \sum_j\sum_{\sigma\in S_n} [\sigma]_i^i[\sigma]_j^j      
		\]
		A $i,j$ fissati, $\sum_{\sigma\in S_n} [\sigma]_i^i[\sigma]_j^j$ conta le permutazioni che fissano sia $i$ che $j$.
		
		Quindi divido nei due casi $i=j$ e $i\ne j$ e ottengo
		\[
			\frac1{n!}\sum_i (n-1)! +  \frac1{n!}\sum_{i\ne j} (n-2)! = 2
		\]
		
		L'unica possibilità è dunque che la rappresentazione sia somma di due irriducibili distinte.
		\end{proof}
	





