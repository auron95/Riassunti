\documentclass[a4paper,10pt,oneside]{math_article}

\renewcommand{\phi}{\varphi}
%\renewcommand{\Phi}{\varPhi}
\newcommand{\herm}[2]{\left(#1 | #2\right)}
\newcommand{\id}{I}
\newcommand{\tensor}{\otimes}
\newcommand{\embed}{\hookrightarrow}
\renewcommand{\bar}{\overline}
\renewcommand{\Re}{\operatorname{Re}}
\renewcommand{\Im}{\operatorname{Im}}

\newcommand{\twovec}[2]{\begin{pmatrix}#1\\#2\end{pmatrix}}

\newcommand{\cmat}[2]{\begin{pmatrix}#1&-#2\\#2&#1\end{pmatrix}}
\newcommand{\cmatconj}[2]{\begin{pmatrix}#1&#2\\-#2&#1\end{pmatrix}}
\newcommand{\scmat}[2]{\left(\begin{smallmatrix}#1&-#2\\#2&#1\end{smallmatrix}\right)}
\let\conj\overline
\title{Sistemi dinamici}
 \author{Matteo Migliorini}
 
\date{}
 
 
\begin{document}
	\section{Definizioni generali}
	
	Sia $W \subseteq \bR^n$ un aperto, e $F: W \rar \bR^n$ una funzione di classe $C^1$. Il generico sistema dinamico (continuo) è un'equazione del tipo
	\[
		\dot X = F(X)
	\]
	dove $X: I \rar W$ è una curva definita su un certo intervallo $I$.
	
	\begin{mydef}[Orbita]
		Ogni curva definita su tutto $\bR$ che soddisfa l'equazione si dice orbita del sistema dinamico. Se è definita solo su un intervallo, si parla genericamente di soluzione. Si parla invece di traiettoria di una soluzione per indicare la sua immagine in $\bR^n$.
	\end{mydef}
	
	\begin{mytheorem}[Cauchy-Lipschitz]
		Nelle ipotesi sopra, esiste un unica soluzione definita localmente per un certa condizione iniziale $X(t_0)=X_0$.
	\end{mytheorem}

	
	\begin{mydef}[Flusso integrale]
		Sia $\gamma_{X}$ la soluzione con condizione iniziale $\gamma_X(0) = X$.
		Si dice \emph{flusso integrale} la famiglia di applicazioni $\{ \Phi^t: t \in \bR\}$ tali che $\Phi^t: W\rar W$ e che mandano ogni $X \in W$ in $\gamma_{X}(t)$.
	\end{mydef}
	
	\begin{myobs}
		Non è detto che il flusso integrale sia definito per tutti i tempi.
	\end{myobs}
	
	\begin{mydef}
		Un sottoinsieme $P$ si dice \emph{invariante} per il flusso se ogni soluzione con condizione iniziale in $P$ è definita globalmente e rimane in $P$ per tutti i tempi.
	\end{mydef}

	
	
	\begin{mydef}[Integrale primo]
		Una funzione $E: W \rar R^n$ è una funzione differenziabile costante sulle soluzioni.
	\end{mydef}

	\begin{myobs}
		Ogni insieme di livello di $E$ è invariante.
	\end{myobs}
	
	
	\begin{mydef}
		Si dice che un sistema dinamico è \emph{conservativo} se ogni $\Phi^t$ conserva l'area orientata.
	\end{mydef}
	
	\subsection{Stabilità}
	\begin{mydef}[Punto di equilibrio]
		Un punto $s \in W$ si dice di \emph{equilibrio} se soddisfa $F(s)=0$. In questo caso la curva costante è un'orbita.
	\end{mydef}
	
	Cerchiamo di classificare il comportamento intorno ai punti di equilibrio.
	
	\begin{mydef}[Punto attrattivo]
		Un punto $s$ di equilibrio si dice attrattivo se ammette un intorno $U$ tale che le soluzioni con condizione iniziale in $U$ tendono a $s$ (nel futuro).
	\end{mydef}
	
	\begin{mydef}[Bacino di attrazione]
		Si dice \emph{bacino di attrazione} di un punto di equilibrio $s$ l'insieme dei punti $X$ per cui la soluzione con condizione iniziale in $X$ ha $s$ come limite.
	\end{mydef}
	
	\begin{myobs}
		Osserviamo che $s$ è attrattivo sta nella parte interna del suo bacino di attrazione. 
	\end{myobs}
	
	\begin{mydef}[Stabilità]
		Un punto $s$ si dice stabile se per ogni $U$ intorno di $s$ esiste un intorno $V$ per cui $\Phi^t(V) \subseteq U$ per ogni $t>0$, ossia se a patto di partire sufficientemente vicino, rimango a distanza arbitrariamente piccola da $s$. 
	\end{mydef}
	
	\begin{myobs}
		Attrattivo non implica stabile: moralmente le soluzioni potrebbero allontanarsi e poi tornare per tendere a $s$.
	\end{myobs}
	
	\begin{mydef}
		Un punto \emph{asintoticamente stabile} è un punto stabile e attrattivo.
	\end{mydef}
	
	\begin{mydef}
		Un punto si dice \emph{instabile} se non è stabile (chi l'avrebbe mai detto).
	\end{mydef}




\section{Sistemi Lineari}
	\subsection{Esponenziale di matrici}
	\begin{mydef}
		Si indica con $e^{A}$, con $A$ matrice quadrata, la serie
		\[
			\sum_{k=0}^{+\infty} \frac{A^kt^k}{k!}
		\]
		In particolare, useremo spesso la funzione $e^{At}$, dove $A$ è una matrice e $t\in \bR$.
	\end{mydef}
	
	\begin{mytheorem}[Convergenza dell'esponenziale]
		Sia $A$ una matrice quadrata, e $t \in \bR$. Allora
		\begin{itemize}
		 \item $e^{At}$ converge uniformemente sui compatti per ogni $t$.
		 \item $e^{At}$ ha come limite una funzione lineare i cui coefficienti sono funzioni continue della $t$.
		\end{itemize}

	\end{mytheorem}
	
	\begin{myprop}[Propretà dell'esponenziale]
		Valgono le seguenti:
		\begin{itemize} 
		 \item Se $AB=BA$, allora $e^A \cdot e^B = e^{A+B}$.
		 \item $e^{At}$ è derivabile e vale $\dede t e^{At} = Ae^{At}$
		\end{itemize}
	\end{myprop}
	
	
	\subsection{Diagonalizzazione}
		\begin{myprop}
			Sia $A=MDM\inv$ una matrice diagonalizzabile, con autovalori $\lambda_i$ e autovettori $v_i$ che formano una base. Allora
			\[
				M\inv = \left(
					\begin{array}{c|c|c|c}
						& & & \\
						v_1 & v_2 & \dots & v_n \\
						& & & \\
					\end{array}
				\right)
			\]
			e ovviamente
			\[
				D=\begin{pmatrix}
					\lambda_1 \\
					& \lambda_2 \\
					&& \ddots \\
					&&& \lambda_n
				  \end{pmatrix}
			\]
		\end{myprop}
		
		\begin{myprop}
			Se una matrice è diagonalizzabile, con $A=MDM\inv$, allora
			\[
				e^{At} = Me^{Dt}M\inv = M\begin{pmatrix}
					e^{\lambda_1} \\
					& e^{\lambda_2} \\
					&& \ddots \\
					&&& e^{\lambda_n}
				  \end{pmatrix}
			\]

		\end{myprop}
	
	\subsection{Diagonalizzazione complessa}
		\subsection{Forma matriciale dei numeri complessi}
		
		Consideriamo $J=\cmat 01$. Allora $\Span(I,J) \isom \bC$, con l'ovvio isomorfismo di anelli.
		
		In particolare vale che 
		\[
			\cmat ab \twovec cd = \twovec {ac-bd}{ad+bc} 
		\]

		ossia, se pensiamo i numeri complessi come vettori in 2-dimensionali reali, la moltiplicazione per numeri complessi funziona bene.
		
		Inoltre $\det \scmat ab = \abs{a+bi}^2$, mentre il coniugato si traduce nella trasposizione.
		
		Passando in polari, il numero complesso $\rho e^{i\theta}$ è rappesentato dalla matrice
		\[
			\cmat {\rho\cos \theta}{\rho\sin \theta}
		\]
		
		Per quanto riguarda l'esponenziale, $e^{a+bi}$ diventa
		\[
			\cmat {e^{at} \cos(bt)}{e^{at} \sin(bt)}
		\]

		

		
		\begin{mydef}[Matrice semisemplice]
			Una matrice $A$ si dice semisemplice se è diagonalizzabile in senso complesso.
		\end{mydef}
		
		Possiamo ora portare una matrice semisemplice in una forma normale.
		Consideriamo $A$ semisemplice, e siano $\lambda_i$ gli autovalori, e $v_i$ i relativi autovettori (in generale complessi).
		Poichè $A$ è reale, allora gli autovettori relativi a autovalori reali sono reali. Per quanto riguarda gli autovalori complessi, essi sono a due a due coniugati, pertanto si possono scegliere anche i relativi autovettori in modo che siano anch'essi coniugati. A questo punto, dati due autovalori complessi $\lambda_i$ e $\lambda_{i+1}=\conj{\lambda_i}$, sostituiamo $v_i$ e $v_{i+1}$ con $\Im v_i$ e $\Re v_i$ rispettivamente. 
		
		A questo punto passando a questa nuova base ho che $A$ è della forma
		
		\[
			A= MDM\inv = M \begin{pmatrix}
		                 \lambda_1 \\
		                 & \ddots \\
		                 && \Re{\lambda_i} & -\Im{\lambda_i} \\
		                 && \Im{\lambda_i} & \Re{\lambda_i} \\
		                 &&&& \ddots
		                \end{pmatrix}
		\]
		
		Cioè, scelgo un autovalore per ogni coppia, ci metto nella base di autovettori parte immaginaria e reale nell'ordine, e la matrice viene diagonale, solo che ho anche le rappresentazioni algebriche degli autovalori complessi selezionati.
		
		
		

		
		

	
	
	\subsection{Sistema lineare}
		\begin{mydef}
			Un sistema dinamico continuo si dice lineare se $F$ è una funzione lineare da $\bR^n$ in sè.
		\end{mydef}
		
		\begin{myprop}
			Sia $\dot X = AX$ un sistema lineare. Allora 
			\[
				\Phi^t(X) = e^{At}X
			\]
		\end{myprop}
		
		Supponiamo per il momento $A$ semisemplice.
		
		Per risolvere il sistema dinamico lineare conviene quindi diagonalizzare $A$. A questo punto, nella nuova base, il comportamento qualitativo si può classificare facilmente.
		
		Mettiamoci quindi in dimensione 2 e studiamo qualitativamente il comportamento, e diamo definizioni diverse del tipo di equilibrio che ha l'origine (che hanno senso solo nel caso lineare).
		\begin{mydef}
		L'origine si dice:
		\begin{itemize}
		 \item nodo se gli autovalori sono reali concordi e distinti;
		 \item sella se gli autovalori sono reali discordi;
		 \item fuoco se gli autovalori sono complessi con parte reale nonnulla;
		 \item centro se gli autovalori sono immaginari puri.
		\end{itemize}
		\end{mydef}

		
		
	







 

\end{document}
