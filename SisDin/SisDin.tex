\documentclass[a4paper,10pt,oneside]{math_article}

\renewcommand{\phi}{\varphi}
%\renewcommand{\Phi}{\varPhi}
\newcommand{\herm}[2]{\left(#1 | #2\right)}
\newcommand{\id}{I}
\newcommand{\tensor}{\otimes}
\newcommand{\embed}{\hookrightarrow}
\renewcommand{\bar}{\overline}
\renewcommand{\Re}{\operatorname{Re}}


\let\conj\overline
\title{Riassunto di Introduzione alla Teoria delle Rappresentazioni}
 \author{Matteo Migliorini}
 
\date{}
 
 
\begin{document}
	\section{Definizioni generali}
	
	Sia $W \subseteq \bR^n$ un aperto, e $F: W \rar \bR^n$ una funzione di classe $C^1$. Il generico sistema dinamico (continuo) è un'equazione del tipo
	\[
		\dot X = F(X)
	\]
	dove $X: I \rar W$ è una curva definita su un certo intervallo $I$.
	
	\begin{mydef}[Orbita]
		Ogni curva definita su tutto $\bR$ che soddisfa l'equazione si dice orbita del sistema dinamico. Se è definita solo su un intervallo, si parla genericamente di soluzione.
	\end{mydef}
	
	\begin{mytheorem}[Cauchy-Lipschitz]
		Nelle ipotesi sopra, esiste un unica soluzione definita localmente per un certa condizione iniziale $X(t_0)=X_0$.
	\end{mytheorem}

	
	\begin{mydef}[Flusso integrale]
		Sia $\gamma_{X}$ la soluzione con condizione iniziale $\gamma_X(0) = X$.
		Si dice \emph{flusso integrale} la famiglia di applicazioni $\{ \Phi^t: t \in \bR\}$ tali che $\Phi^t: W\rar W$ e che mandano ogni $X \in W$ in $\gamma_{X}(t)$.
	\end{mydef}
	
	\begin{myobs}
		Non è detto che il flusso integrale sia definito per tutti i tempi.
	\end{myobs}
	
	\begin{mydef}
		Un sottoinsieme $P$ si dice \emph{invariante} per il flusso se ogni soluzione con condizione iniziale in $P$ è definita globalmente e rimane in $P$ per tutti i tempi.
	\end{mydef}

	
	
	\begin{mydef}[Integrale primo]
		Una funzione $E: W \rar R^n$ è una funzione differenziabile costante sulle soluzioni.
	\end{mydef}

	\begin{myobs}
		Ogni insieme di livello di $E$ è invariante.
	\end{myobs}
	
	
	\begin{mydef}
		Si dice che un sistema dinamico è \emph{conservativo} se ogni $\Phi^t$ conserva l'area orientata.
	\end{mydef}
	
	\subsection{Stabilità}
	\begin{mydef}[Punto di equilibrio]
		Un punto $s \in W$ si dice di \emph{equilibrio} se soddisfa $F(s)=0$. In questo caso la curva costante è un'orbita.
	\end{mydef}
	
	Cerchiamo di classificare il comportamento intorno ai punti di equilibrio.
	
	\begin{mydef}[Punto attrattivo]
		Un punto $s$ di equilibrio si dice attrattivo se ammette un intorno $U$ tale che le soluzioni con condizione iniziale in $U$ tendono a $s$ (nel futuro).
	\end{mydef}
	
	\begin{mydef}[Bacino di attrazione]
		Si dice \emph{bacino di attrazione} di un punto di equilibrio $s$ l'insieme dei punti $X$ per cui la soluzione con condizione iniziale in $X$ ha $s$ come limite.
	\end{mydef}
	
	\begin{myobs}
		Osserviamo che $s$ è attrattivo sta nella parte interna del suo bacino di attrazione. 
	\end{myobs}
	
	\begin{mydef}[Stabilità]
		Un punto $s$ si dice stabile se per ogni $U$ intorno di $s$ esiste un intorno $V$ per cui $\Phi^t(V) \subseteq U$ per ogni $t>0$, ossia se a patto di partire sufficientemente vicino, rimango a distanza arbitrariamente piccola da $s$. 
	\end{mydef}
	
	\begin{myobs}
		Attrattivo non implica stabile: moralmente le soluzioni potrebbero allontanarsi e poi tornare per tendere a $s$.
	\end{myobs}
	
	\begin{mydef}
		Un punto \emph{asintoticamente stabile} è un punto stabile e attrattivo.
	\end{mydef}
	
	\begin{mydef}
		Un punto si dice \emph{instabile} se non è stabile (chi l'avrebbe mai detto).
	\end{mydef}




\section{Sistemi Lineari}
	\subsection{Esponenziale di matrici}




 

\end{document}
